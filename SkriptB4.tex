\documentclass[
twoside=semi,
fontsize=12,
DIV=12, 
cleardoublepage=current,
leqno,
headings=optiontoheadandtoc, 
toc=idx
]{scrbook}

\usepackage{imakeidx}
\makeindex

\usepackage[german]{babel}
\usepackage[utf8]{inputenc}
\usepackage[T1]{fontenc}

\usepackage{datetime}


\usepackage[headsepline]{scrlayer-scrpage}
\setkomafont{pageheadfoot}{\normalfont}
\setkomafont{pagefoot}{\slshape}

\defpagestyle{scrheadings}{%
	{}{}{}
}{
	{\pagemark\hfill}{\hfill \pagemark}{}
}

\defpagestyle{plain}{%
	{}{}{}(0pt,0pt)
}{
	{\pagemark \hfill}{\hfill \pagemark}{}
}

\usepackage{chngcntr}\counterwithout{equation}{section}

\usepackage{amsmath}
\usepackage{amssymb}
\usepackage{amsthm}
\usepackage{stmaryrd}
\usepackage{enumerate}
\usepackage[pass]{geometry}
\usepackage{csquotes}


\usepackage{hyperref}
\MakeOuterQuote{"}

\newcommand{\N}{\mathbb{N}}
\newcommand{\Z}{\mathbb{Z}}
\newcommand{\Q}{\mathbb{Q}}
\newcommand{\R}{\mathbb{R}}
\newcommand{\F}{\mathbb{F}}

\newcommand{\comment}[1]{}

\newcommand{\brac}[1]{\left( #1 \right)}
\newcommand{\bracB}[1]{\left[ #1 \right]}
\newcommand{\bracC}[1]{\left< #1 \right>}
\newcommand{\abs}[1]{\left| #1 \right|}
\newcommand{\set}[1]{\left\{ #1 \right\}}

\newcommand*{\ORIGchapterheadstartvskip}{}%
\let\ORIGchapterheadstartvskip=\chapterheadstartvskip
\renewcommand*{\chapterheadstartvskip}{%
	\ORIGchapterheadstartvskip
	\noindent\rule[\baselineskip]{\linewidth}{4pt}\par
}
\newcommand*{\ORIGchapterheadendvskip}{}%
\let\ORIGchapterheadendvskip=\chapterheadendvskip
\renewcommand*{\chapterheadendvskip}{%
	\ORIGchapterheadendvskip
	\noindent\rule[\baselineskip]{\linewidth}{4pt}\par
}

\DeclareMathOperator{\im}{im}
\DeclareMathOperator{\supp}{supp}
\DeclareMathOperator{\ann}{ann}
\DeclareMathOperator{\id}{id}
\DeclareMathOperator{\rk}{rk}

\theoremstyle{definition}
\newtheorem{definition}{Definition}[section]
\newtheorem{bemerkung}[definition]{Bemerkung}
\newtheorem{beispiele}[definition]{Beispiele}
\newtheorem{warnung}[definition]{Warnung}
\newtheorem{satz}[definition]{Satz}
\newtheorem{lemma}[definition]{Lemma}
\newtheorem{proposition}[definition]{Proposition}
\newtheorem{notation}[definition]{Notation}
\newtheorem{prop-def}[definition]{Proposition und Definition}
\newtheorem{def-prop-satz-not}[definition]{Definitionen, Propositionen, S\"atze und Notationen}


\begin{document}
	\pagenumbering{Alph}
	\pagestyle{empty}
	\tableofcontents
	\mainmatter
	\chapter[tocentry={Moduln}]{Moduln}
	\pagestyle{scrheadings}
	
	\section{Definitionen und grundlegende Tatsachen}
	\setcounter{chapter}{1}
	\setcounter{section}{1}
	\begin{definition}
		\label{1.1.1}
		
		Ein \emph{Modul}\index{Modul@\textbf{Modul}} ist ein Tupel $(R, +_R, \cdot_R, M, +, \cdot)$, wobei $(R, +_R, \cdot_R)$ ein Ring (mit $1$, nicht notwendigerweise kommutativ), $(M, +)$ eine abelsche Gruppe und \\\noindent$\cdot:R\times M \to M$ eine (meist gar nicht oder infix geschriebene) Abbildung mit folgenden Eigenschaften
		
		\begin{itemize}
			\item[$(\overset{\rightarrow}{D})$] $\forall a \in R: \forall x, y \in M: a(x + y) = ax + ay$ \hfill "distributiv"
			
			\item[$(D')$] $\forall a, b \in R: \forall x \in M: (a+b)x = ax + bx$ \hfill "distributiv"
			
			\item[$(N)$] $\forall x \in M: 1_R \cdot x = x$ \hfill "normiert"
			
			\item[$(V)$] $\forall a, b \in R: \forall x \in M: (ab)x = a(bx)$ \hfill "vertr\"aglich"
		\end{itemize}
	\end{definition}
	
	
	\begin{bemerkung}\label{1.1.2}
			
			\begin{enumerate}[(a)]
			\item Schlampiger Sprachgebrauch: 
			\begin{itemize}
				\item "Sei $M$ ein $R$-Modul" statt "Sei $(R, +_R, \cdot_R, M, +, \cdot)$ ein Modul"
				
				\item "Sei $M$ ein Modul" statt "Es gebe einen Ring $R$ so, dass $M$ ein $R$-Modul ist"
			\end{itemize}
			
			\item Statt "$R$-Modul" sagt man auch "Modul \"uber $R$"	
			
			\item Vektorr\"aume sind Moduln \"uber K\"orper. Viele Sprechweisen (wie "Skalar", "Linearkombination", nicht jedoch "Vektor") \"ubertragen wir stillschweigend von Vektorr\"aumen auf Moduln, ebenso 
			Konventionen (wie "Punkt vor Strich").
			
			\item Abelsche Gruppen "sind" $\Z$-Moduln. Sei $G$ eine abelsche Gruppe. Dann gibt es genau eine Skalarmultiplikation $\cdot:\Z\times G \to G$ verm\"oge derer $G$ zu einem $\Z$-Modul wird, n\"amlich die nat\"urliche, die durch 
			\[n \cdot a := \begin{cases}
				\underbrace{a + a + \cdots + a}_{n\textrm{-mal}} & \textrm{falls } n > 0\\
				0 & \textrm{falls } n = 0\\
				\underbrace{-a - a - \cdots - a}_{(-n)\textrm{-mal}} & \textrm{falls } n < 0
			\end{cases}\]
			gegeben ist.
			
			\item $(\overset{\rightarrow}{D})$ besagt, dass f\"ur alle $a \in R$ die Abbildung $M \to M, x \mapsto ax$ ein Gruppenhomomorphismus ist. Insbesondere gilt $a \cdot 0 = 0$ und $a \cdot (-x) = -ax$ f\"ur alle $a \in R, x \in M$.
			
			$(D')$ besagt, dass f\"ur alle $x \in M$ die Abbildung $R \to M, a \mapsto ax$ ein Gruppenhomomorphismus ist. Insbesondere gilt $0 \cdot x = 0$ und $(-a) \cdot x = -ax$ f\"ur alle $a \in R, x \in M$.
		\end{enumerate}
		
	\end{bemerkung}
	
	\begin{beispiele}\label{1.1.3}
			\begin{enumerate}[(a)]
			\item Nullmoduln $\set{0}$
			
			\item Sei $A$ ein Unterring des Ringes $B$. Dann ist $B$ ein $A$-Modul verm\"oge der Skalarmultiplikation $\cdot: A \times B \to B, (a, x) \mapsto ax$
			
			Insbesondere ist jeder Ring ein Modul \"uber sich selbst.
			
			\item Sei $R$ ein kommutativer Ring und $n \in \N_0$. Dann wird die abelsche Gruppe $R^n$ zu einem $R^{n\times n}$-Modul verm\"oge der Skalarmultiplikation
			\[\cdot: R^{n\times n} \times R^n \to R^n, (A, x) \mapsto Ax\]
			Dies folgt aus den Rechenregeln f\"ur Matrixmultiplikation.
		\end{enumerate}
	\end{beispiele}
	
	\begin{def-prop-satz-not}\label{1.1.4}
		Sei $R$ ein Ring. Die folgenden f\"ur die Theorie der $R$-Moduln grundlegenden Begriffe und Resultate sind eine direkte Verallgemeinerung der entsprechenden Tatsachen f\"ur Vektorr\"aume (also f\"ur den Fall, dass $R$ ein K\"orper) und f\"ur abelsche Gruppen (also $R = \Z$) aus der Linearen Algebra:
		
		\begin{enumerate}[(a)]
			\item Genauso wie bei Vektorr\"aumen f\"uhrt man \emph{direkte Produkte}\index{Modul@\textbf{Modul}!Direktes Produkt} von $R$-Moduln ein.
			
			\item Sind $M$ und $N$ $R$-Moduln, so hei\ss t $N$ ein \emph{Untermodul}\index{Modul@\textbf{Modul}!Untermodul} von $M$, wenn die $N$ zugrunde liegende abelsche Gruppe eine Untergruppe der $M$ zugrunde liegenden abelschen Gruppe ist und 
			\[\forall a \in R: \forall x \in M: a \cdot_N x = a \cdot_M x\]
			
			Ein Untermodul eines Moduls ist offenbar durch seine Tr\"agermenge (d.h. seine zugrunde liegende Menge) eindeutig bestimmt.
			
			Ist $M$ ein $R$-Modul und $N \subseteq M$, so ist $N$ offenbar genau dann (Tr\"agermenge) ein(e) Untermodul(s) von $M$, wenn $0\in N, \forall x, y\in N: x + y \in N, \forall a \in R: \forall x \in N: ax \in N$
			
			\item Sei $M$ ein Modul und $(N_i)_{i \in I}$ eine Familie von Untermoduln von $M$. Dann ist 
			$\bigcap_{i \in I} N_i := \bigcap \set{N_i | i \in I}$ (mit $\bigcap_{i \in I} N_i = M$, falls $I = \emptyset$)
			wieder ein Untermodul von $M$ und zwar der gr\"o\ss te Untermodul von $M$, der in allen $N_i$ enthalten ist.
			
			Weiter ist auch $\sum_{i \in I} N_i := \set{\sum_{i \in I} x_i | (x_i)_{i \in I} \in \prod_{i\in I} N_i, \set{i \in I| x_i \neq 0} \textrm{endlich}}$ Untermodul von $M$ und zwar der kleinste Untermodul von $M$, der alle $N_i$ enth\"alt.
			
			\item Sei $M$ ein $R$-Modul. Ist $x \in M$, so ist $Rx := \set{ax| a \in R}$ ein Untermodul von $M$ und zwar der kleinste Untermodul, der $x$ enth\"alt.
			
			Ist $(x_i)_{i \in I}$ eine Familie von Elementen von $M$, so ist $\sum_{i \in I} Rx_i$ der kleinste Untermodul von $M$, der alle $x_i$ enth\"alt.
			
			Man nennt ihn den von den $x_i$ ($i \in I$) (oder $\set{x_i|i \in I}$) erzeugten Untermodul von $M$ (oder lineare H\"ulle der Span von $\set{x_i| i \in I}$). 
			
			Man nennt $M$ \emph{zyklisch}\index{Modul@\textbf{Modul}!Zyklische Moduln}, wenn $M$ von einem Element erzeugt wird, d.h. es ein $x \in M$ gibt mit $M = Rx$. Man nennt $M$ endlich erzeugt (e.e.), wenn $M$ von endlich vielen Elementen
			erzeugt wird, d.h. es ein $n \in \N_0$ und $x_1, \dots, x_n \in M$ gibt mit 
			\[M = Rx_1 + \cdots + Rx_n := \sum_{i=1}^{n} Rx_i := \sum_{i \in \set{1, \dots, n}} Rx_i\] 
			
			\item Sei $M$ ein $R$-Modul. Eine Familie $(x_i)_{i\in I}$ in $M$ hei\ss t \emph{linear unabh\"angig}\index{Modul@\textbf{Modul}!Linear unabh\"angig (l.u.)} (l.u.), wenn f\"ur alle $n\in \N_0$, alle paarweise verschiedenen $i_1, \dots, i_n \in I$ und alle $a_1, \dots, a_n \in I$ gilt 
			\[\sum_{j=1}^{n} a_jx_{i_j} = 0 \Rightarrow a_1 = \cdots = a_n = 0\]
			
			Weiter nennt man $x_1, \dots, x_n$ linear unabh\"angig, wenn $(x_1, \dots, x_n) = (x_i)_{i \in \set{1, \dots, n}}$ linear unabh\"angig ist, d.h. f\"ur alle $a_1, \dots, a_n \in R$ gilt 
			\begin{align}
				a_1x_1 + \cdots a_nx_n = 0 \Rightarrow a_1 = \cdots = a_n = 0 \label{lin_ind}
			\end{align}
			
			Schlie\ss lich hei\ss t eine Menge $F\subseteq M$ linear unabh\"angig, wenn $(x)_{x \in F}$ linear unabh\"angig ist, d.h. f\"ur alle $n\in \N_0$, alle paarweise verschiedenen $x_1, \dots, x_n \in F$ und alle $a_1, \dots, a_n \in R$ wieder \ref{lin_ind} gilt.
			
			\item Sei $M$ ein Modul. Eine Familie $(x_i)_{i \in I}$ in $M$ hei\ss t eine \emph{Basis}\index{Modul@\textbf{Modul}!Basis} von $M$, wenn sie $M$ erzeugt und linear unabh\"angig ist. Weiter
			sagt man $x_1, \dots, x_n \in M$ bilden eine Basis von $M$, wenn $(x_1, \dots, x_n) = (x_i)_{i \in \set{1, \dots, n}}$ eine Basis von $M$ ist. Schlie\ss lich hei\ss t $B \subseteq M$ eine Basis, wenn $B$ den Modul $M$ erzeugt und linear unabh\"angig ist.
			
			\item Seien $M$ und $N$ $R$-Moduln. Dann hei\ss t $f$ ein \emph{($R$-)(Modul-)Homomorphismus}\index{Modul@\textbf{Modul}!Homomorphismus} oder eine $(R-)$ lineare Abbildung von $M$ nach $N$, wenn $f:M\to N$ ein Gruppenhomomorphismus der $M$ und $N$ zugrundeliegenden abelschen Gruppen ist und 
			\[\forall a \in R: \forall x \in M: f(ax) = af(x) \]
			
			Ein Modulhomomorphismus $f:M\to N$ hei\ss t Einbettung/Monomorphismus (Epimorphismus, Isomorphismus), wenn $f$ injektiv (surjektiv, bijektiv) ist. 
			
			Ein Modulhomomorphismus $f:M \to M$ hei\ss t \emph{(Modul-)Endomorphismus}\index{Modul@\textbf{Modul}!Homomorphismus!Endomorphismus} von $M$. Ein Endomorphismus, der ein Isomorphismus ist, hei\ss t \emph{Automorphismus}\index{Modul@\textbf{Modul}!Automorphismus}. Es heißen $M$ und $N$ \emph{isomorph}, in Zeichen $M \cong N$, wenn es einen Isomorphismus $M \to N$ gibt.
			
			Hintereinanderschaltungen von Modulhomomorphismen sind wieder Modulhomomorphismen. Umkehrabbildungen von Modulisomorphismen sind wieder Modulisomorphismen.
			
			\item Sei $M$ ein $R$-Modul. Eine \emph{Kongruenzrelation}\index{Modul@\textbf{Modul}!Kongruenzrelation} auf $M$ ist eine \"Aquivalenzrelation $\equiv$ der $M$ zugrundeliegenden Menge, f\"ur die gilt
			\[\forall x, y, x', y' \in M: (x \equiv x' \land y \equiv y') \Rightarrow x + y \equiv x' + y'\]
			und
			\[\forall x, x' \in M: \forall a \in R: x \equiv x' \Rightarrow ax \equiv ax'\]
			
			Diese Definition wurde gerade so gemacht, dass 
			\[+:( M/\equiv) \times (M/\equiv) \to (M/\equiv), (\overline{x}, \overline{y}) \mapsto \overline{x + y}\]
			und
			\[\cdot: R \times (M/\equiv) \to (M/\equiv), (a, \overline{x}) \mapsto \overline{ax}\]
			wohldefiniert sind.
			
			Ist $M$ ein $R$-Modul und $\equiv$ eine Kongruenzrelation auf $M$, so wird die Quotientenmenge $M/\equiv$ verm\"oge der Addition $+$ und der Skalarmultiplikation $\cdot$ ein $R$-Modul, wie man
			durch direktes Nachrechnen sieht. Die Zuordnungen
			\begin{align*}
				\equiv  &\overset{f}{\mapsto} \overline{0}\\
				\equiv_N &\overset{g}{\mapsfrom} N
			\end{align*}
			vermitteln eine Bijektion zwischen der Menge der Kongruenzrelationen auf $M$ und der Menge der Untermoduln von $M$, wobei $\equiv_N$ gegeben ist durch
			\[a \equiv_N b :\Leftrightarrow a - b \in N\]
			f\"ur $a, b \in M$.
			
			Ist $N$ ein Untermodul von $M$, so nennt man $M/N := M/\equiv_N$ auch den \emph{Quotientenmodul}\index{Modul@\textbf{Modul}!Quotientenmodul} von $M$ nach $N$.
			
			\item Sind $M$ und $N$ $R$-Moduln und $f:M \to N$ ein Modulhomomorphismus, so ist der \emph{Kern}\index{Modul@\textbf{Modul}!Homomorphismus!Kern} $\ker f := \set{x \in M| f(x) = 0}$ von $f$ ein Untermodul von $M$ und das \emph{Bild}\index{Modul@\textbf{Modul}!Homomorphismus!Bild} $\im f := \set{f(x) | x \in M}$ von $f$ ist ein Untermodul von $N$.
			
			\item Homomorphiesatz: Seien $M$ und $N$ $R$-Moduln und $L$ ein Untermodul von $M$ und $f:M \to N$ ein Modulhomomorphismus mit $L \subseteq \ker f$. Dann gibt es (genau) einen Modulhomomorphismus $\overline{f}: (M/L) \to N$ mit $\overline{f}(\overline{x}) = f(x)$ f\"ur alle $x \in M$.
			
			Ferner gilt, dass
			\begin{itemize}
				\item $\overline{f}$ ist injektiv $\Leftrightarrow L = \ker f$ und
				\item $\overline{f}$ ist surjektiv $\Leftrightarrow f$ ist surjektiv
			\end{itemize}
			
			\item Isomorphiesatz: Seien $M$ und $N$ $R$-Moduln und $f:M \to N$ ein Modulhomomorphismus. Dann ist $\overline{f}:(M/\ker f) \to \im f$ definiert durch $\overline{f}(\overline{x}) = f(x)$ f\"ur alle $x\in M$ ein $R$-Modulisomorphismus. Insbesondere ist $M/\ker f \cong \im f$
		\end{enumerate}
	\end{def-prop-satz-not}
	
	\begin{bemerkung}\label{1.1.5}
		Sei $R$ ein kommutativer Ring. Dann sind die Untermoduln des $R$-Modul $R$ [$\to$\ref{1.1.3}(b)] (oder kurz gesagt die $R$-Untermoduln von $R$) genau die Ideale des Ringes $R$.
		Insbesondere sind zum Beispiel das von einem $a \in R$ erzeugte Ideal und der davon erzeugte Untermodul als Menge dasselbe $(a)_R = Ra \overset{R\ komm.}{=} \set{ab | b \in R} = aR$.
		Trotzdem macht es vom Sinn her einen Unterschied. ob man $(a)$ oder $Ra$ schreibt. Zum Beispiel meint man mit $R/(a)$ den Ring und mit $R/aR$ den $R$-Modul (deren zugrundeliegenden abelschen Gruppen dieselben sind)
	\end{bemerkung}

	\begin{warnung}\label{1.1.6}
		F\"ur den mit Vektorr\"aumen, aber nicht mit Moduln vertrauten H\"orern ist Vorsicht geboten:
		\begin{enumerate}[(a)]
			\item In einem $R$-Modul $M$ kann $ax = 0$ f\"ur ein $a \in R$ und ein $x \in M$ gelten, ohne dass $a = 0$ oder $x = 0$ gilt (zum Beispiel $2 \cdot \overline{1} = \overline{2} = 0$ im $\Z$-Modul $\Z/2\Z$)
			
			\item Nicht jeder Modul hat eine Basis:
			zum Beispiel ist jedes Element des $\Z$-Moduls $\Z/2\Z$ linear abh\"angig, denn $1 \cdot \overline{0} = \overline{0} = 0$ und $2 \cdot \overline{1} = \overline{2} = 0$ in $\Z/2\Z$, womit die einzige linear unabh\"angige Teilmenge von $\Z/2\Z$ die leere Menge is, welche aber $\Z/2\Z$ nicht erzeugt.
		\end{enumerate}
	\end{warnung}	

	\begin{beispiele}\label{1.1.7}
		\begin{enumerate}[(a)]
			\item F\"ur jeden Ring $R$ ist $R^n$ ein $R$-Modul mit der \emph{Standardbasis}\index{Modul@\textbf{Modul}!Standardbasis} $\underline{e} = (e_1, \dots, e_n)$, wobei $e_i := \begin{pmatrix}
				0\\
				\vdots\\
				0\\
				1\\
				0\\
				\vdots\\
				0
			\end{pmatrix}$ mit einer $1$ an der $i$-ten Stelle.
		
			\item $\R^2$ ist ein zyklischer $\R^{2 \times 2}$ Modul [$\to$\ref{1.1.3}(c)], welcher von jedem $x \in \R^{2\times 2} \setminus \set{0}$ erzeugt ist. Da aber jedes $x \in \R^{2 \times 2}$ linear abh\"angig ist, hat dieser Modul keine Basis.  
		\end{enumerate}
	\end{beispiele}
	\newpage

	\section{Direkte Summen von Moduln und freie Moduln}
	\begin{definition}\label{1.2.1}
		Sei $R$ ein Ring und $(M_i)_{i \in I}$ eine Familie von $R$-Moduln. Dann nennt man den $R$-Untermodul 
			\[\bigoplus_{i \in I} M_i := \set{x \in \prod_{i\in I}M_i| \supp(x)\ \mathrm{endlich} } \]
		von $\displaystyle\prod_{i\in I} M_i$ die \emph{(\"au\ss ere) direkte Summe}\index{Modul@\textbf{Modul}!\"Au\ss ere Direkte Summe} der $M_i$ ($i \in I$). Man fasst $M_j$ ($j \in I$ h\"aufig) als Untermodul von $\displaystyle\bigoplus_{i \in I} M_i$ auf verm\"oge der Einbettung 
		 \[\rho_j:M_j \to \prod_{i\in I} M_i, x \mapsto \brac{i \mapsto \begin{cases}
		 		x & \textrm{falls } i = j\\
		 		0 & \textrm{sonst}
		 \end{cases}}\] 
	 	
	 	\noindent Ist $M_i = M$ f\"ur alle $i \in I$, so schreibt man \[M^{(I)} := \bigoplus_{i \in I} M \subseteq \prod_{i\in I} M = M^I\]
	\end{definition}
	
	\begin{proposition}\label{1.2.2}
		Sei $R$ ein Ring, $(M_i)_{i \in I}$ eine Familie von Modulhomomorphismen $f_i: M_i \to N$. Dann gibt es genau einen Modulhomomorphismus $f:\bigoplus_{i \in I} M_i \to N$ mit $f\big|_{M_i} = f_i$ f\"ur alle $i \in I$ ($f \circ \rho_i = f_i$ f\"ur $i \in I$).
	\end{proposition}
	\begin{proof}
		F\"ur jedes $x \in \bigoplus_{i \in I} M_i$ gilt $x = \sum_{i \in \supp(x)} \rho_i (x(i))$. Um $f \circ \rho_i = f_i$ f\"ur $i \in I$ zu erf\"ullen, kann man daher nur 
		\[f: \bigoplus_{i \in I} M_i \to N, x \mapsto \sum_{i \in I} f_i(x(i))\] 
		definieren. Man \"uberpr\"uft sofort, dass das so definierte $f$ ein Homomorphismus ist.
	\end{proof}
	
	\begin{prop-def}\label{1.2.3}
		Sei $R$ ein Ring, $M$ ein $R$-Modul und $(N_i)_{i \in I}$ eine Familie von Untermoduln on $M$. Dann sind die folgenden Bedingungen \"aquivalent
		\begin{enumerate}[(a)]
			\item Die Abbildung von der \"au\ss eren direkten Summe $\bigoplus_{i \in I} N_i$ nach $M$, die auf $N_i$ die Identit\"at ist, ist ein Isomorphismus
			\item $M = \sum_{i \in I} N_i$ und f\"ur alle $n \in \N$, paarweise verschiedenen $i_1, \dots, i_n \in I$ und alle $x_1 \in N_{i_1}, \dots, x_n \in N_{i_n}$ gilt 
				\[(x_1 + \cdots + x_n = 0) \Rightarrow (x_1 = \cdots = x_n = 0)\]  
		\end{enumerate}
	Gelten diese Bedingungen, so nennt man $M$ die \emph{(innere) direkte Summe}\index{Modul@\textbf{Modul}!Innere Direkte Summe} der $N_i$ ($i \in I$) und schreibt (angesichts der Isomorphismus aus $(a)$) wieder $M = \bigoplus_{i \in I} N_i$
	\end{prop-def}

	\begin{definition}\label{1.2.4}
		Sei $R$ ein Ring, $M$ ein $R$-Modul und $x \in M$. Der Kern des $R$-Modulhomomorphismus $R \to M, a \mapsto ax$ nennt man \emph{Annihilator}\index{Modul@\textbf{Modul}!Annihilator} von $x$, in Zeichen $\ann(x) = \set{a \in R| ax = 0}$.\newline
		Es hei\ss t $x$ ein \emph{Torsionselement}\index{Modul@\textbf{Modul}!Torsionselement} von $M$ wenn $\ann(x) \neq \set{0}$.
	\end{definition}

	\begin{satz}\label{1.2.5}
		Sei $R$ ein Ring, $M$ ein $R$-Modul und $B \subseteq M$. Dann sind \"aquivalent
		\begin{enumerate}[(a)]
			\item $B$ ist eine Basis von $M$
			\item $M = \bigoplus_{x \in B} Rx$ und $B$ enth\"alt kein Torsionselement
			\item F\"ur jeden $R$-Modul $N$ und jede Abbildung $g:B \to N$ gibt e genau einen Homomorphismus $f:M \to N$ mit $f\big|_b = g$.
		\end{enumerate}
	
		\begin{proof}\hfill
			\begin{enumerate}
				\item[$(a) \Rightarrow (b)$] klar
				\item[$(b) \Rightarrow (c)$] Gelte (b). Sei $N$ ein $R$-Modul und $g:B \to N$ eine Abbildung. Zu zeigen sind Existenz und Eindeutigkeit eines Homomorphismus $f: M \to N$ mit $f\big|_B = g$
					\begin{itemize}
						\item Eindeutigkeit: klar aus $M = \sum_{x \in B} Rx$
						\item Existenz: Fixiere zun\"achst $x \in B$. Dann ist $R \to Rx, a \mapsto ax$ ein Isomorphismus (mit Kern $\ann(x)$), dessen Umkehrfunktion ein Isomorphismus $Rx \to R$ ist, der $x$ auf $1$ abbildet. Schaltet man den Homomorphismus $R \to N, a \mapsto ag(x)$ dahinter, so erh\"alt man einen Homomorphismus $Rx \to N$, der $x$ auf $g(x)$ abbildet. Da $x \in B$ beliebig war, erh\"alt man mit \ref{1.2.2} einen Homomorphismus $f: M = \bigoplus_{x \in B} Rx \to N$, der jedes $x \in B$ auf $g(x)$ abbildet.  
					\end{itemize}
				\item[$(c) \Rightarrow (a)$] Gelte (c). Zu zeigen ist, dass $B$ linear unabh\"angig ist und $M$ erzeugt.
				\begin{enumerate}[1.]
					\item $B$ linear unabh\"angig: Seien $x_1, \dots, x_n \in B$ paarweise verschieden und $a_1, \dots, a_n \in R$ mit $a_1x_1 + \cdots + a_nx_n = 0$. Sei $i \in \set{1, \dots, n}$. Zu zeigen ist $a_i = 0$. Gem\"a\ss (c) gibt es einen Homomorphismus $f: M \to R$ mit $f(x_i) = 1$ und $f(x_j) = 0$ f\"ur $j \in \set{1, \dots, n} \setminus \set{i}$.
					Dann \[0 = f(0) = f\brac{\sum_{j=1}^n a_jx_j} = \sum_{j=1}a_jf(x_j) = a_if(x_i) = a_i\]
					
					\item $B$ erzeugt $M$: Nach (c) gibt es einen Homomorphismus $M \to M$, der auf $B$ die Identit\"at ist. Einerseits ist $\id_M$ ein solcher, andererseits auch $\rho \circ f$, wobei $f: M \to N:= \sum_{x \in B} Rx$ der nach (c) existierende Homomorphismus mit $f\big|_{B} = \id_B$ ist und $\iota: N \hookrightarrow M, x \mapsto x$ die Inklusion. Also $\id_M = \iota \circ f$, insbesondere $M = \im(\id_M) = \im(f) = N$
				\end{enumerate}
 			\end{enumerate}
		\end{proof}
	\end{satz}

	\begin{definition}\label{1.2.6}
		Ein Modul hei\ss t \emph{frei}\index{Modul@\textbf{Modul}!Freie Moduln}, wenn er eine Basis besitzt.
	\end{definition}

	\begin{bemerkung}\label{1.2.7}
		Sei $R$ ein Ring, $M$ ein $R$-Modul, $n \in \N_0$ und $x_1, \dots, x_n \in M$. Dann bilden $x_1, \dots, x_n$ genau dann eine Basis von $M$, wenn der Homomorphismus 
		\[R^n \to M, \begin{pmatrix}
			a_1\\\vdots\\a_n
		\end{pmatrix} \mapsto \sum_{i=1}^n a_ix_i\] ein Isomorphismus ist.
	\end{bemerkung}
	
	\begin{bemerkung}\label{1.2.8}
		Ist $M$ ein $\set{0}$-Modul, so ist $M = \set{0}$, denn ist $x \in M$, so ist $x=1 \cdot x = 0 \cdot x = 0$
	\end{bemerkung}

	\begin{lemma}\label{1.2.9}
		Ein endlich erzeugter Modul hat niemals eine unendliche Basis.
		
		\begin{proof}
			Sei $M$ ein endlich erzeugter $R$-Modul, etwa $M = \sum_{i=1}^n Rx_i$ mit $x_1, \dots, x_n \in M$. \newline
			Annahme: $B$ ist eine unendliche Basis von $M$. Dann gibt es f\"ur jedes $i \in \set{1, \dots, n}$ ein endliches $B_i \subseteq B$ mit $x_i \in \sum_{y \in B_i}Ry$. Dann ist $B' := B_1 \cup \cdots \cup B_n \subseteq B$ endlich mit $M = \sum_{y \in B'}Ry$. Da $B$ unendlich h ist, gibt es ein $z \in B\setminus B'$
			
			Nun gilt $z \in \sum_{y \in B'}Ry$, was im Widerspruch zur linearen Unabh\"angigkeit von $B$ steh, au\ss er wenn $1 = 0$ in $R$, d.h. $R = \set{0}$. Im letzten Fall ist aber nach \ref{1.2.8} nichts zu zeigen.
		\end{proof}
	\end{lemma}

	\begin{bemerkung}\label{1.2.10}
		\begin{enumerate}[(a)]
			\item 
			Jeder Modul \"uber dem Nullring hat genau zwei Basen, n\"amlich $\emptyset$ und $\set{0}$. In der Tat: Nach \ref{1.2.8} handelt es sich um den Nullmodul und in einem $\set{0}$-Modul ist $0$ linear unabh\"angig.
			\item In den \"Ubungen geben wir einen Ring $R \neq 0$, der als $R$-Modul zu $R^2$ isomorph ist. Durch Induktion schlie\ss t man, dass $R \cong R^n$ f\"ur alle $n \in \N$. Damit besitzt $R$ als $R$-Modul f\"ur jedes $n \in \N$ eine $n$-elementige Basis, aber nach \ref{1.2.9} keine unendliche Basis.
		\end{enumerate}
	\end{bemerkung}

	\begin{satz}\label{1.2.11}
		Sei $R$ ein kommutativer Ring mit $1 \neq 0$. Dann sind je zwei Basen eines $R$-Moduls entweder beide unendlich oder beide endlich mit der selben Anzahl von Elementen
		
		\begin{proof}
			Sei $M$ ein $R$-Modul mit Basen $B$ und $C$. Im Fall von $|B| =  \infty = |C|$ sind wir fertig, sonst ist $M$ endlich erzeugt und daher $m = |B|, n = |C| \in \N_0$ nach Lemma \ref{1.2.9}. Nach \ref{1.2.7} gilt $R^n \cong M \cong R^m$, somit reicht es zu zeigen: Sei $R$ ein kommutativer Ring und $m, n \in \N_0, m > n$ mit $R^m \cong R^n$ als $R$-Modul, dann gilt $1 = 0$ in $R$. 
			
			\noindent Um dies zu zeigen, w\"ahle zueinander inverse $R$-Modulisomorphismen $f:R^n \to R^m, g: R^m \to R^n$. Bezeichne mit $\underline{x} = (x_1, \dots, x_n)$ und $\underline{y} = (y_1, \dotsm y_m)$ die Standardbasen des $R^n$ und $R^m$. W\"ahle $A=(a_{ij})_{1\leq i \leq m, 1 \leq j \leq n}\in R^{m \times n}$ mit $f(x_j) = \sum_{i=1}^m a_{ij}y_i$ f\"ur $j \in \set{1, \dots, n}$ und $B = (b_{ji})_{1 \leq j \leq n, 1 \leq i \leq m} \in R^{n \times m}$ mit $f(y_i) = \sum_{j=1}^n b_{ji}x_j$ f\"ur $i \in \set{1, \dots, m}$. Dann gilt f\"ur $k \in \set{1, \dots, m}$
				\begin{align*}
					y_k &= (f \circ g)(y_k) = f(g(y_k))\\
					&= f\brac{\sum_{j=1}^n b_{jk}x_j}\\
					&= \sum_{j=1}^n b_{jk}f(x_j)\\
					&= \sum_{j=1}^n b_{jk}\sum_{i=1}^ma_{ij}y_i\\
					&= \sum_{i=1}^m\brac{\sum_{j=1}^n b_{jk}a_{ij} }y_i \overset{R\ \mathrm{komm.}}{=} \sum_{i=1}^m\brac{\sum_{j=1}^n a_{ij}b_{jk} }y_i
				\end{align*} 
			und daher
			\[\sum_{j=1}^n a_{ij}b_{jk} = \begin{cases}
				1 & \mathrm{falls }\ k = i\\
				0 & \mathrm{sonst}
			\end{cases}\] 
			f\"ur alle $i, k \in \set{1, \dots, m}$, d.h. $AB = I_m$.\newline
			Wegen $n < m$ k\"onnen wir $\displaystyle A' := (A\ \underbrace{0}_{(m-n)-\mathrm{Spalten}}) \in R^{m\times m}$ und $B' := \begin{pmatrix}
				B\\0
			\end{pmatrix} \in R^{m \times m}$ (mit $m-n$ $0$-Zeilen) setzen, so dass $A'B' = AB = I_m$. \newline
			Mit dem Determinantenproduktsatz folgt 
			\[0 = 0\cdot 0 = (\det A')(\det B') = \det(A'B') = 1\] 
		\end{proof}
	\end{satz}

	\begin{bemerkung}\label{1.2.12}
		Statt den Determinantenproduktsatz \"uber kommutativen Ringen zu verwenden, kann man den Beweis des letzten Satzes auch mit der Theorie kommutativer Ringe auf die Dimensionstheorie von Vektorr\"aumen zur\"uckspielen. 
		
		\noindent Sei $R$ ein kommutativer Ring mit $1 \neq 0$, $m, n\in \N_0$ mit $R^m \cong R^n$. Wir zeigen $m = n$.
		
		\begin{proof}
			W\"ahle ein maximales Ideal $\mathfrak{m}$ von $R$. W\"ahle einen $R$-Modulisomorphismus $f:R^m \to R^n$. Betrachte die $R$-Untermoduln
			\[\mathfrak{m}R^m := \set{\sum_{i=1}^n a_ix_i|n \in \N_0, a_i \in \mathfrak{m}, x_i \in R^m} = \mathfrak{m}^m\]
			von $R^m$ und 
			\[f(\mathfrak{m}R^n) = \set{\sum_{i=1}^n a_iy_i|n \in \N_0, a_i \in \mathfrak{m}, y_i \in R^n} = \mathfrak{m}^n\]
			von $R^n$
			
			\noindent Mit dem Isomorphiesatz erhalten wir einen Modulisomorphismus $R^m/\mathfrak{m}^m \to R^n/\mathfrak{m}^n$ und offensichtlich gilt $R^m/\mathfrak{m}^m  \cong (R/\mathfrak{m})^m$ (betrachte z.B. $R^m \to (R/\mathfrak{m})^m$).\newline
			Da nun $(R/\mathfrak{m})^m$ und $(R/\mathfrak{m})^n$ als $R$-Moduln isomorph sind, sind sie auch als $(R/\mathfrak{m})$-Moduln isomorph. F\"ur den K\"orper $K:= R/\mathfrak{m}$ gilt also 
			\[m = \dim_K K^m = \dim_K K^n = n\]
		\end{proof}
	\end{bemerkung}

	\begin{definition}\label{1.2.13}
		Sei $R$ ein kommutativer Ring mit $1 \neq 0$ und $M$ ein freier $R$-Modul mit Basis $B$. Dann hei\ss t $\rk M := |B| \in \N_0 \cup \set{\infty}$ der \emph{Rang}\index{Modul@\textbf{Modul}!Rang} von $M$ [h\"angt nach \ref{1.2.11} nicht von der Wahl der Basis $B$ ab ]
	\end{definition}
	\newpage
	
	\section{Halbeinfache Moduln}
	\begin{notation}\label{1.3.1}
		$0 := \set{0}$ Nullmodul
	\end{notation}
	
	\begin{definition}\label{1.3.2}
		Ein Modul $M$ hei\ss t \emph{einfach}\index{Modul@\textbf{Modul}!Einfache Moduln} (oder irreduzibel), falls $M \neq 0$ und $0$ und $M$ die einzigen Untermoduln von $M$ sind.
	\end{definition}
	
	\begin{bemerkung}\label{1.3.3}
		Sei $N$ ein Untermoduln von $M$.
		\begin{enumerate}[(a)]
			\item Bezeichne $\varphi: M \to M/N$ den kanonischen Epimorphismus. Dann vermitteln die Zuordnungen
			\begin{align*}
				L &\mapsto L/N = \varphi(L) \\
				\varphi^{-1}(P) &\mapsfrom P
			\end{align*}
			Eine Bijektion zwischen der Menge der Untermoduln $L$ von $M$ mit $N \subseteq L$ und der Menge der Untermoduln von $M/N$
					
			\item Es folgt, dass $M/N$ einfach ist genau dann, wenn $N$ ein maximaler echter Untermodul ist.
		\end{enumerate}
	\end{bemerkung}

	\begin{beispiele}\label{1.3.4}
		\begin{enumerate}[(a)]
			\item Sei $R$ ein kommutativer Ring und $I$ ein $R$-Untermodul von $R$, d.h. ein Ideal von $R$ [$\to$\ref{1.1.5}]. Dann ist $R/I$ ein einfacher $R$-Modul $\Leftrightarrow$ $I$ ist ein maximales Ideal von $R$ $\Leftrightarrow$ $R/I$ ist ein K\"orper.
			
			\item Sei $b$ ein Hauptidealring und $p \in R\setminus \set{0}$. Dann ist $R/pR$ ein einfacher Modul genau dann, wenn $p$ irreduzibel in $R$ ist.
			
			\begin{proof}
				\begin{itemize}
					\item[$\Rightarrow$] Ist $(p)$ ein maximales Ideal von $R$, so auch ein Primideal, d.h. $p$ ist prim in $R$ und daher auch irreduzibel in $R$ (wegen $p \neq 0$) 
					
					\item[$\Leftarrow$] Ist $p$ irreduzibel in $R$, so ist $R/(p)$ ein K\"orper und daher ist $(p)$ ein maximales Ideal in $R$.
				\end{itemize}
			\end{proof}
		\end{enumerate}
	\end{beispiele}

	\begin{lemma}\label{1.3.5}
		Sei $R$ ein Ring und $M$ ein $R$-Modul. Es sind \"aquivalent:
		\begin{enumerate}[(i)]
			\item $M$ ist einfach
			\item $M \neq 0$ und jedes Element von $M \setminus \set{0}$ erzeugt $M$
			\item Es gibt einen maximalen echten $R$-Untermodul $N$ von $R$ mit $R/N \cong M$
		\end{enumerate}
	
		\begin{proof}\hfill
			\begin{enumerate}
				\item[$(a) \Rightarrow (c)$] Gelte (a) W\"ahle $x \in M \setminus \set{0}$. Dann ist der Homomorphismus $\varphi:R \to M, a \mapsto ax$ surjektiv und daher $R/N \cong M$ mit $N:=\ker \varphi$. Mit $M$ ist auch $R/N$ einfach, weswegen nach \ref{1.3.3}(b) $N$ ein maximaler echter Untermodul von $R$ ist.
				\item[$(c) \Rightarrow (b)$] trivial
				\item[$(b) \Rightarrow (a)$] trivial
			\end{enumerate}
		\end{proof}
	\end{lemma}
\backmatter
\printindex
\end{document}