\documentclass[
twoside=semi,
fontsize=12,
DIV=12, 
cleardoublepage=current,
leqno,
headings=optiontoheadandtoc, 
toc=idx
]{scrbook}

\usepackage{imakeidx}
\makeindex

\usepackage[german]{babel}
\usepackage[utf8]{inputenc}
\usepackage[T1]{fontenc}

\usepackage{datetime}


\usepackage[headsepline]{scrlayer-scrpage}
\setkomafont{pageheadfoot}{\normalfont}
\setkomafont{pagefoot}{\slshape}

\defpagestyle{scrheadings}{%
	{}{}{}
}{
	{\pagemark\hfill}{\hfill \pagemark}{}
}

\defpagestyle{plain}{%
	{}{}{}(0pt,0pt)
}{
	{\pagemark \hfill}{\hfill \pagemark}{}
}

\usepackage{chngcntr}\counterwithout{equation}{section}

\usepackage{amsmath}
\usepackage{amssymb}
\usepackage{amsthm}
\usepackage{stmaryrd}
\usepackage{enumerate}
\usepackage[pass]{geometry}
\usepackage{csquotes}


\usepackage{hyperref}
\MakeOuterQuote{"}

\newcommand{\N}{\mathbb{N}}
\newcommand{\Z}{\mathbb{Z}}
\newcommand{\Q}{\mathbb{Q}}
\newcommand{\R}{\mathbb{R}}
\newcommand{\F}{\mathbb{F}}

\newcommand{\comment}[1]{}

\newcommand{\brac}[1]{\left( #1 \right)}
\newcommand{\bracB}[1]{\left[ #1 \right]}
\newcommand{\bracC}[1]{\left< #1 \right>}
\newcommand{\abs}[1]{\left| #1 \right|}
\newcommand{\set}[1]{\left\{ #1 \right\}}

\newcommand*{\ORIGchapterheadstartvskip}{}%
\let\ORIGchapterheadstartvskip=\chapterheadstartvskip
\renewcommand*{\chapterheadstartvskip}{%
	\ORIGchapterheadstartvskip
	\noindent\rule[\baselineskip]{\linewidth}{4pt}\par
}
\newcommand*{\ORIGchapterheadendvskip}{}%
\let\ORIGchapterheadendvskip=\chapterheadendvskip
\renewcommand*{\chapterheadendvskip}{%
	\ORIGchapterheadendvskip
	\noindent\rule[\baselineskip]{\linewidth}{4pt}\par
}

\DeclareMathOperator{\im}{im}

\theoremstyle{definition}
\newtheorem{definition}{Definition}[section]
\newtheorem{bemerkung}[definition]{Bemerkung}
\newtheorem{beispiele}[definition]{Beispiele}
\newtheorem{warnung}[definition]{Warnung}
\newtheorem{def-prop-satz-not}[definition]{Definitionen, Propositionen, S\"atze und Notationen}

\begin{document}
	\pagenumbering{Alph}
	\pagestyle{empty}
	\tableofcontents
	\mainmatter
	\pagestyle{scrheadings}
	\chapter[tocentry={Moduln}]{Moduln}
	
	
	\section{Definitionen und grundlegende Tatsachen}
	\setcounter{chapter}{1}
	\setcounter{section}{1}
	\begin{definition}
		\label{1.1.1}
		
		Ein \emph{Modul}\index{Modul@\textbf{Modul}} ist ein Tupel $(R, +_R, \cdot_R, M, +, \cdot)$, wobei $(R, +_R, \cdot_R)$ ein Ring (mit $1$, nicht notwendigerweise kommutativ), $(M, +)$ eine abelsche Gruppe und \\\noindent$\cdot:R\times M \to M$ eine (meist gar nicht oder infix geschriebene) Abbildung mit folgenden Eigenschaften
		
		\begin{itemize}
			\item[$(\overset{\rightarrow}{D})$] $\forall a \in R: \forall x, y \in M: a(x + y) = ax + ay$ \hfill "distributiv"
			
			\item[$(D')$] $\forall a, b \in R: \forall x \in M: (a+b)x = ax + bx$ \hfill "distributiv"
			
			\item[$(N)$] $\forall x \in M: 1_R \cdot x = x$ \hfill "normiert"
			
			\item[$(V)$] $\forall a, b \in R: \forall x \in M: (ab)x = a(bx)$ \hfill "vertr\"aglich"
		\end{itemize}
	\end{definition}
	
	
	\begin{bemerkung}\label{1.1.2}
			
			\begin{enumerate}[(a)]
			\item Schlampiger Sprachgebrauch: 
			\begin{itemize}
				\item "Sei $M$ ein $R$-Modul" statt "Sei $(R, +_R, \cdot_R, M, +, \cdot)$ ein Modul"
				
				\item "Sei $M$ ein Modul" statt "Es gebe einen Ring $R$ so, dass $M$ ein $R$-Modul ist"
			\end{itemize}
			
			\item Statt "$R$-Modul" sagt man auch "Modul \"uber $R$"	
			
			\item Vektorr\"aume sind Moduln \"uber K\"orper. Viele Sprechweisen (wie "Skalar", "Linearkombination", nicht jedoch "Vektor") \"ubertragen wir stillschweigend von Vektorr\"aumen auf Moduln, ebenso 
			Konventionen (wie "Punkt vor Strich").
			
			\item Abelsche Gruppen "sind" $\Z$-Moduln. Sei $G$ eine abelsche Gruppe. Dann gibt es genau eine Skalarmultiplikation $\cdot:\Z\times G \to G$ verm\"oge derer $G$ zu einem $\Z$-Modul wird, n\"amlich die nat\"urliche, die durch 
			\[n \cdot a := \begin{cases}
				\underbrace{a + a + \cdots + a}_{n\textrm{-mal}} & \textrm{falls } n > 0\\
				0 & \textrm{falls } n = 0\\
				\underbrace{-a - a - \cdots - a}_{(-n)\textrm{-mal}} & \textrm{falls } n < 0
			\end{cases}\]
			gegeben ist.
			
			\item $(\overset{\rightarrow}{D})$ besagt, dass f\"ur alle $a \in R$ die Abbildung $M \to M, x \mapsto ax$ ein Gruppenhomomorphismus ist. Insbesondere gilt $a \cdot 0 = 0$ und $a \cdot (-x) = -ax$ f\"ur alle $a \in R, x \in M$.
			
			$(D')$ besagt, dass f\"ur alle $x \in M$ die Abbildung $R \to M, a \mapsto ax$ ein Gruppenhomomorphismus ist. Insbesondere gilt $0 \cdot x = 0$ und $(-a) \cdot x = -ax$ f\"ur alle $a \in R, x \in M$.
		\end{enumerate}
		
	\end{bemerkung}
	
	\begin{beispiele}\label{1.1.3}
			\begin{enumerate}[(a)]
			\item Nullmoduln $\set{0}$
			
			\item Sei $A$ ein Unterring des Ringes $B$. Dann ist $B$ ein $A$-Modul verm\"oge der Skalarmultiplikation $\cdot: A \times B \to B, (a, x) \mapsto ax$
			
			Insbesondere ist jeder Ring ein Modul \"uber sich selbst.
			
			\item Sei $R$ ein kommutativer Ring und $n \in \N_0$. Dann wird die abelsche Gruppe $R^n$ zu einem $R^{n\times n}$-Modul verm\"oge der Skalarmultiplikation
			\[\cdot: R^{n\times n} \times R^n \to R^n, (A, x) \mapsto Ax\]
			Dies folgt aus den Rechenregeln f\"ur Matrixmultiplikation.
		\end{enumerate}
	\end{beispiele}
	
	\begin{def-prop-satz-not}\label{1.1.4}
		Sei $R$ ein Ring. Die folgenden f\"ur die Theorie der $R$-Moduln grundlegenden Begriffe und Resultate sind eine direkte Verallgemeinerung der entsprechenden Tatsachen f\"ur Vektorr\"aume (also f\"ur den Fall, dass $R$ ein K\"orper) und f\"ur abelsche Gruppen (also $R = \Z$) aus der Linearen Algebra:
		
		\begin{enumerate}[(a)]
			\item Genauso wie bei Vektorr\"aumen f\"uhrt man \emph{direkte Produkte}\index{Modul@\textbf{Modul}!Direktes Produkt} von $R$-Moduln ein.
			
			\item Sind $M$ und $N$ $R$-Moduln, so hei\ss t $N$ ein \emph{Untermodul}\index{Modul@\textbf{Modul}!Untermodul} von $M$, wenn die $N$ zugrunde liegende abelsche Gruppe eine Untergruppe der $M$ zugrunde liegenden abelschen Gruppe ist und 
			\[\forall a \in R: \forall x \in M: a \cdot_N x = a \cdot_M x\]
			
			Ein Untermodul eines Moduls ist offenbar durch seine Tr\"agermenge (d.h. seine zugrunde liegende Menge) eindeutig bestimmt.
			
			Ist $M$ ein $R$-Modul und $N \subseteq M$, so ist $N$ offenbar genau dann (Tr\"agermenge) ein(e) Untermodul(s) von $M$, wenn $0\in N, \forall x, y\in N: x + y \in N, \forall a \in R: \forall x \in N: ax \in N$
			
			\item Sei $M$ ein Modul und $(N_i)_{i \in I}$ eine Familie von Untermoduln von $M$. Dann ist 
			$\bigcap_{i \in I} N_i := \bigcap \set{N_i | i \in I}$ (mit $\bigcap_{i \in I} N_i = M$, falls $I = \emptyset$)
			wieder ein Untermodul von $M$ und zwar der gr\"o\ss te Untermodul von $M$, der in allen $N_i$ enthalten ist.
			
			Weiter ist auch $\sum_{i \in I} N_i := \set{\sum_{i \in I} x_i | (x_i)_{i \in I} \in \prod_{i\in I} N_i, \set{i \in I| x_i \neq 0} \textrm{endlich}}$ Untermodul von $M$ und zwar der kleinste Untermodul von $M$, der alle $N_i$ enth\"alt.
			
			\item Sei $M$ ein $R$-Modul. Ist $x \in M$, so ist $Rx := \set{ax| a \in R}$ ein Untermodul von $M$ und zwar der kleinste Untermodul, der $x$ enth\"alt.
			
			Ist $(x_i)_{i \in I}$ eine Familie von Elementen von $M$, so ist $\sum_{i \in I} Rx_i$ der kleinste Untermodul von $M$, der alle $x_i$ enth\"alt.
			
			Man nennt ihn den von den $x_i$ ($i \in I$) (oder $\set{x_i|i \in I}$) erzeugten Untermodul von $M$ (oder lineare H\"ulle der Span von $\set{x_i| i \in I}$). 
			
			Man nennt $M$ \emph{zyklisch}\index{Modul@\textbf{Modul}!Zyklische Moduln}, wenn $M$ von einem Element erzeugt wird, d.h. es ein $x \in M$ gibt mit $M = Rx$. Man nennt $M$ endlich erzeugt (e.e.), wenn $M$ von endlich vielen Elementen
			erzeugt wird, d.h. es ein $n \in \N_0$ und $x_1, \dots, x_n \in M$ gibt mit 
			\[M = Rx_1 + \cdots + Rx_n := \sum_{i=1}^{n} Rx_i := \sum_{i \in \set{1, \dots, n}} Rx_i\] 
			
			\item Sei $M$ ein $R$-Modul. Eine Familie $(x_i)_{i\in I}$ in $M$ hei\ss t \emph{linear unabh\"angig}\index{Modul@\textbf{Modul}!Linear unabh\"angig (l.u.)} (l.u.), wenn f\"ur alle $n\in \N_0$, alle paarweise verschiedenen $i_1, \dots, i_n \in I$ und alle $a_1, \dots, a_n \in I$ gilt 
			\[\sum_{j=1}^{n} a_jx_{i_j} = 0 \Rightarrow a_1 = \cdots = a_n = 0\]
			
			Weiter nennt man $x_1, \dots, x_n$ linear unabh\"angig, wenn $(x_1, \dots, x_n) = (x_i)_{i \in \set{1, \dots, n}}$ linear unabhängig ist, d.h. f\"ur alle $a_1, \dots, a_n \in R$ gilt 
			\begin{align}
				a_1x_1 + \cdots a_nx_n = 0 \Rightarrow a_1 = \cdots = a_n = 0 \label{lin_ind}
			\end{align}
			
			Schlie\ss lich hei\ss t eine Menge $F\subseteq M$ linear unabh\"angig, wenn $(x)_{x \in F}$ linear unabh\"angig ist, d.h. f\"ur alle $n\in \N_0$, alle paarweise verschiedenen $x_1, \dots, x_n \in F$ und alle $a_1, \dots, a_n \in R$ wieder \ref{lin_ind} gilt.
			
			\item Sei $M$ ein Modul. Eine Familie $(x_i)_{i \in I}$ in $M$ hei\ss t eine \emph{Basis}\index{Modul@\textbf{Modul}!Basis} von $M$, wenn sie $M$ erzeugt und linear unabh\"angig ist. Weiter
			sagt man $x_1, \dots, x_n \in M$ bilden eine Basis von $M$, wenn $(x_1, \dots, x_n) = (x_i)_{i \in \set{1, \dots, n}}$ eine Basis von $M$ ist. Schlie\ss lich hei\ss t $B \subseteq M$ eine Basis, wenn $B$ den Modul $M$ erzeugt und linear unabh\"angig ist.
			
			\item Seien $M$ und $N$ $R$-Moduln. Dann hei\ss t $f$ ein \emph{($R$-)(Modul-)Homomorphismus}\index{Modul@\textbf{Modul}!Homomorphismus} oder eine $(R-)$ lineare Abbildung von $M$ nach $N$, wenn $f:M\to N$ ein Gruppenhomomorphismus der $M$ und $N$ zugrundeliegenden abelschen Gruppen ist und 
			\[\forall a \in R: \forall x \in M: f(ax) = af(x) \]
			
			Ein Modulhomomorphismus $f:M\to N$ hei\ss t Einbettung/Monomorphismus (Epimorphismus, Isomorphismus), wenn $f$ injektiv (surjektiv, bijektiv) ist. 
			
			Ein Modulhomomorphismus $f:M \to M$ hei\ss t \emph{(Modul-)Endomorphismus}\index{Modul@\textbf{Modul}!Homomorphismus!Endomorphismus} von $M$. Ein Endomorphismus, der ein Isomorphismus ist, hei\ss t \emph{Automorphismus}\index{Modul@\textbf{Modul}!Automorphismus}. Es heißen $M$ und $N$ \emph{isomorph}, in Zeichen $M \cong N$, wenn es einen Isomorphismus $M \to N$ gibt.
			
			Hintereinanderschaltungen von Modulhomomorphismen sind wieder Modulhomomorphismen. Umkehrabbildungen von Modulisomorphismen sind wieder Modulisomorphismen.
			
			\item Sei $M$ ein $R$-Modul. Eine \emph{Kongruenzrelation}\index{Modul@\textbf{Modul}!Kongruenzrelation} auf $M$ ist eine \"Aquivalenzrelation $\equiv$ der $M$ zugrundeliegenden Menge, f\"ur die gilt
			\[\forall x, y, x', y' \in M: (x \equiv x' \land y \equiv y') \Rightarrow x + y \equiv x' + y'\]
			und
			\[\forall x, x' \in M: \forall a \in R: x \equiv x' \Rightarrow ax \equiv ax'\]
			
			Diese Definition wurde gerade so gemacht, dass 
			\[+:( M/\equiv) \times (M/\equiv) \to (M/\equiv), (\overline{x}, \overline{y}) \mapsto \overline{x + y}\]
			und
			\[\cdot: R \times (M/\equiv) \to (M/\equiv), (a, \overline{x}) \mapsto \overline{ax}\]
			wohldefiniert sind.
			
			Ist $M$ ein $R$-Modul und $\equiv$ eine Kongruenzrelation auf $M$, so wird die Quotientenmenge $M/\equiv$ verm\"oge der Addition $+$ und der Skalarmultiplikation $\cdot$ ein $R$-Modul, wie man
			durch direktes Nachrechnen sieht. Die Zuordnungen
			\begin{align*}
				\equiv  &\overset{f}{\mapsto} \overline{0}\\
				\equiv_N &\overset{g}{\mapsfrom} N
			\end{align*}
			vermitteln eine Bijektion zwischen der Menge der Kongruenzrelationen auf $M$ und der Menge der Untermoduln von $M$, wobei $\equiv_N$ gegeben ist durch
			\[a \equiv_N b :\Leftrightarrow a - b \in N\]
			f\"ur $a, b \in M$.
			
			Ist $N$ ein Untermodul von $M$, so nennt man $M/N := M/\equiv_N$ auch den \emph{Quotientenmodul}\index{Modul@\textbf{Modul}!Quotientenmodul} von $M$ nach $N$.
			
			\item Sind $M$ und $N$ $R$-Moduln und $f:M \to N$ ein Modulhomomorphismus, so ist der \emph{Kern}\index{Modul@\textbf{Modul}!Homomorphismus!Kern} $\ker f := \set{x \in M| f(x) = 0}$ von $f$ ein Untermodul von $M$ und das \emph{Bild}\index{Modul@\textbf{Modul}!Homomorphismus!Bild} $\im f := \set{f(x) | x \in M}$ von $f$ ist ein Untermodul von $N$.
			
			\item Homomorphiesatz: Seien $M$ und $N$ $R$-Moduln und $L$ ein Untermodul von $M$ und $f:M \to N$ ein Modulhomomorphismus mit $L \subseteq \ker f$. Dann gibt es (genau) einen Modulhomomorphismus $\overline{f}: (M/L) \to N$ mit $\overline{f}(\overline{x}) = f(x)$ f\"ur alle $x \in M$.
			
			Ferner gilt, dass
			\begin{itemize}
				\item $\overline{f}$ ist injektiv $\Leftrightarrow L = \ker f$ und
				\item $\overline{f}$ ist surjektiv $\Leftrightarrow f$ ist surjektiv
			\end{itemize}
			
			\item Isomorphiesatz: Seien $M$ und $N$ $R$-Moduln und $f:M \to N$ ein Modulhomomorphismus. Dann ist $\overline{f}:(M/\ker f) \to \im f$ definiert durch $\overline{f}(\overline{x}) = f(x)$ f\"ur alle $x\in M$ ein $R$-Modulisomorphismus. Insbesondere ist $M/\ker f \cong \im f$
		\end{enumerate}
	\end{def-prop-satz-not}
	
	\begin{bemerkung}\label{1.1.5}
		Sei $R$ ein kommutativer Ring. Dann sind die Untermoduln des $R$-Modul $R$ [$\to$\ref{1.1.3}(b)] (oder kurz gesagt die $R$-Untermoduln von $R$) genau die Ideale des Ringes $R$.
		Insbesondere sind zum Beispiel das von einem $a \in R$ erzeugte Ideal und der davon erzeugte Untermodul als Menge dasselbe $(a)_R = Ra \overset{R\ komm.}{=} \set{ab | b \in R} = aR$.
		Trotzdem macht es vom Sinn her einen Unterschied. ob man $(a)$ oder $Ra$ schreibt. Zum Beispiel meint man mit $R/(a)$ den Ring und mit $R/aR$ den $R$-Modul (deren zugrundeliegenden abelschen Gruppen dieselben sind)
	\end{bemerkung}

	\begin{warnung}\label{1.1.6}
		F\"ur den mit Vektorr\"aumen, aber nicht mit Moduln vertrauten H\"orern ist Vorsicht geboten:
		\begin{enumerate}[(a)]
			\item In einem $R$-Modul $M$ kann $ax = 0$ f\"ur ein $a \in R$ und ein $x \in M$ gelten, ohne dass $a = 0$ oder $x = 0$ gilt (zum Beispiel $2 \cdot \overline{1} = \overline{2} = 0$ im $\Z$-Modul $\Z/2\Z$)
			
			\item Nicht jeder Modul hat eine Basis:
			zum Beispiel ist jedes Element des $\Z$-Moduls $\Z/2\Z$ linear abh\"angig, denn $1 \cdot \overline{0} = \overline{0} = 0$ und $2 \cdot \overline{1} = \overline{2} = 0$ in $\Z/2\Z$, womit die einzige linear unabh\"angige Teilmenge von $\Z/2\Z$ die leere Menge is, welche aber $\Z/2\Z$ nicht erzeugt.
		\end{enumerate}
	\end{warnung}	

	\begin{beispiele}\label{1.1.7}
		\begin{enumerate}[(a)]
			\item F\"ur jeden Ring $R$ ist $R^n$ ein $R$-Modul mit der \emph{Standardbasis}\index{Modul@\textbf{Modul}!Standardbasis} $\underline{e} = (e_1, \dots, e_n)$, wobei $e_i := \begin{pmatrix}
				0\\
				\vdots\\
				0\\
				1\\
				0\\
				\vdots\\
				0
			\end{pmatrix}$ mit einer $1$ an der $i$-ten Stelle.
		
			\item $\R^2$ ist ein zyklischer $\R^{2 \times 2}$ Modul [$\to$\ref{1.1.3}(c)], welcher von jedem $x \in \R^{2\times 2} \setminus \set{0}$ erzeugt ist. Da aber jedes $x \in \R^{2 \times 2}$ linear abh\"angig ist, hat dieser Modul keine Basis.  
		\end{enumerate}
	\end{beispiele}
\backmatter
\printindex
\end{document}