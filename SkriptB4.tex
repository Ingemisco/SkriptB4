\documentclass[
twoside=semi,
fontsize=12,
DIV=12, 
cleardoublepage=current,
leqno,
headings=optiontoheadandtoc, 
toc=idx
]{scrbook}

\usepackage{imakeidx}
\makeindex

\usepackage[german]{babel}
\usepackage[utf8]{inputenc}
\usepackage[T1]{fontenc}

\usepackage{datetime}


\usepackage[headsepline]{scrlayer-scrpage}
\setkomafont{pageheadfoot}{\normalfont}
\setkomafont{pagefoot}{\slshape}

\defpagestyle{scrheadings}{%
	{}{}{}
}{
	{\pagemark\hfill}{\hfill \pagemark}{}
}

\defpagestyle{plain}{%
	{}{}{}(0pt,0pt)
}{
	{\pagemark \hfill}{\hfill \pagemark}{}
}

\usepackage{chngcntr}\counterwithout{equation}{section}

\usepackage{amsmath}
\usepackage{amssymb}
\usepackage{amsthm}
\usepackage{stmaryrd}
\usepackage{enumerate}
\usepackage[pass]{geometry}
\usepackage{csquotes}


\usepackage{hyperref}
\MakeOuterQuote{"}

\newcommand{\N}{\mathbb{N}}
\newcommand{\Z}{\mathbb{Z}}
\newcommand{\Q}{\mathbb{Q}}
\newcommand{\R}{\mathbb{R}}
\newcommand{\F}{\mathbb{F}}

\newcommand{\comment}[1]{}

\newcommand{\brac}[1]{\left( #1 \right)}
\newcommand{\bracB}[1]{\left[ #1 \right]}
\newcommand{\bracC}[1]{\left< #1 \right>}
\newcommand{\abs}[1]{\left| #1 \right|}
\newcommand{\set}[1]{\left\{ #1 \right\}}

\newcommand*{\ORIGchapterheadstartvskip}{}%
\let\ORIGchapterheadstartvskip=\chapterheadstartvskip
\renewcommand*{\chapterheadstartvskip}{%
	\ORIGchapterheadstartvskip
	\noindent\rule[\baselineskip]{\linewidth}{4pt}\par
}
\newcommand*{\ORIGchapterheadendvskip}{}%
\let\ORIGchapterheadendvskip=\chapterheadendvskip
\renewcommand*{\chapterheadendvskip}{%
	\ORIGchapterheadendvskip
	\noindent\rule[\baselineskip]{\linewidth}{4pt}\par
}

\DeclareMathOperator{\im}{im}
\DeclareMathOperator{\supp}{supp}
\DeclareMathOperator{\ann}{ann}
\DeclareMathOperator{\id}{id}
\DeclareMathOperator{\rk}{rk}
\DeclareMathOperator{\End}{End}

\theoremstyle{definition}
\newtheorem{definition}{Definition}[section]
\newtheorem{bemerkung}[definition]{Bemerkung}
\newtheorem{beispiele}[definition]{Beispiele}
\newtheorem{warnung}[definition]{Warnung}
\newtheorem{satz}[definition]{Satz}
\newtheorem{lemma}[definition]{Lemma}
\newtheorem{proposition}[definition]{Proposition}
\newtheorem{notation}[definition]{Notation}
\newtheorem{korollar}[definition]{Korollar}
\newtheorem{prop-def}[definition]{Proposition und Definition}
\newtheorem{def-prop-satz-not}[definition]{Definitionen, Propositionen, S\"atze und Notationen}
\newtheorem{defueb}[definition]{Definition und \"Ubung}

\begin{document}
	\pagenumbering{Alph}
	\pagestyle{empty}
	\tableofcontents
	\mainmatter
	\chapter[tocentry={Moduln}]{Moduln}
	\pagestyle{scrheadings}
	
	\section{Definitionen und grundlegende Tatsachen}
	\setcounter{chapter}{1}
	\setcounter{section}{1}
	\begin{definition}
		\label{1.1.1}
		
		Ein \emph{Modul}\index{Modul@\textbf{Modul}} ist ein Tupel $(R, +_R, \cdot_R, M, +, \cdot)$, wobei $(R, +_R, \cdot_R)$ ein Ring (mit $1$, nicht notwendigerweise kommutativ), $(M, +)$ eine abelsche Gruppe und \\\noindent$\cdot:R\times M \to M$ eine (meist gar nicht oder infix geschriebene) Abbildung mit folgenden Eigenschaften
		
		\begin{itemize}
			\item[$(\overset{\rightarrow}{D})$] $\forall a \in R: \forall x, y \in M: a(x + y) = ax + ay$ \hfill "distributiv"
			
			\item[$(D')$] $\forall a, b \in R: \forall x \in M: (a+b)x = ax + bx$ \hfill "distributiv"
			
			\item[$(N)$] $\forall x \in M: 1_R \cdot x = x$ \hfill "normiert"
			
			\item[$(V)$] $\forall a, b \in R: \forall x \in M: (ab)x = a(bx)$ \hfill "vertr\"aglich"
		\end{itemize}
	\end{definition}
	
	
	\begin{bemerkung}\label{1.1.2}
			
			\begin{enumerate}[(a)]
			\item Schlampiger Sprachgebrauch: 
			\begin{itemize}
				\item "Sei $M$ ein $R$-Modul" statt "Sei $(R, +_R, \cdot_R, M, +, \cdot)$ ein Modul"
				
				\item "Sei $M$ ein Modul" statt "Es gebe einen Ring $R$ so, dass $M$ ein $R$-Modul ist"
			\end{itemize}
			
			\item Statt "$R$-Modul" sagt man auch "Modul \"uber $R$"	
			
			\item Vektorr\"aume sind Moduln \"uber K\"orper. Viele Sprechweisen (wie "Skalar", "Linearkombination", nicht jedoch "Vektor") \"ubertragen wir stillschweigend von Vektorr\"aumen auf Moduln, ebenso 
			Konventionen (wie "Punkt vor Strich").
			
			\item Abelsche Gruppen "sind" $\Z$-Moduln. Sei $G$ eine abelsche Gruppe. Dann gibt es genau eine Skalarmultiplikation $\cdot:\Z\times G \to G$ verm\"oge derer $G$ zu einem $\Z$-Modul wird, n\"amlich die nat\"urliche, die durch 
			\[n \cdot a := \begin{cases}
				\underbrace{a + a + \cdots + a}_{n\textrm{-mal}} & \textrm{falls } n > 0\\
				0 & \textrm{falls } n = 0\\
				\underbrace{-a - a - \cdots - a}_{(-n)\textrm{-mal}} & \textrm{falls } n < 0
			\end{cases}\]
			gegeben ist.
			
			\item $(\overset{\rightarrow}{D})$ besagt, dass f\"ur alle $a \in R$ die Abbildung $M \to M, x \mapsto ax$ ein Gruppenhomomorphismus ist. Insbesondere gilt $a \cdot 0 = 0$ und $a \cdot (-x) = -ax$ f\"ur alle $a \in R, x \in M$.
			
			$(D')$ besagt, dass f\"ur alle $x \in M$ die Abbildung $R \to M, a \mapsto ax$ ein Gruppenhomomorphismus ist. Insbesondere gilt $0 \cdot x = 0$ und $(-a) \cdot x = -ax$ f\"ur alle $a \in R, x \in M$.
		\end{enumerate}
		
	\end{bemerkung}
	
	\begin{beispiele}\label{1.1.3}
			\begin{enumerate}[(a)]
			\item Nullmoduln $\set{0}$
			
			\item Sei $A$ ein Unterring des Ringes $B$. Dann ist $B$ ein $A$-Modul verm\"oge der Skalarmultiplikation $\cdot: A \times B \to B, (a, x) \mapsto ax$
			
			Insbesondere ist jeder Ring ein Modul \"uber sich selbst.
			
			\item Sei $R$ ein kommutativer Ring und $n \in \N_0$. Dann wird die abelsche Gruppe $R^n$ zu einem $R^{n\times n}$-Modul verm\"oge der Skalarmultiplikation
			\[\cdot: R^{n\times n} \times R^n \to R^n, (A, x) \mapsto Ax\]
			Dies folgt aus den Rechenregeln f\"ur Matrixmultiplikation.
		\end{enumerate}
	\end{beispiele}
	
	\begin{def-prop-satz-not}\label{1.1.4}
		Sei $R$ ein Ring. Die folgenden f\"ur die Theorie der $R$-Moduln grundlegenden Begriffe und Resultate sind eine direkte Verallgemeinerung der entsprechenden Tatsachen f\"ur Vektorr\"aume (also f\"ur den Fall, dass $R$ ein K\"orper) und f\"ur abelsche Gruppen (also $R = \Z$) aus der Linearen Algebra:
		
		\begin{enumerate}[(a)]
			\item Genauso wie bei Vektorr\"aumen f\"uhrt man \emph{direkte Produkte}\index{Modul@\textbf{Modul}!Direktes Produkt} von $R$-Moduln ein.
			
			\item Sind $M$ und $N$ $R$-Moduln, so hei\ss t $N$ ein \emph{Untermodul}\index{Modul@\textbf{Modul}!Untermodul} von $M$, wenn die $N$ zugrunde liegende abelsche Gruppe eine Untergruppe der $M$ zugrunde liegenden abelschen Gruppe ist und 
			\[\forall a \in R: \forall x \in M: a \cdot_N x = a \cdot_M x\]
			
			Ein Untermodul eines Moduls ist offenbar durch seine Tr\"agermenge (d.h. seine zugrunde liegende Menge) eindeutig bestimmt.
			
			Ist $M$ ein $R$-Modul und $N \subseteq M$, so ist $N$ offenbar genau dann (Tr\"agermenge) ein(e) Untermodul(s) von $M$, wenn $0\in N, \forall x, y\in N: x + y \in N, \forall a \in R: \forall x \in N: ax \in N$
			
			\item Sei $M$ ein Modul und $(N_i)_{i \in I}$ eine Familie von Untermoduln von $M$. Dann ist 
			$\bigcap_{i \in I} N_i := \bigcap \set{N_i | i \in I}$ (mit $\bigcap_{i \in I} N_i = M$, falls $I = \emptyset$)
			wieder ein Untermodul von $M$ und zwar der gr\"o\ss te Untermodul von $M$, der in allen $N_i$ enthalten ist.
			
			Weiter ist auch $\sum_{i \in I} N_i := \set{\sum_{i \in I} x_i | (x_i)_{i \in I} \in \prod_{i\in I} N_i, \set{i \in I| x_i \neq 0} \textrm{endlich}}$ Untermodul von $M$ und zwar der kleinste Untermodul von $M$, der alle $N_i$ enth\"alt.
			
			\item Sei $M$ ein $R$-Modul. Ist $x \in M$, so ist $Rx := \set{ax| a \in R}$ ein Untermodul von $M$ und zwar der kleinste Untermodul, der $x$ enth\"alt.
			
			Ist $(x_i)_{i \in I}$ eine Familie von Elementen von $M$, so ist $\sum_{i \in I} Rx_i$ der kleinste Untermodul von $M$, der alle $x_i$ enth\"alt.
			
			Man nennt ihn den von den $x_i$ ($i \in I$) (oder $\set{x_i|i \in I}$) erzeugten Untermodul von $M$ (oder lineare H\"ulle der Span von $\set{x_i| i \in I}$). 
			
			Man nennt $M$ \emph{zyklisch}\index{Modul@\textbf{Modul}!Zyklische Moduln}, wenn $M$ von einem Element erzeugt wird, d.h. es ein $x \in M$ gibt mit $M = Rx$. Man nennt $M$ endlich erzeugt (e.e.), wenn $M$ von endlich vielen Elementen
			erzeugt wird, d.h. es ein $n \in \N_0$ und $x_1, \dots, x_n \in M$ gibt mit 
			\[M = Rx_1 + \cdots + Rx_n := \sum_{i=1}^{n} Rx_i := \sum_{i \in \set{1, \dots, n}} Rx_i\] 
			
			\item Sei $M$ ein $R$-Modul. Eine Familie $(x_i)_{i\in I}$ in $M$ hei\ss t \emph{linear unabh\"angig}\index{Modul@\textbf{Modul}!Linear unabh\"angig (l.u.)} (l.u.), wenn f\"ur alle $n\in \N_0$, alle paarweise verschiedenen $i_1, \dots, i_n \in I$ und alle $a_1, \dots, a_n \in I$ gilt 
			\[\sum_{j=1}^{n} a_jx_{i_j} = 0 \Rightarrow a_1 = \cdots = a_n = 0\]
			
			Weiter nennt man $x_1, \dots, x_n$ linear unabh\"angig, wenn $(x_1, \dots, x_n) = (x_i)_{i \in \set{1, \dots, n}}$ linear unabh\"angig ist, d.h. f\"ur alle $a_1, \dots, a_n \in R$ gilt 
			\begin{align}
				a_1x_1 + \cdots a_nx_n = 0 \Rightarrow a_1 = \cdots = a_n = 0 \label{lin_ind}
			\end{align}
			
			Schlie\ss lich hei\ss t eine Menge $F\subseteq M$ linear unabh\"angig, wenn $(x)_{x \in F}$ linear unabh\"angig ist, d.h. f\"ur alle $n\in \N_0$, alle paarweise verschiedenen $x_1, \dots, x_n \in F$ und alle $a_1, \dots, a_n \in R$ wieder \ref{lin_ind} gilt.
			
			\item Sei $M$ ein Modul. Eine Familie $(x_i)_{i \in I}$ in $M$ hei\ss t eine \emph{Basis}\index{Modul@\textbf{Modul}!Basis} von $M$, wenn sie $M$ erzeugt und linear unabh\"angig ist. Weiter
			sagt man $x_1, \dots, x_n \in M$ bilden eine Basis von $M$, wenn $(x_1, \dots, x_n) = (x_i)_{i \in \set{1, \dots, n}}$ eine Basis von $M$ ist. Schlie\ss lich hei\ss t $B \subseteq M$ eine Basis, wenn $B$ den Modul $M$ erzeugt und linear unabh\"angig ist.
			
			\item Seien $M$ und $N$ $R$-Moduln. Dann hei\ss t $f$ ein \emph{($R$-)(Modul-)Homomorphismus}\index{Modul@\textbf{Modul}!Homomorphismus} oder eine $(R-)$ lineare Abbildung von $M$ nach $N$, wenn $f:M\to N$ ein Gruppenhomomorphismus der $M$ und $N$ zugrundeliegenden abelschen Gruppen ist und 
			\[\forall a \in R: \forall x \in M: f(ax) = af(x) \]
			
			Ein Modulhomomorphismus $f:M\to N$ hei\ss t Einbettung/Monomorphismus (Epimorphismus, Isomorphismus), wenn $f$ injektiv (surjektiv, bijektiv) ist. 
			
			Ein Modulhomomorphismus $f:M \to M$ hei\ss t \emph{(Modul-)Endomorphismus}\index{Modul@\textbf{Modul}!Homomorphismus!Endomorphismus} von $M$. Ein Endomorphismus, der ein Isomorphismus ist, hei\ss t \emph{Automorphismus}\index{Modul@\textbf{Modul}!Automorphismus}. Es hei\ss en $M$ und $N$ \emph{isomorph}, in Zeichen $M \cong N$, wenn es einen Isomorphismus $M \to N$ gibt.
			
			Hintereinanderschaltungen von Modulhomomorphismen sind wieder Modulhomomorphismen. Umkehrabbildungen von Modulisomorphismen sind wieder Modulisomorphismen.
			
			\item Sei $M$ ein $R$-Modul. Eine \emph{Kongruenzrelation}\index{Modul@\textbf{Modul}!Kongruenzrelation} auf $M$ ist eine \"Aquivalenzrelation $\equiv$ der $M$ zugrundeliegenden Menge, f\"ur die gilt
			\[\forall x, y, x', y' \in M: (x \equiv x' \land y \equiv y') \Rightarrow x + y \equiv x' + y'\]
			und
			\[\forall x, x' \in M: \forall a \in R: x \equiv x' \Rightarrow ax \equiv ax'\]
			
			Diese Definition wurde gerade so gemacht, dass 
			\[+:( M/\equiv) \times (M/\equiv) \to (M/\equiv), (\overline{x}, \overline{y}) \mapsto \overline{x + y}\]
			und
			\[\cdot: R \times (M/\equiv) \to (M/\equiv), (a, \overline{x}) \mapsto \overline{ax}\]
			wohldefiniert sind.
			
			Ist $M$ ein $R$-Modul und $\equiv$ eine Kongruenzrelation auf $M$, so wird die Quotientenmenge $M/\equiv$ verm\"oge der Addition $+$ und der Skalarmultiplikation $\cdot$ ein $R$-Modul, wie man
			durch direktes Nachrechnen sieht. Die Zuordnungen
			\begin{align*}
				\equiv  &\overset{f}{\mapsto} \overline{0}\\
				\equiv_N &\overset{g}{\mapsfrom} N
			\end{align*}
			vermitteln eine Bijektion zwischen der Menge der Kongruenzrelationen auf $M$ und der Menge der Untermoduln von $M$, wobei $\equiv_N$ gegeben ist durch
			\[a \equiv_N b :\Leftrightarrow a - b \in N\]
			f\"ur $a, b \in M$.
			
			Ist $N$ ein Untermodul von $M$, so nennt man $M/N := M/\equiv_N$ auch den \emph{Quotientenmodul}\index{Modul@\textbf{Modul}!Quotientenmodul} von $M$ nach $N$.
			
			\item Sind $M$ und $N$ $R$-Moduln und $f:M \to N$ ein Modulhomomorphismus, so ist der \emph{Kern}\index{Modul@\textbf{Modul}!Homomorphismus!Kern} $\ker f := \set{x \in M| f(x) = 0}$ von $f$ ein Untermodul von $M$ und das \emph{Bild}\index{Modul@\textbf{Modul}!Homomorphismus!Bild} $\im f := \set{f(x) | x \in M}$ von $f$ ist ein Untermodul von $N$.
			
			\item Homomorphiesatz: Seien $M$ und $N$ $R$-Moduln und $L$ ein Untermodul von $M$ und $f:M \to N$ ein Modulhomomorphismus mit $L \subseteq \ker f$. Dann gibt es (genau) einen Modulhomomorphismus $\overline{f}: (M/L) \to N$ mit $\overline{f}(\overline{x}) = f(x)$ f\"ur alle $x \in M$.
			
			Ferner gilt, dass
			\begin{itemize}
				\item $\overline{f}$ ist injektiv $\Leftrightarrow L = \ker f$ und
				\item $\overline{f}$ ist surjektiv $\Leftrightarrow f$ ist surjektiv
			\end{itemize}
			
			\item Isomorphiesatz: Seien $M$ und $N$ $R$-Moduln und $f:M \to N$ ein Modulhomomorphismus. Dann ist $\overline{f}:(M/\ker f) \to \im f$ definiert durch $\overline{f}(\overline{x}) = f(x)$ f\"ur alle $x\in M$ ein $R$-Modulisomorphismus. Insbesondere ist $M/\ker f \cong \im f$
		\end{enumerate}
	\end{def-prop-satz-not}
	
	\begin{bemerkung}\label{1.1.5}
		Sei $R$ ein kommutativer Ring. Dann sind die Untermoduln des $R$-Modul $R$ [$\to$\ref{1.1.3}(b)] (oder kurz gesagt die $R$-Untermoduln von $R$) genau die Ideale des Ringes $R$.
		Insbesondere sind zum Beispiel das von einem $a \in R$ erzeugte Ideal und der davon erzeugte Untermodul als Menge dasselbe $(a)_R = Ra \overset{R\ komm.}{=} \set{ab | b \in R} = aR$.
		Trotzdem macht es vom Sinn her einen Unterschied. ob man $(a)$ oder $Ra$ schreibt. Zum Beispiel meint man mit $R/(a)$ den Ring und mit $R/aR$ den $R$-Modul (deren zugrundeliegenden abelschen Gruppen dieselben sind)
	\end{bemerkung}

	\begin{warnung}\label{1.1.6}
		F\"ur den mit Vektorr\"aumen, aber nicht mit Moduln vertrauten H\"orern ist Vorsicht geboten:
		\begin{enumerate}[(a)]
			\item In einem $R$-Modul $M$ kann $ax = 0$ f\"ur ein $a \in R$ und ein $x \in M$ gelten, ohne dass $a = 0$ oder $x = 0$ gilt (zum Beispiel $2 \cdot \overline{1} = \overline{2} = 0$ im $\Z$-Modul $\Z/2\Z$)
			
			\item Nicht jeder Modul hat eine Basis:
			zum Beispiel ist jedes Element des $\Z$-Moduls $\Z/2\Z$ linear abh\"angig, denn $1 \cdot \overline{0} = \overline{0} = 0$ und $2 \cdot \overline{1} = \overline{2} = 0$ in $\Z/2\Z$, womit die einzige linear unabh\"angige Teilmenge von $\Z/2\Z$ die leere Menge is, welche aber $\Z/2\Z$ nicht erzeugt.
		\end{enumerate}
	\end{warnung}	

	\begin{beispiele}\label{1.1.7}
		\begin{enumerate}[(a)]
			\item F\"ur jeden Ring $R$ ist $R^n$ ein $R$-Modul mit der \emph{Standardbasis}\index{Modul@\textbf{Modul}!Standardbasis} $\underline{e} = (e_1, \dots, e_n)$, wobei $e_i := \begin{pmatrix}
				0\\
				\vdots\\
				0\\
				1\\
				0\\
				\vdots\\
				0
			\end{pmatrix}$ mit einer $1$ an der $i$-ten Stelle.
		
			\item $\R^2$ ist ein zyklischer $\R^{2 \times 2}$ Modul [$\to$\ref{1.1.3}(c)], welcher von jedem $x \in \R^{2\times 2} \setminus \set{0}$ erzeugt ist. Da aber jedes $x \in \R^{2 \times 2}$ linear abh\"angig ist, hat dieser Modul keine Basis.  
		\end{enumerate}
	\end{beispiele}
	\newpage

	\section{Direkte Summen von Moduln und freie Moduln}
	\begin{definition}\label{1.2.1}
		Sei $R$ ein Ring und $(M_i)_{i \in I}$ eine Familie von $R$-Moduln. Dann nennt man den $R$-Untermodul 
			\[\bigoplus_{i \in I} M_i := \set{x \in \prod_{i\in I}M_i| \supp(x)\ \mathrm{endlich} } \]
		von $\displaystyle\prod_{i\in I} M_i$ die \emph{(\"au\ss ere) direkte Summe}\index{Modul@\textbf{Modul}!\"Au\ss ere Direkte Summe} der $M_i$ ($i \in I$). Man fasst $M_j$ ($j \in I$ h\"aufig) als Untermodul von $\displaystyle\bigoplus_{i \in I} M_i$ auf verm\"oge der Einbettung 
		 \[\rho_j:M_j \to \prod_{i\in I} M_i, x \mapsto \brac{i \mapsto \begin{cases}
		 		x & \textrm{falls } i = j\\
		 		0 & \textrm{sonst}
		 \end{cases}}\] 
	 	
	 	\noindent Ist $M_i = M$ f\"ur alle $i \in I$, so schreibt man \[M^{(I)} := \bigoplus_{i \in I} M \subseteq \prod_{i\in I} M = M^I\]
	\end{definition}
	
	\begin{proposition}\label{1.2.2}
		Sei $R$ ein Ring, $(M_i)_{i \in I}$ eine Familie von Modulhomomorphismen $f_i: M_i \to N$. Dann gibt es genau einen Modulhomomorphismus $f:\bigoplus_{i \in I} M_i \to N$ mit $f\big|_{M_i} = f_i$ f\"ur alle $i \in I$ ($f \circ \rho_i = f_i$ f\"ur $i \in I$).
	\end{proposition}
	\begin{proof}
		F\"ur jedes $x \in \bigoplus_{i \in I} M_i$ gilt $x = \sum_{i \in \supp(x)} \rho_i (x(i))$. Um $f \circ \rho_i = f_i$ f\"ur $i \in I$ zu erf\"ullen, kann man daher nur 
		\[f: \bigoplus_{i \in I} M_i \to N, x \mapsto \sum_{i \in I} f_i(x(i))\] 
		definieren. Man \"uberpr\"uft sofort, dass das so definierte $f$ ein Homomorphismus ist.
	\end{proof}
	
	\begin{prop-def}\label{1.2.3}
		Sei $R$ ein Ring, $M$ ein $R$-Modul und $(N_i)_{i \in I}$ eine Familie von Untermoduln on $M$. Dann sind die folgenden Bedingungen \"aquivalent
		\begin{enumerate}[(a)]
			\item Die Abbildung von der \"au\ss eren direkten Summe $\bigoplus_{i \in I} N_i$ nach $M$, die auf $N_i$ die Identit\"at ist, ist ein Isomorphismus
			\item $M = \sum_{i \in I} N_i$ und f\"ur alle $n \in \N$, paarweise verschiedenen $i_1, \dots, i_n \in I$ und alle $x_1 \in N_{i_1}, \dots, x_n \in N_{i_n}$ gilt 
				\[(x_1 + \cdots + x_n = 0) \Rightarrow (x_1 = \cdots = x_n = 0)\]  
		\end{enumerate}
	Gelten diese Bedingungen, so nennt man $M$ die \emph{(innere) direkte Summe}\index{Modul@\textbf{Modul}!Innere Direkte Summe} der $N_i$ ($i \in I$) und schreibt (angesichts der Isomorphismus aus $(a)$) wieder $M = \bigoplus_{i \in I} N_i$
	\end{prop-def}

	\begin{definition}\label{1.2.4}
		Sei $R$ ein Ring, $M$ ein $R$-Modul und $x \in M$. Der Kern des $R$-Modulhomomorphismus $R \to M, a \mapsto ax$ nennt man \emph{Annihilator}\index{Modul@\textbf{Modul}!Annihilator} von $x$, in Zeichen $\ann(x) = \set{a \in R| ax = 0}$.\newline
		Es hei\ss t $x$ ein \emph{Torsionselement}\index{Modul@\textbf{Modul}!Torsionselement} von $M$ wenn $\ann(x) \neq \set{0}$.
	\end{definition}

	\begin{satz}\label{1.2.5}
		Sei $R$ ein Ring, $M$ ein $R$-Modul und $B \subseteq M$. Dann sind \"aquivalent
		\begin{enumerate}[(a)]
			\item $B$ ist eine Basis von $M$
			\item $M = \bigoplus_{x \in B} Rx$ und $B$ enth\"alt kein Torsionselement
			\item F\"ur jeden $R$-Modul $N$ und jede Abbildung $g:B \to N$ gibt e genau einen Homomorphismus $f:M \to N$ mit $f\big|_b = g$.
		\end{enumerate}
	
		\begin{proof}\hfill
			\begin{enumerate}
				\item[$(a) \Rightarrow (b)$] klar
				\item[$(b) \Rightarrow (c)$] Gelte (b). Sei $N$ ein $R$-Modul und $g:B \to N$ eine Abbildung. Zu zeigen sind Existenz und Eindeutigkeit eines Homomorphismus $f: M \to N$ mit $f\big|_B = g$
					\begin{itemize}
						\item Eindeutigkeit: klar aus $M = \sum_{x \in B} Rx$
						\item Existenz: Fixiere zun\"achst $x \in B$. Dann ist $R \to Rx, a \mapsto ax$ ein Isomorphismus (mit Kern $\ann(x)$), dessen Umkehrfunktion ein Isomorphismus $Rx \to R$ ist, der $x$ auf $1$ abbildet. Schaltet man den Homomorphismus $R \to N, a \mapsto ag(x)$ dahinter, so erh\"alt man einen Homomorphismus $Rx \to N$, der $x$ auf $g(x)$ abbildet. Da $x \in B$ beliebig war, erh\"alt man mit \ref{1.2.2} einen Homomorphismus $f: M = \bigoplus_{x \in B} Rx \to N$, der jedes $x \in B$ auf $g(x)$ abbildet.  
					\end{itemize}
				\item[$(c) \Rightarrow (a)$] Gelte (c). Zu zeigen ist, dass $B$ linear unabh\"angig ist und $M$ erzeugt.
				\begin{enumerate}[1.]
					\item $B$ linear unabh\"angig: Seien $x_1, \dots, x_n \in B$ paarweise verschieden und $a_1, \dots, a_n \in R$ mit $a_1x_1 + \cdots + a_nx_n = 0$. Sei $i \in \set{1, \dots, n}$. Zu zeigen ist $a_i = 0$. Gem\"a\ss (c) gibt es einen Homomorphismus $f: M \to R$ mit $f(x_i) = 1$ und $f(x_j) = 0$ f\"ur $j \in \set{1, \dots, n} \setminus \set{i}$.
					Dann \[0 = f(0) = f\brac{\sum_{j=1}^n a_jx_j} = \sum_{j=1}a_jf(x_j) = a_if(x_i) = a_i\]
					
					\item $B$ erzeugt $M$: Nach (c) gibt es einen Homomorphismus $M \to M$, der auf $B$ die Identit\"at ist. Einerseits ist $\id_M$ ein solcher, andererseits auch $\rho \circ f$, wobei $f: M \to N:= \sum_{x \in B} Rx$ der nach (c) existierende Homomorphismus mit $f\big|_{B} = \id_B$ ist und $\iota: N \hookrightarrow M, x \mapsto x$ die Inklusion. Also $\id_M = \iota \circ f$, insbesondere $M = \im(\id_M) = \im(f) = N$
				\end{enumerate}
 			\end{enumerate}
		\end{proof}
	\end{satz}

	\begin{definition}\label{1.2.6}
		Ein Modul hei\ss t \emph{frei}\index{Modul@\textbf{Modul}!Freie Moduln}, wenn er eine Basis besitzt.
	\end{definition}

	\begin{bemerkung}\label{1.2.7}
		Sei $R$ ein Ring, $M$ ein $R$-Modul, $n \in \N_0$ und $x_1, \dots, x_n \in M$. Dann bilden $x_1, \dots, x_n$ genau dann eine Basis von $M$, wenn der Homomorphismus 
		\[R^n \to M, \begin{pmatrix}
			a_1\\\vdots\\a_n
		\end{pmatrix} \mapsto \sum_{i=1}^n a_ix_i\] ein Isomorphismus ist.
	\end{bemerkung}
	
	\begin{bemerkung}\label{1.2.8}
		Ist $M$ ein $\set{0}$-Modul, so ist $M = \set{0}$, denn ist $x \in M$, so ist $x=1 \cdot x = 0 \cdot x = 0$
	\end{bemerkung}

	\begin{lemma}\label{1.2.9}
		Ein endlich erzeugter Modul hat niemals eine unendliche Basis.
		
		\begin{proof}
			Sei $M$ ein endlich erzeugter $R$-Modul, etwa $M = \sum_{i=1}^n Rx_i$ mit $x_1, \dots, x_n \in M$. \newline
			Annahme: $B$ ist eine unendliche Basis von $M$. Dann gibt es f\"ur jedes $i \in \set{1, \dots, n}$ ein endliches $B_i \subseteq B$ mit $x_i \in \sum_{y \in B_i}Ry$. Dann ist $B' := B_1 \cup \cdots \cup B_n \subseteq B$ endlich mit $M = \sum_{y \in B'}Ry$. Da $B$ unendlich h ist, gibt es ein $z \in B\setminus B'$
			
			Nun gilt $z \in \sum_{y \in B'}Ry$, was im Widerspruch zur linearen Unabh\"angigkeit von $B$ steh, au\ss er wenn $1 = 0$ in $R$, d.h. $R = \set{0}$. Im letzten Fall ist aber nach \ref{1.2.8} nichts zu zeigen.
		\end{proof}
	\end{lemma}

	\begin{bemerkung}\label{1.2.10}
		\begin{enumerate}[(a)]
			\item 
			Jeder Modul \"uber dem Nullring hat genau zwei Basen, n\"amlich $\emptyset$ und $\set{0}$. In der Tat: Nach \ref{1.2.8} handelt es sich um den Nullmodul und in einem $\set{0}$-Modul ist $0$ linear unabh\"angig.
			\item In den \"Ubungen geben wir einen Ring $R \neq 0$, der als $R$-Modul zu $R^2$ isomorph ist. Durch Induktion schlie\ss t man, dass $R \cong R^n$ f\"ur alle $n \in \N$. Damit besitzt $R$ als $R$-Modul f\"ur jedes $n \in \N$ eine $n$-elementige Basis, aber nach \ref{1.2.9} keine unendliche Basis.
		\end{enumerate}
	\end{bemerkung}

	\begin{satz}\label{1.2.11}
		Sei $R$ ein kommutativer Ring mit $1 \neq 0$. Dann sind je zwei Basen eines $R$-Moduls entweder beide unendlich oder beide endlich mit der selben Anzahl von Elementen
		
		\begin{proof}
			Sei $M$ ein $R$-Modul mit Basen $B$ und $C$. Im Fall von $|B| =  \infty = |C|$ sind wir fertig, sonst ist $M$ endlich erzeugt und daher $m = |B|, n = |C| \in \N_0$ nach Lemma \ref{1.2.9}. Nach \ref{1.2.7} gilt $R^n \cong M \cong R^m$, somit reicht es zu zeigen: Sei $R$ ein kommutativer Ring und $m, n \in \N_0, m > n$ mit $R^m \cong R^n$ als $R$-Modul, dann gilt $1 = 0$ in $R$. 
			
			\noindent Um dies zu zeigen, w\"ahle zueinander inverse $R$-Modulisomorphismen $f:R^n \to R^m, g: R^m \to R^n$. Bezeichne mit $\underline{x} = (x_1, \dots, x_n)$ und $\underline{y} = (y_1, \dotsm y_m)$ die Standardbasen des $R^n$ und $R^m$. W\"ahle $A=(a_{ij})_{1\leq i \leq m, 1 \leq j \leq n}\in R^{m \times n}$ mit $f(x_j) = \sum_{i=1}^m a_{ij}y_i$ f\"ur $j \in \set{1, \dots, n}$ und $B = (b_{ji})_{1 \leq j \leq n, 1 \leq i \leq m} \in R^{n \times m}$ mit $f(y_i) = \sum_{j=1}^n b_{ji}x_j$ f\"ur $i \in \set{1, \dots, m}$. Dann gilt f\"ur $k \in \set{1, \dots, m}$
				\begin{align*}
					y_k &= (f \circ g)(y_k) = f(g(y_k))\\
					&= f\brac{\sum_{j=1}^n b_{jk}x_j}\\
					&= \sum_{j=1}^n b_{jk}f(x_j)\\
					&= \sum_{j=1}^n b_{jk}\sum_{i=1}^ma_{ij}y_i\\
					&= \sum_{i=1}^m\brac{\sum_{j=1}^n b_{jk}a_{ij} }y_i \overset{R\ \mathrm{komm.}}{=} \sum_{i=1}^m\brac{\sum_{j=1}^n a_{ij}b_{jk} }y_i
				\end{align*} 
			und daher
			\[\sum_{j=1}^n a_{ij}b_{jk} = \begin{cases}
				1 & \mathrm{falls }\ k = i\\
				0 & \mathrm{sonst}
			\end{cases}\] 
			f\"ur alle $i, k \in \set{1, \dots, m}$, d.h. $AB = I_m$.\newline
			Wegen $n < m$ k\"onnen wir $\displaystyle A' := (A\ \underbrace{0}_{(m-n)-\mathrm{Spalten}}) \in R^{m\times m}$ und $B' := \begin{pmatrix}
				B\\0
			\end{pmatrix} \in R^{m \times m}$ (mit $m-n$ $0$-Zeilen) setzen, so dass $A'B' = AB = I_m$. \newline
			Mit dem Determinantenproduktsatz folgt 
			\[0 = 0\cdot 0 = (\det A')(\det B') = \det(A'B') = 1\] 
		\end{proof}
	\end{satz}

	\begin{bemerkung}\label{1.2.12}
		Statt den Determinantenproduktsatz \"uber kommutativen Ringen zu verwenden, kann man den Beweis des letzten Satzes auch mit der Theorie kommutativer Ringe auf die Dimensionstheorie von Vektorr\"aumen zur\"uckspielen. 
		
		\noindent Sei $R$ ein kommutativer Ring mit $1 \neq 0$, $m, n\in \N_0$ mit $R^m \cong R^n$. Wir zeigen $m = n$.
		
		\begin{proof}
			W\"ahle ein maximales Ideal $\mathfrak{m}$ von $R$. W\"ahle einen $R$-Modulisomorphismus $f:R^m \to R^n$. Betrachte die $R$-Untermoduln
			\[\mathfrak{m}R^m := \set{\sum_{i=1}^n a_ix_i|n \in \N_0, a_i \in \mathfrak{m}, x_i \in R^m} = \mathfrak{m}^m\]
			von $R^m$ und 
			\[f(\mathfrak{m}R^n) = \set{\sum_{i=1}^n a_iy_i|n \in \N_0, a_i \in \mathfrak{m}, y_i \in R^n} = \mathfrak{m}^n\]
			von $R^n$
			
			\noindent Mit dem Isomorphiesatz erhalten wir einen Modulisomorphismus $R^m/\mathfrak{m}^m \to R^n/\mathfrak{m}^n$ und offensichtlich gilt $R^m/\mathfrak{m}^m  \cong (R/\mathfrak{m})^m$ (betrachte z.B. $R^m \to (R/\mathfrak{m})^m$).\newline
			Da nun $(R/\mathfrak{m})^m$ und $(R/\mathfrak{m})^n$ als $R$-Moduln isomorph sind, sind sie auch als $(R/\mathfrak{m})$-Moduln isomorph. F\"ur den K\"orper $K:= R/\mathfrak{m}$ gilt also 
			\[m = \dim_K K^m = \dim_K K^n = n\]
		\end{proof}
	\end{bemerkung}

	\begin{definition}\label{1.2.13}
		Sei $R$ ein kommutativer Ring mit $1 \neq 0$ und $M$ ein freier $R$-Modul mit Basis $B$. Dann hei\ss t $\rk M := |B| \in \N_0 \cup \set{\infty}$ der \emph{Rang}\index{Modul@\textbf{Modul}!Rang} von $M$ [h\"angt nach \ref{1.2.11} nicht von der Wahl der Basis $B$ ab ]
	\end{definition}
	\newpage
	
	\section{Halbeinfache Moduln}
	\begin{notation}\label{1.3.1}
		$0 := \set{0}$ Nullmodul
	\end{notation}
	
	\begin{definition}\label{1.3.2}
		Ein Modul $M$ hei\ss t \emph{einfach}\index{Modul@\textbf{Modul}!Einfache Moduln} (oder irreduzibel), falls $M \neq 0$ und $0$ und $M$ die einzigen Untermoduln von $M$ sind.
	\end{definition}
	
	\begin{bemerkung}\label{1.3.3}
		Sei $N$ ein Untermoduln von $M$.
		\begin{enumerate}[(a)]
			\item Bezeichne $\varphi: M \to M/N$ den kanonischen Epimorphismus. Dann vermitteln die Zuordnungen
			\begin{align*}
				L &\mapsto L/N = \varphi(L) \\
				\varphi^{-1}(P) &\mapsfrom P
			\end{align*}
			Eine Bijektion zwischen der Menge der Untermoduln $L$ von $M$ mit $N \subseteq L$ und der Menge der Untermoduln von $M/N$
					
			\item Es folgt, dass $M/N$ einfach ist genau dann, wenn $N$ ein maximaler echter Untermodul ist.
		\end{enumerate}
	\end{bemerkung}

	\begin{beispiele}\label{1.3.4}
		\begin{enumerate}[(a)]
			\item Sei $R$ ein kommutativer Ring und $I$ ein $R$-Untermodul von $R$, d.h. ein Ideal von $R$ [$\to$\ref{1.1.5}]. Dann ist $R/I$ ein einfacher $R$-Modul $\Leftrightarrow$ $I$ ist ein maximales Ideal von $R$ $\Leftrightarrow$ $R/I$ ist ein K\"orper.
			
			\item Sei $R$ ein Hauptidealring und $p \in R\setminus \set{0}$. Dann ist $R/pR$ ein einfacher Modul genau dann, wenn $p$ irreduzibel in $R$ ist.
			
			\begin{proof}
				\begin{itemize}
					\item[$\Rightarrow$] Ist $(p)$ ein maximales Ideal von $R$, so auch ein Primideal, d.h. $p$ ist prim in $R$ und daher auch irreduzibel in $R$ (wegen $p \neq 0$) 
					
					\item[$\Leftarrow$] Ist $p$ irreduzibel in $R$, so ist $R/(p)$ ein K\"orper und daher ist $(p)$ ein maximales Ideal in $R$.
				\end{itemize}
			\end{proof}
		\end{enumerate}
	\end{beispiele}

	\begin{lemma}\label{1.3.5}
		Sei $R$ ein Ring und $M$ ein $R$-Modul. Es sind \"aquivalent:
		\begin{enumerate}[(i)]
			\item $M$ ist einfach
			\item $M \neq 0$ und jedes Element von $M \setminus \set{0}$ erzeugt $M$
			\item Es gibt einen maximalen echten $R$-Untermodul $N$ von $R$ mit $R/N \cong M$
		\end{enumerate}
	
		\begin{proof}\hfill
			\begin{enumerate}
				\item[$(a) \Rightarrow (c)$] Gelte (a) W\"ahle $x \in M \setminus \set{0}$. Dann ist der Homomorphismus $\varphi:R \to M, a \mapsto ax$ surjektiv und daher $R/N \cong M$ mit $N:=\ker \varphi$. Mit $M$ ist auch $R/N$ einfach, weswegen nach \ref{1.3.3}(b) $N$ ein maximaler echter Untermodul von $R$ ist.
				\item[$(c) \Rightarrow (b)$] trivial
				\item[$(b) \Rightarrow (a)$] trivial
			\end{enumerate}
		\end{proof}
	\end{lemma}

	\begin{lemma}\label{1.3.6}
		Lemma von Schur. \newline
		Sei $R$ ein Ring, $M$ und $N$ einfache $R$-Moduln und $f:M \to N$ ein Homomorphismus. Dann ist $f$ entweder die Nullabbildung oder ein Isomorphismus
		
		\begin{proof}
			Ist $f\neq 0$, so ist $\ker f \neq M$ und $\im f \neq 0$, also $\ker f = 0$ und $\im f = N$.
		\end{proof}
	\end{lemma}

	\begin{definition}\label{1.3.7}
		Ein Modul hei\ss t \emph{halbeinfach}\index{Modul@\textbf{Modul}!Halbeinfache Moduln} (oder vollst\"andig reduzibel), wenn er direkte Summe von einfachen Moduln ist.
	\end{definition}

	\begin{lemma}\label{1.3.8}
		Jeder endlich erzeugte Modul $\neq 0$ besitzt einen einfachen Quotienten.
		
		\begin{proof}
			Sei $M$ ein $R$-Modul und seinen $x_1, \dots, x_n \in M$ mit $0 \neq M = Rx_1 + \cdots + Rx_n$. Zu zeigen: Es gibt einen Untermodul $N$ von $M$ mit $M/N$ einfach.
			
			\noindent Betrachte die durch Inklusion halbgeordnete Menge 
				\[X:=\set{P|P\ \textrm{Untermodul von} \ M, P \subsetneq M} = \set{P|P\ \textrm{Untermodul von} \ M, \set{x_1, \dots, x_n} \nsubseteq P}\]
			
			\noindent Jede Kette $K \subseteq X$ besitzt eine obere Schranke in $X$ ($0$ f\"ur $K = \emptyset$, da $M \neq 0$ und $\bigcup K$ f\"ur $K \neq \emptyset$, da $\set{x_1, \dots, x_n}$ endlich)
			
			\noindent Nach dem Lemma von Zorn gibt es daher ein maximales Element $N$ in $X$. Gem\"a\ss \ \ref{1.3.3}(b) ist $M/N$ einfach. 
		\end{proof}
	\end{lemma}

	\begin{definition}\label{1.3.9}
		Sei $M$ ein Modul und $N$ ein Untermodul von $M$. Dann hei\ss t $N$ ein \emph{direkter Summand}\index{Modul@\textbf{Modul}!Direkter Summand} von $m$, wenn es einen Untermodul $P$ von $M$ gibt mit $M = N \oplus P$.
	\end{definition}

	\begin{satz}\label{1.3.10}
		Sei $M$ ein Modul. Dann sind folgende Aussage \"aquivalent
		\begin{enumerate}[(a)]
			\item $M$ ist halbeinfach
			
			\item $M$ ist die Summe seiner einfacher Untermoduln
			
			\item Jeder Untermodul von $M$ ist ein direkter Summand von $M$.
		\end{enumerate}	
		
		\begin{proof}\hfill\newline
			(a)$\Rightarrow$(b) ist klar\newline
			(b)$\Rightarrow$(c). Gelte (b) und sei $N$ ein Untermodul von $M$.
				\[X:= \set{P| P \ \mathrm{Untermodul von }\ M, N \cap P = 0}\]
			
			\noindent Jede Kette $K \subseteq X$ besitzt eine obere Schranke in $X$ ($0$ f\"ur $K = \emptyset$, $\bigcup K$ f\"ur $K \neq \emptyset$)
			
			\noindent Nach dem Lemma von Zorn gibt es daher ein maximales Element $P$ in $X$. Um $M = N + P$ zu zeigen, reicht es wegen (b) zu zeigen, dass jeder einfache Untermodul $L$ von $M$ in $N + P$ enthalten ist. Sei also $L$ ein einfacher Untermodul von $M$. Dann ist entweder $L \cap (N + P) = 0$ oder $L \cap (N + P) = L$. Im letzteren Fall sind wir fertig.
			
			\noindent Der erste Fall tritt aber nicht ein:\newline
			Ist $L \cap (N+P) = 0$, so $(L+P)\cap N = 0$ (ist $x \in L$ und $y \in P$ mit $x+y \in N$, so $x \in L\cap (N+P) = 0$ und daher $y \in N\cap P = 0$), woraus wegen der Maximalit\"at von $P$ folgt $P = L + P$, also $L \subseteq P \lightning$
			
			
			\noindent (c)$\Rightarrow$(a). Gelte (c).\newline
			Hilfsbehauptung: Jeder Untermodul eines Untermoduls $N$ von $M$ ist ein direkter Summand von $N$.
			
			\noindent Begr\"undung: Sei $N$ ein Untermodul von $M$ und $P$ ein Untermodul von $N$.
			W\"ahle $Q$ mit $M = P \oplus Q$. Setze $R = Q \cap N$. Wir zeigen $N = P \oplus R$. Es ist klar, dass $P \cap R = 0$ (denn $P \cap Q = 0$) und $P + R \subseteq N$. Zu zeigen it also noch $N \subseteq P + R$. \newline
			Sei hierzu $x \in N$. Schreibe $x = p + q$ mit $p \in P$ und $q \in Q$, dann $q = x - p \in N \cap Q = R$.  
			
			\noindent Betrachte nun die durch Inklusion halbgeordnete Menge 
				\[X:= \set{Y\big|Y\ \mathrm{Menge\ von\ einfachen\ Untermoduln\ von}\ M\ \mathrm{mit}\ \sum_{N \in Y}N = \bigoplus_{N \in Y}N}\]
				
			\noindent Sei $K$ eine Kette in $X$. Wir behaupten, dass dann $Z:= \bigcap K \in X$ gilt und $Z$ eine obere Schranke von $K$ in $X$ ist. \newline
			Zu zeigen: $\sum_{N \in Z} N = \bigoplus_{N \in Z} N$
			
			\noindent Seien nun $n \in \N$ und $N_1, \dots, N_n \in Z$ paarweise verschieden und $x_1 \in N_1, \dots, x_n \in N_n$ mit $x_1 + \cdots + x_n = 0$ [$\rightarrow$\ref{1.2.3}(b)]. Da $X$ eine Kette ist, gibt es $Y \in K$ mit $\set{N_1, \dots, N_n} \subseteq Y$. Wegen $\sum_{N \in Y}N = \bigoplus_{N \in Y}N$ folgt mit \ref{1.2.3}(b), dass $x_1 = \cdots = x_n = 0$.
			
			\noindent Da die Kette $K \subseteq X$ beliebig war, gibt es nach dem Lemma von Zorn ein in $X$ maximales Element $Z$. Setze $P = \sum_{N \in Z} N = \bigoplus_{N \in Z}N$. Wir zeigen $M = P$.
			
			\noindent Angenommen $M \setminus P \neq \emptyset$. W\"ahle gem\"a\ss\ (c) $Q$ mit $M = P \bigoplus Q$. Dann $Q \neq 0$. W\"ahle einen endlich erzeugten Untermodul $Q'\neq 0$ von $Q$. Nach Lemma \ref{1.3.8} gibt es einen Untermodul $Q''$ von $Q'$ mit $Q'/Q''$ einfach.
			
			\noindent W\"ahle gem\"a\ss\ Hilfsbehauptung $R$ mit $Q' = Q'' \bigoplus R$. Dann ist $R \subseteq Q' \subseteq Q$ und daher $P \cap R = 0$. Weiter ist $R\cong Q'/Q''$ einfach. Es folgt $\sum_{N \in Z \cup \set{R}}N = \bigoplus_{N \in Z \cup \set{R}}N$. Daher ist $Z\cup \set{R} \in X$. Wegen der Maximalit\"at von $Z$ in $X$ gilt $R \in Z$ und daher $R \subseteq P \lightning$.
 		\end{proof} 
	\end{satz}

	\begin{korollar}\label{1.3.11}
		Direkte Summen, Untermoduln und Quotienten von halbeinfachen Moduln sind halbeinfach.
		
		\begin{proof}
			direkte Summen: klar nach \ref{1.3.7}\newline
			Untermoduln: Sei $N$ ein Untermodul des halbeinfachen Moduls $M$. Wir verwenden \ref{1.3.10}(c) um zu zeigen, dass $N$ auch halbeinfach ist. Sei also $L$ ein Untermodul von $N$. Da $M$ halbeinfach ist, gibt es einen Untermodul $P$ von $M$ mit $M = L \oplus P$. Dann gilt $N = L \oplus (P \cap N)$, wie man sofort sieht.\newline
			Quotienten: Sei $N$ ein Untermodul des halbeinfachen Moduls $M$. Zu zeigen: $M/N$ ist halbeinfach.\newline
			W\"ahle einen Untermodul $P$ von $M$ mit $M = N \oplus P$. Dann ist $M/N \cong P$ halbeinfach nach dem gerade Gezeigten (betrachte den Homomorphismus $M = N \oplus P \to P, x+y \mapsto y$ und wende den Homomorphiesatz an).
		\end{proof}
	\end{korollar}
	\newpage
	
	\section{Noethersche und artinsche Moduln}
	\begin{definition}\label{1.4.1}
		Ein Modul $M$ hei\ss t \emph{noethersch}\index{Modul@\textbf{Modul}!Noethersche Moduln} bzw. \emph{artinsch}\index{Modul@\textbf{Modul}!Artinsche Moduln}, wenn jede aufsteigende bzw. absteigende Kette von Untermoduln $M_1 \subseteq M_2 \subseteq \cdots$ bzw. $M_1 \supseteq M_2 \supseteq \cdots$ von $M$ station\"ar wird (d.h. $\exists k \in \N: \forall n \geq k: M_n = M_k$).
		
		\noindent Ein Ring $R$ hei\ss t noethersch bzw. artinsch, wenn er als $R$-Modul noethersch bzw. artinsch ist.
	\end{definition}

	\begin{bemerkung}\label{1.4.2}
		Sei $R$ ein kommutativer Ring
		
		\begin{enumerate}[(a)]
			\item $R$ ist genau dann noethersch, wenn jede aufsteigende Kette von Idealen in $R$ station\"ar wird [$\to$\ref{1.1.5}]
			
			\item Ist $S = R[a_1, \dots, a_n]$ ein kommutativer Ring mit $n \in \N_0, a_1, \dots, a_n \in S$, so besagt der \emph{Hilbersche Basissatz}\index{Modul@\textbf{Modul}!Hilbertscher Basissatz}:
				$R$ noethersch $\Rightarrow$ $S$ noethersch.
		\end{enumerate}
	\end{bemerkung}

	\begin{satz}\label{1.4.3}
		Ein Modul ist noethersch genau dann, wenn alle seine Untermoduln endlich erzeugt sind [$\rightarrow$\ref{1.1.4}(d)].
	\end{satz}

	\begin{lemma}\label{1.4.4}
		Seien $L, L'$ und $N$ Untermoduln des Moduls $M$ mit $L \subseteq L'$, $L \cap N = L' \cap N$ und $L+N=L'+N$. Dann gilt $L=L'$
		
		\begin{proof}
			Sei $x \in L'$. Zu zeigen ist $x \in L$. Schreibe $x = l + n$ mit $l \in L$ und $n \in N$. Dann ist $x-l = n \in L' \cap N = L \cap N$ und daher $x = (x-l)+l \in L$. 
		\end{proof}
	\end{lemma}

	\begin{satz}\label{1.4.5}
		Sei $N$ ein Untermodul des Moduls $M$. Dann ist $M$ noethersch bzw. artinsch genau dann, wenn sowohl $N$ als auch $M/N$ noethersch bzw. artinsch ist.
		
		\begin{proof}
			klar mit \ref{1.3.3}(a) und \ref{1.4.4}
		\end{proof}
	\end{satz}

	\begin{korollar}\label{1.4.6}
		Endlich Summen noetherscher bzw. artinscher Moduln sind auch noethersch bzw. artinsch.
		
		\begin{proof}
			Sind $N_1, \dots, N_n$ noethersche bzw. artinsche Untermoduln des Moduls $M$ mit $M = \sum_{i=1}^n N_i$, so gibt es nach \ref{1.2.2} einen Epimorphismus 
				\[\bigoplus_{i=1}^n N_i \to \sum_{i=1}^n N_i\]
			weshalb $M = \sum_{i=1}^nN_i \cong \brac{\bigoplus_{i=1}^n N_i} / L$ f\"ur einen Untermodul $L$ von $\bigoplus_{i=1}^nN_i$ gilt.\newline
			Mit \ref{1.4.5} reicht es daher, die Behauptung f\"ur direkte Summen zu zeigen.\newline
			Durch Induktion nach $n \in \N_0$ zeigen wir daher, dass f\"ur alle noetherschen bzw. artinschen $R$-Moduln $N_1, \dots, N_n$ auch $\bigoplus_{i=1}^n N_i$ noethersch bzw. artinsch ist.
			
			\noindent Induktionsanfang f\"ur $n = 0$: klar
			
			\noindent Induktionsschritt $n-1 \to n, (n \in \N)$: Seien $N_1, \dots, N_n$ noethersche bzw. artinsche $R$-Moduln. Dann ist $\bigoplus_{i=1}^{n-1} N_i$ noethersch bzw. artinsch nach Induktionsvoraussetzung. Wegen 
				\[\brac{\bigoplus_{i=1}^n N_i} / \brac{\bigoplus_{i=1}^{n-1} N_i} \cong N_n\]
			folgt mit \ref{1.4.5}, dass $\bigoplus_{i=1}^n N_i$ auch noethersch bzw. artinsch ist.
		\end{proof}
	\end{korollar}

	\begin{korollar}\label{1.4.7}
		Jeder endlich erzeugte Modul \"uber einem noetherschen bzw. artinschen Ring ist noethersch bzw. artinsch.
		
		\begin{proof}
			Sei $R$ ein noetherscher bzw. artinscher Ring und $M$ ein endlich erzeugter $R$-Modul. Nach \ref{1.4.6} ist ohne Einschr\"ankung $M$ zyklisch. Dann ist $M \cong R/N$ f\"ur einen $R$-Untermodul $N$ von $R$. Mit $R$ ist nach \ref{1.4.5} auch $R/N$ noethersch bzw. artinsch.
		\end{proof}
	\end{korollar}

	\begin{definition}\label{1.4.8}
		Sei $M$ ein Modul. Dann heißt 
			\[\ell(M) := \sup \set{n \in \N_0| \textrm{es gibt Untermoduln } M_0, \dots, M_n \textrm{ von } M \textrm{ mit } M_0 \supsetneq \cdots \supsetneq M_n} \in \N_0 \cup \set{\infty}\]
		
		die \emph{L\"ange}\index{Modul@\textbf{Modul}!L\"ange} von $M$.
	
		\noindent Es heißt $M$ von endlicher L\"ange, wenn $\ell(M) < \infty$.
	\end{definition}

	\begin{beispiele}\label{1.4.9}
		Sei $M$ ein Modul. Dann 
		\begin{itemize}
			\item $\ell(M) = 0 \Leftrightarrow M = 0$
			\item $\ell(M) = 1 \Leftarrow M$ ist einfach
		\end{itemize}
	\end{beispiele}

	\begin{satz}\label{1.4.10}
		Sei $N$ ein Untermodul des Moduls $M$. Dann gilt
			\[\ell(M) < \infty \Leftrightarrow \brac{\ell(M/N) < \infty \land \ell(N)< \infty}\]
		und falls $\ell(M) < \infty$
			\[\ell(M) = \ell(M/N) + \ell(N)\]
		
		\begin{proof}
			Man sieht sofort $\ell(M) = \sup \hat M, \ell(M/N) \overset{\ref{1.3.3}(a)}{=} \sup \hat{K}$ und $\ell(N) = \sup \hat N$ mit 
				\[\hat M := \set{m \in \N_0| \exists \textrm{Untermoduln } M_0, \dots, M_m \textrm{ von } M: M = M_0 \supsetneq \cdots \supsetneq M_m = 0}\]
				\[\hat K := \set{k \in \N_0| \exists \textrm{Untermoduln } L_0, \dots, L_k \textrm{ von } M: M = L_0 \supsetneq \cdots \supsetneq L_k = N}\]
				\[\hat N := \set{n \in \N_0| \exists \textrm{Untermoduln } N_0, \dots, N_n \textrm{ von } M: N = N_0 \supsetneq \cdots \supsetneq N_n = 0}\]
			
			Offensichtlich gilt $\forall k \in \hat K: \forall n \in \hat N: k + n \in \hat M$, was "$\Rightarrow$" und "$\geq$" beweist.
			
			\noindent Um "$\Leftarrow$" und "$\leq$" zu beweisen, reicht es 
				\[\forall m \in \hat M: \exists k \in \hat K: \exists n \in \hat N: m \leq k + n\]
			zu zeigen. Sei hierzu $m \in \hat M$. W\"ahle Untermoduln $M_0, \dots, M_m$ von $M$ mit $M=M_0 \supsetneq \cdots M_m = 0$.
			Setze $L_i := M_i + N$ und $N_i := M_i \cap N$ f\"ur $i \in \set{0, \dots, m}$. Nach Lemma \ref{1.4.4} ist dann jeweils mindestens eine der beiden Inklusionen $L_i \supseteq L_{i+1}$ und $N_i \supseteq N_{i+1}$ echt. (f\"ur $i \in \set{0, \dots, m-1}$). Setzt man
				\[k := |\set{i \in \set{0, \dots, m - 1}| L_i \supsetneq L_{i+1}}| \in \hat K\] und
				\[n := |\set{i \in \set{0, \dots, m - 1}| N_i \supsetneq N_{i+1}}| \in \hat N\]
			so folgt $m \leq k + n$ 
		\end{proof}
	\end{satz}

	\begin{definition}\label{1.4.11}
		Sei $M$ ein Modul. Es heißt $(M_0, \dots, M_n)$ eine \emph{Kompositionsreihe}\index{Modul@\textbf{Modul}!Kompositionsreihe} von (der \emph{L\"ange} $n$) von $M$, wenn $M_0, \dots, M_n$ Untermoduln von $M$ sind mit 
			\[M=M_0 \supsetneq \cdots \supsetneq M_n = 0\]
		derart, dass die sogenannten Faktoren $M_i/M_{i+1}$ ($i \in \set{0, \dots, n-1}$) alle einfach sind.
	\end{definition}

	\begin{bemerkung}\label{1.4.12}
		Jeder endliche Modul besitzt nat\"urlich eine Kompositionsreihe. Folgender Satz verallgemeinert dies.
	\end{bemerkung}

	\begin{satz}\label{1.4.13}
		Sei $M$ ein Modul. Es sind folgende Aussagen \"aquivalent
		
		\begin{enumerate}[(a)]
			\item $\ell(M) < \infty$
			\item $M$ ist noethersch und artinsch
			\item $M$ besitzt eine Kompositionsreihe.
		\end{enumerate}
		
		In diesem Fall ist die L\"ange einer jeden Kompositionsreihe von $M$ gleich der L\"ange von $M$.
		
		\begin{proof}\hfill
			\begin{enumerate}
				\item[$(a)\Rightarrow(b)$] trivial
				\item[$(b)\Rightarrow(c)$]
				Sei $M$ noethersch und artinsch. Da $M$ noethersch ist, gibt es zu jedem Untermodul $N \neq 0$ von $M$ einen Untermodul $N'$ von $N$ mit $N/N'$ einfach (sonst k\"onnte man eine aufsteigende Kette $0 \subsetneq N_1 \cdots \subsetneq  \cdots$ von echten Untermoduln von $M$ konstruieren).
				
				\noindent Setze nun $M_0 = N$ und w\"ahle f\"ur $i = 0, 1, \dots$ solange $M_i \neq 0$ einen Untermodul $M_{i+1}$ von $M_i$ mit $M_i/M_{i+1}$ einfach.
				
				\noindent Dieses Verfahren bricht ab, da $M$ artinsch ist.
				
				\item[$(c)\Rightarrow(a)$] und Zusatz: 
				Sei $(M_0, \dots, M_n)$ eine Kompositionsreihe von $M$. Dann $\ell(M) \overset{\ref{1.4.10}}{=} \ell(M_0/M_1) + \cdots + \ell(M_{n-1}/M_n) \overset{\ref{1.4.9}}{=} n$
			\end{enumerate}
		\end{proof}
	\end{satz}

	\begin{satz}\label{1.4.14}
		Satz von Jordan-H\"older\newline
		Sei $M$ ein Modul endlicher L\"ange $n$ und seien $M = M_0 \supsetneq \cdots \supsetneq M_n = 0$ und $M = N_0 \supsetneq \cdots \supsetneq N_n = 0$ zwei Kompositionsreihen von $M$. Dann gibt es 
		$\sigma \in S_n$ mit $M_{i-1}/M_i \cong N_{\sigma(i)-1}/N_{\sigma(i)}$ f\"ur $i \in \set{1, \dots, n}$
		
		\begin{proof}
			Induktion nach $n \in \N_0$\newline
			$n = 0$: trivial\newline
			$n-1 \rightarrow n$ ($n \in \N$): Setze $L:=N_1$ und betrachte 
				\begin{align}
					M &= L + M_0 \supseteq \cdots \supseteq L + M_n = L \label{1.4.14.1}\\
					L &= L \cap M_0 \supseteq \cdots \supseteq L \cap M_n = 0 \label{1.4.14.2}
				\end{align}
			
			\noindent Hilfsbehauptung: F\"ur alle $i \in \set{1, \dots, n}$ gilt 
			
			\textbf{entweder} $(L+M_{i+1})/(L+M_i) = 0$ und $(L\cap M_{i+1})/(L\cap M_i) \cong M_{i-1}/M_i$
			
			\textbf{oder} $(L+M_{i+1})/(L+M_i) \cong M_{i-1}/M_i$ und $(L\cap M_{i+1})/(L\cap M_i) = 0$
			
			\noindent Begr\"undung: Sei $i \in \set{1, \dots, n}$. Ist $(L\cap M_{i-1})/(L\cap M_i) \neq 0$, so ist $(L\cap M_{i-1})/(L\cap M_i) \hookrightarrow M_{i-1}/M_i$
			ein Isomorphismus, da $M_{i-1}/M_i$ einfach.
			
			\noindent Ist $(L+M_{i-1})/(L+M_i) \neq 0$, so ist $M_{i-1}/M_i \twoheadrightarrow (L+M_{i-1})/(L+M_i)$ ein Isomorphismus, da $M_{i-1}/M_i$ einfach. Daher reicht es zu zeigen, dass genau einer der Moduln $(L\cap M_{i-1})/(L\cap M_i)$ und $(L+M_{i-1})/(L+M_i)$ ein Nullmodul ist.
			
			\noindent Wegen Lemma \ref{1.4.4} k\"onnen nicht beide $0$ sein. Es reicht daher zu zeigen, dass genau $n$ der $2n$ Inklusionen (\ref{1.4.14.1}) und (\ref{1.4.14.2}) echt sind. Dies folgt aus (\ref{1.4.13}), indem man aus \ref{1.4.14.1} und \ref{1.4.14.2} eine Kompositionsreihe gewinnt.
			
			\noindent Da $M/L$ einfach ist, ist genau eine der $n$ Inklusionen in (\ref{1.4.14.1}) echt, etwa $L+M_{k-1} \supsetneq L + M_k$. Nach der Hilfsbehauptung erh\"alt man aus (\ref{1.4.14.2}) eine Kompositionsreihe von $L$ der L\"ange $n-1$ (beachte $L\cap M_{k-1} = L\cap M_k$). Da $L=N_1\supsetneq \cdots \supsetneq N_n = 0$ ebenfalls eine solche ist, gibt es nach Induktionsvoraussetzung eine Bijektion $\tau: \set{2, \dots, n} \to \set{1, \dots, n} \setminus \set{k}$ mit $N_{i-1}/N_i \cong (L\cap M_{\tau(i)-1})/(L\cap M_{\tau(i)}) \cong M_{\tau(i)-1} / M_{\tau(i)}$ f\"ur $i \in \set{1,\dots, n}\setminus \set{k}$. Zusammen mit $N_0/N_1 \cong M/L = (L+M_{k-1})/(L+M_k) \cong M_{k-1}/M_k$ liefert dies die gew\"unschte Bijektion.
		\end{proof}
	\end{satz}

	\begin{definition}\label{1.4.15}
		[$\rightarrow$\ref{1.2.4}]\newline
		Sei $R$ ein Ring, $M$ ein $R$-Modul und $E\subseteq M$. Dann nennt man den $R$-Modul $\ann(E) := \set{a \in R|\forall x \in E: ax = 0}$ von $R$ den Annihilator von $E$.
	\end{definition}

	\begin{bemerkung}\label{1.4.16}
		Sei $R$ ein kommutativer Ring, $M$ ein $R$-Modul und $E\subseteq M$. Dann ist $\ann(E)=\ann\brac{\sum_{x\in E} Rx}$. Insbesondere gilt f\"ur $M=R/aR$ mit $a \in R$, dass
			\[\ann(R/aR) = \ann(\set{\overline{1}}) = \ann(\overline{1}) = aR\]
	\end{bemerkung}

	\begin{beispiele}\label{1.4.17}
		\begin{enumerate}[(a)]
			\item Ist $V$ ein K-Vektorraum, so $\ell(V) = \dim(V)$
			\item Sei $R$ ein Hauptidealring, $n\in\N_0, p_1, \dots, p_n \in R$ irreduzibel und $m:= p_1 \cdot \cdots \cdot p_n$.
			
			Dann gilt $\ell(R/mR) = n$ und 
				\[R/mR \supsetneq p_1R/mR \supsetneq \cdots \supsetneq p_1\cdot\cdots\cdot p_nR/mR\]
			mit Faktoren $(p_1\cdot \cdot \cdot p_{i-1}R/mR)/(p_1\cdot \cdot \cdot p_iR/mR) \cong (p_1\cdot \cdot \cdot p_{i-1}R)/(p_1\cdot \cdot \cdot p_iR) \cong R/p_iR$
			f\"ur $i \in \set{1,\dots,n}$.
			
			\noindent Nach dem Satz von Jordan-H\"older gibt es f\"ur alle Kompositionsreihen $R/mR = M_0 \supsetneq \cdots \supsetneq M_n = 0$ ein $\sigma \in S_n$ mit $M_{i-1}/M_i \cong p_{\sigma(i)}R$ und daher $\ann(M_{i-1}/M_i) = \ann(R/p_{\sigma(i)}R) \overset{\ref{1.4.16}}{=}p_{\sigma(i)}R$
			
			Die Faktoren einer jeden Kompositionsreihe von $R/mR$ liefern also bis auf Reihenfolge und Assoziiertheit genau die Faktoren von $m = p_1 \cdot \cdots \cdot p_n$.
  		\end{enumerate}
	\end{beispiele}

	\newpage
	\section{Unzerlegbare Moduln}
	\begin{definition}\label{1.5.1}
		Ein Modul $M$ hei\ss t \emph{unzerlegbar}\index{Modul@\textbf{Modul}!Unzerlegbare Moduln}, falls $M \neq 0$ und f\"ur alle Untermoduln $L$ und $N$ von $M$ gilt 
			\[M = L\oplus N \Rightarrow (L = 0 \lor N = 0)\]
	\end{definition}

	\begin{bemerkung}\label{1.5.2}
		Jeder einfache Modul [$\rightarrow$\ref{1.3.2}] ist unzerlegbar, aber die Umkehrung stimmt nicht, wie \ref{1.3.4}(b) in Verbindung mit Satz \ref{1.5.4} unten zeigt.
	\end{bemerkung}

	\begin{lemma}\label{1.5.3}
		Sei $M$ ein zyklischer Modul
		\begin{enumerate}[(a)]
			\item Jeder direkte Summand von $M$ [$\rightarrow$\ref{1.3.9}] ist wieder zyklisch
			\item $M\cong R/N$ f\"ur einen $R$-Untermodul $N$ von $R$.
		\end{enumerate}
	
		\begin{proof}
			\begin{enumerate}[(a)]
				\item Seien $L$ und $N$ Untermoduln von $M$ mit $M = L \oplus N$. Schreibe $M=Rx$ mit $x \in M$ und $x = y+z$ mit $y\in L, z \in N$.
				Wir zeigen $L = Ry$. Sei hierzu $w \in L$. Zu zeigen ist, dass $w \in Ry$. Schreibe $az = ax -ay = w - ay \in L \cap N = 0$. Also $w = ax = ay \in Ry$. 
				
				\item Schreibe $M=Rx$ mit $x \in M$. W\"ahle f\"ur $N$ den Kern des $R$-Modulhomomorphismus $R\twoheadrightarrow M, a \mapsto ax$.
			\end{enumerate}
		\end{proof}
	\end{lemma}

	\begin{satz}\label{1.5.4}
		Sei $R$ ein Hauptidealring und $a \in R$. Dann ist $R/aR$ unzerlegbar genau dann, wenn es ein Primelement $p\in R$ und ein $n\in\N$ gibt mit $(a)=(p^n)$
		
		\begin{proof}
			Ohne Einschr\"ankung $a \notin R^*$. Gebe es zun\"achst keine solchen $p$ und $n$. Dann gibt es $b, c \in R \setminus R^*$ mit $a = bc$ und $(b,c) = (1)$. Nach dem Chinesischen Restsatz ist dann der kanonische $R$-Modulhomomorphismus $R/aR \to (R/bR)\times (R/cR)$ bijektiv. Daher $R/aR \cong (\underbrace{R/bR}_{\neq 0}) \oplus (\underbrace{R/cR}_{\neq 0})$
			
			\noindent Seien nun $p\in R$ prim und $n\in \N$ mit $(a) = (p^n)$.
			Gelte $R/p^nR = L \oplus M$. Zu zeigen $L=0$ oder $M = 0$. Jede Kompositionsreihe von $R/p^nR$ hat L\"ange $n$ mit allen Faktoren isomorph zu $R/pR$ nach \ref{1.4.17}(b). Alle Faktoren von Kompositionsreihen von $L$ und $M$ sind daher isomorph zu $R/pR$, denn aus je zwei Kompositionsreihen von $(L \oplus M)/M \cong L$ und $M$ kann man eine solche von $R/p^nR$ gewinnen. Nach \ref{1.5.3} gibt es aber Ideale $I$ und $J$ von $R$ mit $L \cong R/I$ und $M \cong R/J$. Da $I$ und $J$ Hauptideale sind, folgt mit \ref{1.4.17}(b) also $I=(p^l)$ und $J = (p^m)$ f\"ur gewisse $l,m \in \N_0$.
			
			Nun gilt einerseits 
			\begin{flalign*}
				n &= \ell(R/p^nR) = \ell(L \oplus M)\\ 
				&= \ell((L \oplus M)/M) + \ell(M) && \ref{1.4.10}\\
				&= \ell(L) + \ell(M) = \ell(R/p^lR) + \ell(R/p^mR) = l + m
			\end{flalign*}
			und andererseits
			\begin{flalign*}
				(p^n) &= \ann(R/p^nR) && \ref{1.4.16}\\
				&= \ann(L) \cap \ann(M) && R/p^nR = L + M\\
				&= \ann(R/p^lR) \cap \ann(R/p^mR) = (p^l) \cap (p^m)
			\end{flalign*}
			Hieraus folgt $l = 0$ oder $m = 0$. Also $L = 0$ oder $M = 0$.
		\end{proof}
	\end{satz}

	\begin{satz}\label{1.5.5}
		Jeder noethersche oder artinsche Modul ist die direkte Summe endlich vieler unzerlegbarer Untermoduln.
		
		\begin{proof}
			Sei $M$ ein noetherscher (artinscher) Modul. zu jedem direkten Summanden $N \neq 0$ von $M$ gibt es einen maximalen (minimalen) direkten Summanden $N'' \neq N$ ($N' \neq 0$) von $N$ und daher Untermoduln $N'$ und $N''$ von $N$ mit $N = N' \oplus N''$ und $N'$ unzerlegbar.
			
			Setze nun $M_0 := M$ und w\"ahle f\"ur $i=0,1,\dots$ solange $M_i\neq 0$ Untermoduln $N_{i+1}$ und $M_{i+1}$ von $M_i$ mit $M_i = N_{i+1} \oplus M_{i+1}$ und $N_{i+1}$ unzerlegbar. Dieses Verfahren bricht ab, da $N_1 \subsetneq N_1 \oplus N_2 \subsetneq \cdots$ ($M_0 \supsetneq M_1 \supsetneq \cdots$) und $M$ noethersch (artinsch) ist. Ist $M_n = 0$, so $\displaystyle M = \bigoplus_{i=1}^n N_i$
		\end{proof}
	\end{satz}

	\begin{defueb}\label{1.5.6}
		Sei $M$ ein Modul. Dann bildet 
			\[\End(M):= \set{f|f \textrm{ Endomorphismus von } M}\] 
			mit punktweiser Addition und der Hintereinanderschaltung als Multiplikation einen Ring, den sogenannten \emph{Endomorphismenring}\index{Modul@\textbf{Modul}!Endomorphismenring} von $M$.
	\end{defueb}

	\begin{lemma}\label{1.5.7}
		"Fitting-Zerlegun"\newline
		Sei $M$ ein Modul und $f \in \End(M)$ mit $\ker(f) = \ker(f^2)$ und $\im(f) = \im(f^2)$. Dann $M = \ker f \oplus \im f$
		
		\begin{proof}
			Zu zeigen
			\begin{enumerate}[(a)]
				\item $\ker f \cap \im f = 0$
				\item $M = \ker f + \im f$
			\end{enumerate}
			Zu (a): Sei $x \in \ker f \cap \im f$. W\"ahle $y \in M$ mit $x = f(y)$. Dann $f^2(y) = f(x) = 0$ und daher $y \in \ker(f^2) = \ker(f)$, d.h. $x = f(y) = 0$
			
			Zu (b): Sei $x \in M$. Wegen $f(x) \in \im f = \im(f^2)$ gibt es $y \in M$ mit $f(x) = f^2(y)$. Dann $x = \underbrace{(x - f(y))}_{\in \ker f} + \underbrace{f(y)}_{\in \im f}$
		\end{proof}
	\end{lemma}

	\begin{definition}\label{1.5.8}
		Sei $R$ ein Ring (z.B. $R = \End(M)$ f\"ur einen Modul $M$)
		
		\begin{enumerate}[(a)]
			\item Ein Element $a \in R$ hei\ss t \emph{idempotent}\index{Modul@\textbf{Modul}!idempotent} (\emph{nilpotent}\index{Modul@\textbf{Modul}!nilpotent}), wenn $a^2 = a$ ($a^n = 0$ f\"ur ein $n \in \N$)
			
			\item $R$ hei\ss t \emph{lokal}\index{Modul@\textbf{Modul}!lokal}, wenn $0\neq 1$ in $R$ und $\forall a,b \in R\setminus R^*: a + b \in R\setminus R^*$
		\end{enumerate}
	\end{definition}

	\begin{proposition}\label{1.5.9}
		Sei $M$ ein Modul. Dann ist $M$ unzerlegbar genau dann, wenn $\End(M)$ genau zwei idempotente Elemente hat (n\"amlich $0$ und $1 = \id_M \neq 0$).
		
		\begin{proof}
			\begin{enumerate}
				\item[$\Rightarrow$] Sei $M$ unzerlegbar. Wegen $M \neq 0$ gilt $0 \neq 1$ in $\End(M)$.
				
				Sei $f \in \End(M)$ idempotent. Dann $M = \ker f \oplus \im f$ nach \ref{1.5.7}. Es folgt $\ker f = 0$ oder $\im f = 0$. Im zweiten Fall ist $f = 0$. Im ersten Fall ist $f$ injektiv, also $f=1$ (da $f^2 = f$).
				
				\item[$\Leftarrow$] Seien $0\neq 1$ die einzigen idempotenten Elemente von $\End(M)$. Gelte $M = L \oplus N$. Zu zeigen $L = 0$ oder $N = 0$. 
					\[\pi_L:M=L\oplus N \to L, x + y \mapsto x\]
				($x\in L, y \in N$) ist idempotent, also $\pi_L = 0$ oder $\pi_L = 1$. Dann ist $L = 0$ oder $N = 0$.
			\end{enumerate}
		\end{proof}
	\end{proposition}

	\begin{lemma}\label{1.5.10}
		Fitting Lemma\newline
		Sei $M$ ein Modul endlicher L\"ange und $f \in \End(M)$. Dann gibt es $N \in \N$ mit $M = \ker(f^n) \oplus \im(f^n)$ f\"ur alle $n\geq N$.
		
		\begin{proof}
			Die Ketten $\ker f \subseteq \ker f^2 \subseteq \cdots$ und $\im f \supseteq \im f^2 \supseteq \cdots$ werden station\"ar. W\"ahle $N \in \N$ mit $\ker f^n = \ker f^N$ und $\im f^n = \im f^N$ f\"ur alle $n \geq N$ und nehme die Fitting-Zerlegung nach \ref{1.5.7} f\"ur $f^n$.
		\end{proof}
	\end{lemma}

	\begin{korollar}\label{1.5.11}
		Jeder Endomorphismus eines Unzerlegbaren Moduls endlicher L\"ange ist entweder nilpotent oder ein Automorphismus.
	\end{korollar}
\backmatter
\printindex
\end{document}