\documentclass[
twoside=semi,
fontsize=12,
DIV=12, 
cleardoublepage=current,
leqno,
headings=optiontoheadandtoc, 
toc=idx
]{scrbook}

\usepackage{imakeidx}
\makeindex

\usepackage[german]{babel}
\usepackage[utf8]{inputenc}
\usepackage[T1]{fontenc}
\usepackage{color}
\usepackage{datetime}
%\usepackage{ulem}
\usepackage{cancel}
\usepackage{tikz-cd}

\usepackage[headsepline]{scrlayer-scrpage}
\setkomafont{pageheadfoot}{\normalfont}
\setkomafont{pagefoot}{\slshape}

\defpagestyle{main}
{
	{\thesection\ \leftmark \hfill}{\hfill \thesection\ \leftmark}{}
}
{
	{\pagemark\hfill}{\hfill \pagemark}{}
}

\defpagestyle{sectionstart}
{
	{}{}{}(0pt,0pt)
}
{
	{\pagemark\hfill}{\hfill \pagemark}{}
}

\renewcommand{\sectionmark}[1]{\markboth{#1}{}}

\usepackage{chngcntr}\counterwithout{equation}{section}

\usepackage{amsmath}
\usepackage{amssymb}
\usepackage{amsthm}
\usepackage{stmaryrd}
\usepackage{enumerate}
\usepackage[pass]{geometry}
\usepackage{csquotes}
\usepackage{mathtools}
\usepackage{nicematrix}

\usepackage{hyperref}
\MakeOuterQuote{"}

\newcommand{\N}{\mathbb{N}}
\newcommand{\Z}{\mathbb{Z}}
\newcommand{\Q}{\mathbb{Q}}
\newcommand{\R}{\mathbb{R}}
\newcommand{\F}{\mathbb{F}}
\renewcommand{\P}{\mathbb{P}}
\newcommand{\comment}[1]{}
\newcommand{\nsquare}{\cancel{\square}}


\newcommand{\brac}[1]{\left( #1 \right)}
\newcommand{\bracB}[1]{\left[ #1 \right]}
\newcommand{\bracC}[1]{\left< #1 \right>}
\newcommand{\abs}[1]{\left| #1 \right|}
\newcommand{\set}[1]{\left\{ #1 \right\}}

\newcommand{\textcase}[2]{$\begin{Bmatrix} \textrm{#1} \\ \textrm{#2}\end{Bmatrix}$}
\newcommand{\case}[2]{\begin{Bmatrix} #1 \\ #2\end{Bmatrix}}
\newcommand{\schemaSmith}[3]{\begin{NiceTabular}[corners={NE,SE}, hvlines]{cc} #1 & #2 \\ #3 \end{NiceTabular}}
\newcommand{\schemaSmithH}[2]{\begin{NiceTabular}[corners={NE,SE}, hvlines]{cc} #1 & #2 \end{NiceTabular}}
\newcommand{\schemaSmithV}[2]{\begin{NiceTabular}[corners={NE,SE}, hvlines]{c} #1 \\ #2 \end{NiceTabular}}

\newcommand*{\ORIGchapterheadstartvskip}{}
\let\ORIGchapterheadstartvskip=\chapterheadstartvskip
\renewcommand*{\chapterheadstartvskip}{
	\ORIGchapterheadstartvskip
	\noindent\rule[\baselineskip]{\linewidth}{4pt}\par
}
\newcommand*{\ORIGchapterheadendvskip}{}
\let\ORIGchapterheadendvskip=\chapterheadendvskip
\renewcommand*{\chapterheadendvskip}{
	\ORIGchapterheadendvskip
	\noindent\rule[\baselineskip]{\linewidth}{4pt}\par
}

\DeclareMathOperator{\im}{im}
\DeclareMathOperator{\supp}{supp}
\DeclareMathOperator{\ann}{ann}
\DeclareMathOperator{\id}{id}
\DeclareMathOperator{\rk}{rk}
\DeclareMathOperator{\End}{End}
\DeclareMathOperator{\Aut}{Aut}
\DeclareMathOperator{\Hom}{Hom}
\DeclareMathOperator{\GL}{GL}
\DeclareMathOperator{\com}{com}
\DeclareMathOperator{\qf}{qf}
\DeclareMathOperator{\irr}{irr}
\DeclareMathOperator{\tr}{tr}
\DeclareMathOperator{\Char}{char}


\swapnumbers
\theoremstyle{definition}
\newtheorem{definition}{Definition}[section]
\newtheorem{bemerkung}[definition]{Bemerkung}
\newtheorem{beispiel}[definition]{Beispiel}
\newtheorem{warnung}[definition]{Warnung}
\newtheorem{satz}[definition]{Satz}
\newtheorem{lemma}[definition]{Lemma}
\newtheorem{proposition}[definition]{Proposition}
\newtheorem{notation}[definition]{Notation}
\newtheorem{korollar}[definition]{Korollar}
\newtheorem{erinnerung}[definition]{Erinnerung}
\newtheorem{prop-def}[definition]{Proposition und Definition}
\newtheorem{def-prop}[definition]{Definition und Proposition}
\newtheorem{def-satz}[definition]{Definition und Satz}
\newtheorem{def-ueb}[definition]{Definition und \"Ubung}
\newtheorem{satz-not}[definition]{Satz und Notation}
\newtheorem{satz-def}[definition]{Satz und Definition}
\newtheorem{uebung}[definition]{\"Ubung}
\newtheorem{motivation}[definition]{Motivation}
\newtheorem{bem-not}[definition]{Bemerkung und Notation}
\newtheorem{def-prop-satz-not}[definition]{Definitionen, Propositionen, S\"atze und Notationen}
\newtheorem{sprechweise}[definition]{Sprechweise}
\newtheorem{defueb}[definition]{Definition und \"Ubung}

\begin{document}
	\tableofcontents\thispagestyle{empty}
	\newpage\thispagestyle{empty}
	\mainmatter
	\chapter[tocentry={Moduln}]{Moduln}
	\pagestyle{main}
	
	\section{Definitionen und grundlegende Tatsachen}
	\begin{definition}\label{1.1.1}\hfill\newline
		Ein \emph{Modul}\index{Modul@\textbf{Modul}} ist ein Tupel $(R, +_R, \cdot_R, M, +, \cdot)$, wobei $(R, +_R, \cdot_R)$ ein Ring (mit $1$, nicht notwendigerweise kommutativ), $(M, +)$ eine abelsche Gruppe und \\\noindent$\cdot:R\times M \to M$ eine (meist gar nicht oder infix geschriebene) Abbildung mit folgenden Eigenschaften
		
		\begin{itemize}
			\item[$(\overset{\rightarrow}{D})$] $\forall a \in R: \forall x, y \in M: a(x + y) = ax + ay$ \hfill "distributiv"
			
			\item[$(D')$] $\forall a, b \in R: \forall x \in M: (a+b)x = ax + bx$ \hfill "distributiv"
			
			\item[$(N)$] $\forall x \in M: 1_R \cdot x = x$ \hfill "normiert"
			
			\item[$(V)$] $\forall a, b \in R: \forall x \in M: (ab)x = a(bx)$ \hfill "vertr\"aglich"
		\end{itemize}
	\end{definition}
	
	
	\begin{bemerkung}\label{1.1.2}\hfill
			\begin{enumerate}[(a)]
			\item Schlampiger Sprachgebrauch: 
			\begin{itemize}
				\item "Sei $M$ ein $R$-Modul" statt "Sei $(R, +_R, \cdot_R, M, +, \cdot)$ ein Modul"
				
				\item "Sei $M$ ein Modul" statt "Es gebe einen Ring $R$ so, dass $M$ ein $R$-Modul ist"
			\end{itemize}
			
			\item Statt "$R$-Modul" sagt man auch "Modul \"uber $R$"	
			
			\item Vektorr\"aume sind Moduln \"uber K\"orper. Viele Sprechweisen (wie "Skalar", "Linearkombination", nicht jedoch "Vektor") \"ubertragen wir stillschweigend von Vektorr\"aumen auf Moduln, ebenso 
			Konventionen (wie "Punkt vor Strich").
			
			\item Abelsche Gruppen "sind" $\Z$-Moduln. Sei $G$ eine abelsche Gruppe. Dann gibt es genau eine Skalarmultiplikation $\cdot:\Z\times G \to G$ verm\"oge derer $G$ zu einem $\Z$-Modul wird, n\"amlich die nat\"urliche, die durch 
			\[n \cdot a := \begin{cases}
				\underbrace{a + a + \cdots + a}_{n\textrm{-mal}} & \textrm{falls } n > 0\\
				0 & \textrm{falls } n = 0\\
				\underbrace{-a - a - \cdots - a}_{(-n)\textrm{-mal}} & \textrm{falls } n < 0
			\end{cases}\]
			gegeben ist.
			
			\item $(\overset{\rightarrow}{D})$ besagt, dass f\"ur alle $a \in R$ die Abbildung $M \to M, x \mapsto ax$ ein Gruppenhomomorphismus ist. Insbesondere gilt $a \cdot 0 = 0$ und $a \cdot (-x) = -ax$ f\"ur alle $a \in R, x \in M$.
			
			$(D')$ besagt, dass f\"ur alle $x \in M$ die Abbildung $R \to M, a \mapsto ax$ ein Gruppenhomomorphismus ist. Insbesondere gilt $0 \cdot x = 0$ und $(-a) \cdot x = -ax$ f\"ur alle $a \in R, x \in M$.
		\end{enumerate}
	\end{bemerkung}
	
	\begin{beispiel}\label{1.1.3}\hfill
			\begin{enumerate}[(a)]
			\item Nullmoduln $\set{0}$
			
			\item Sei $A$ ein Unterring des Ringes $B$. Dann ist $B$ ein $A$-Modul verm\"oge der Skalarmultiplikation $\cdot: A \times B \to B, (a, x) \mapsto ax$
			
			Insbesondere ist jeder Ring ein Modul \"uber sich selbst.
			
			\item Sei $R$ ein kommutativer Ring und $n \in \N_0$. Dann wird die abelsche Gruppe $R^n$ zu einem $R^{n\times n}$-Modul verm\"oge der Skalarmultiplikation
			\[\cdot: R^{n\times n} \times R^n \to R^n, (A, x) \mapsto Ax\]
			Dies folgt aus den Rechenregeln f\"ur Matrixmultiplikation.
		\end{enumerate}
	\end{beispiel}
	
	\begin{def-prop-satz-not}\label{1.1.4}\hfill\newline
		Sei $R$ ein Ring. Die folgenden f\"ur die Theorie der $R$-Moduln grundlegenden Begriffe und Resultate sind eine direkte Verallgemeinerung der entsprechenden Tatsachen f\"ur Vektorr\"aume (also f\"ur den Fall, dass $R$ ein K\"orper) und f\"ur abelsche Gruppen (also $R = \Z$) aus der Linearen Algebra:
		
		\begin{enumerate}[(a)]
			\item Genauso wie bei Vektorr\"aumen f\"uhrt man \emph{direkte Produkte}\index{Modul@\textbf{Modul}!Direktes Produkt} von $R$-Moduln ein.
			
			\item Sind $M$ und $N$ $R$-Moduln, so hei\ss t $N$ ein \emph{Untermodul}\index{Modul@\textbf{Modul}!Untermodul} von $M$, wenn die $N$ zugrunde liegende abelsche Gruppe eine Untergruppe der $M$ zugrunde liegenden abelschen Gruppe ist und 
			\[\forall a \in R: \forall x \in M: a \cdot_N x = a \cdot_M x\]
			
			Ein Untermodul eines Moduls ist offenbar durch seine Tr\"agermenge (d.h. seine zugrunde liegende Menge) eindeutig bestimmt.
			
			Ist $M$ ein $R$-Modul und $N \subseteq M$, so ist $N$ offenbar genau dann (Tr\"agermenge) ein(es) Untermodul(s) von $M$, wenn 
			\begin{itemize}
				\item $0\in N$
				\item $\forall x, y \in N: x + y \in N$
				\item $\forall a \in R: \forall x \in N: ax \in N$
			\end{itemize}
			
			\item Sei $M$ ein Modul und $(N_i)_{i \in I}$ eine Familie von Untermoduln von $M$. Dann ist 
			$\displaystyle \bigcap_{i \in I} N_i := \bigcap \set{N_i \mid i \in I}$ (mit $\displaystyle \bigcap_{i \in I} N_i = M$, falls $I = \emptyset$)
			wieder ein Untermodul von $M$ und zwar der gr\"o\ss te Untermodul von $M$, der in allen $N_i$ enthalten ist.
			
			Weiter ist auch $\displaystyle \sum_{i \in I} N_i := \set{\sum_{i \in I} x_i \mid (x_i)_{i \in I} \in \prod_{i\in I} N_i, \set{i \in I \mid x_i \neq 0} \textrm{endlich}}$ Untermodul von $M$ und zwar der kleinste Untermodul von $M$, der alle $N_i$ enth\"alt.
			
			\item Sei $M$ ein $R$-Modul. Ist $x \in M$, so ist $Rx := \set{ax \mid a \in R}$ ein Untermodul von $M$ und zwar der kleinste Untermodul, der $x$ enth\"alt.
			
			Ist $(x_i)_{i \in I}$ eine Familie von Elementen von $M$, so ist $\sum_{i \in I} Rx_i$ der kleinste Untermodul von $M$, der alle $x_i$ enth\"alt.
			
			Man nennt ihn den von den $x_i$ ($i \in I$) (oder $\set{x_i \mid i \in I}$) erzeugten Untermodul von $M$ (oder lineare H\"ulle der Span von $\set{x_i \mid i \in I}$). 
			
			Man nennt $M$ \emph{zyklisch}\index{Modul@\textbf{Modul}!Zyklische Moduln}, wenn $M$ von einem Element erzeugt wird, d.h. es ein $x \in M$ gibt mit $M = Rx$. Man nennt $M$ endlich erzeugt (e.e.), wenn $M$ von endlich vielen Elementen
			erzeugt wird, d.h. es ein $n \in \N_0$ und $x_1, \dots, x_n \in M$ gibt mit 
			\[M = Rx_1 + \cdots + Rx_n := \sum_{i=1}^{n} Rx_i := \sum_{i \in \set{1, \dots, n}} Rx_i\] 
			
			\item Sei $M$ ein $R$-Modul. Eine Familie $(x_i)_{i\in I}$ in $M$ hei\ss t \emph{linear unabh\"angig}\index{Modul@\textbf{Modul}!Linear unabh\"angig (l.u.)} (l.u.), wenn f\"ur alle $n\in \N_0$, alle paarweise verschiedenen $i_1, \dots, i_n \in I$ und alle $a_1, \dots, a_n \in I$ gilt 
			\[\sum_{j=1}^{n} a_jx_{i_j} = 0 \Rightarrow a_1 = \cdots = a_n = 0\]
			
			Weiter nennt man $x_1, \dots, x_n \in M$ linear unabh\"angig, wenn $(x_1, \dots, x_n) = (x_i)_{i \in \set{1, \dots, n}}$ linear unabh\"angig ist, d.h. f\"ur alle $a_1, \dots, a_n \in R$ gilt 
			\begin{align}
				a_1x_1 + \cdots a_nx_n = 0 \Rightarrow a_1 = \cdots = a_n = 0 \label{lin_ind}\tag{$*$}
			\end{align}
			
			Schlie\ss lich hei\ss t eine Menge $F\subseteq M$ linear unabh\"angig, wenn $(x)_{x \in F}$ linear unabh\"angig ist, d.h. f\"ur alle $n\in \N_0$, alle paarweise verschiedenen $x_1, \dots, x_n \in F$ und alle $a_1, \dots, a_n \in R$ wieder (\ref{lin_ind}) gilt.
			
			\item Sei $M$ ein Modul. Eine Familie $(x_i)_{i \in I}$ in $M$ hei\ss t eine \emph{Basis}\index{Modul@\textbf{Modul}!Basis} von $M$, wenn sie $M$ erzeugt und linear unabh\"angig ist. Weiter
			sagt man $x_1, \dots, x_n \in M$ bilden eine Basis von $M$, wenn $(x_1, \dots, x_n) = (x_i)_{i \in \set{1, \dots, n}}$ eine Basis von $M$ ist. Schlie\ss lich hei\ss t $B \subseteq M$ eine Basis, wenn $B$ den Modul $M$ erzeugt und linear unabh\"angig ist.
			
			\item Seien $M$ und $N$ $R$-Moduln. Dann hei\ss t $f$ ein \emph{($R$-)(Modul-)Homomorphismus}\index{Modul@\textbf{Modul}!Homomorphismus} oder eine $(R-)$ lineare Abbildung von $M$ nach $N$, wenn $f:M\to N$ ein Gruppenhomomorphismus der $M$ und $N$ zugrundeliegenden abelschen Gruppen ist und 
			\[\forall a \in R: \forall x \in M: f(ax) = af(x) \]
			
			Ein Modulhomomorphismus $f:M\to N$ hei\ss t Einbettung/Monomorphismus (Epimorphismus, Isomorphismus), wenn $f$ injektiv (surjektiv, bijektiv) ist. 
			
			Ein Modulhomomorphismus $f:M \to M$ hei\ss t \emph{(Modul-)Endomorphismus}\index{Modul@\textbf{Modul}!Homomorphismus!Endomorphismus} von $M$. Ein Endomorphismus, der ein Isomorphismus ist, hei\ss t \emph{Automorphismus}\index{Modul@\textbf{Modul}!Automorphismus}. Es hei\ss en $M$ und $N$ \emph{isomorph}, in Zeichen $M \cong N$, wenn es einen Isomorphismus $M \to N$ gibt.
			
			Hintereinanderschaltungen von Modulhomomorphismen sind wieder Modulhomomorphismen. Umkehrabbildungen von Modulisomorphismen sind wieder Modulisomorphismen.
			
			\item Sei $M$ ein $R$-Modul. Eine \emph{Kongruenzrelation}\index{Modul@\textbf{Modul}!Kongruenzrelation} auf $M$ ist eine \"Aquivalenzrelation $\equiv$ der $M$ zugrundeliegenden Menge, f\"ur die gilt
			\[\forall x, y, x', y' \in M: (x \equiv x' \land y \equiv y') \Rightarrow x + y \equiv x' + y'\]
			und
			\[\forall x, x' \in M: \forall a \in R: x \equiv x' \Rightarrow ax \equiv ax'\]
			
			Diese Definition wurde gerade so gemacht, dass 
			\begin{align*}
				+:( M/\equiv) \times (M/\equiv) &\to (M/\equiv)\\
				 (\overline{x}, \overline{y}) &\mapsto \overline{x + y}
			\end{align*}

			und	

			\begin{align*}
				\cdot: R \times (M/\equiv) &\to (M/\equiv)\\
				(a, \overline{x}) &\mapsto \overline{ax}
			\end{align*}
		
			wohldefiniert sind.
			
			Ist $M$ ein $R$-Modul und $\equiv$ eine Kongruenzrelation auf $M$, so wird die Quotientenmenge $M/\equiv$ verm\"oge der Addition $+$ und der Skalarmultiplikation $\cdot$ ein $R$-Modul, wie man
			durch direktes Nachrechnen sieht. Die Zuordnungen
			\begin{align*}
				\equiv  &\overset{f}{\mapsto} \overline{0}\\
				\equiv_N &\overset{g}{\mapsfrom} N
			\end{align*}
			vermitteln eine Bijektion zwischen der Menge der Kongruenzrelationen auf $M$ und der Menge der Untermoduln von $M$, wobei $\equiv_N$ gegeben ist durch
			\[a \equiv_N b :\Leftrightarrow a - b \in N\]
			f\"ur $a, b \in M$.
			
			Ist $N$ ein Untermodul von $M$, so nennt man $M/N := M/\equiv_N$ auch den \emph{Quotientenmodul}\index{Modul@\textbf{Modul}!Quotientenmodul} von $M$ nach $N$.
			
			\item Sind $M$ und $N$ $R$-Moduln und $f:M \to N$ ein Modulhomomorphismus, so ist der \emph{Kern}\index{Modul@\textbf{Modul}!Homomorphismus!Kern} $\ker f := \set{x \in M \mid f(x) = 0}$ von $f$ ein Untermodul von $M$ und das \emph{Bild}\index{Modul@\textbf{Modul}!Homomorphismus!Bild} $\im f := \set{f(x) \mid x \in M}$ von $f$ ist ein Untermodul von $N$.
			
			\item Homomorphiesatz: Seien $M$ und $N$ $R$-Moduln und $L$ ein Untermodul von $M$ und $f:M \to N$ ein Modulhomomorphismus mit $L \subseteq \ker f$. Dann gibt es (genau) einen Modulhomomorphismus $\overline{f}: (M/L) \to N$ mit $\overline{f}(\overline{x}) = f(x)$ f\"ur alle $x \in M$.
			
			Ferner gilt, dass
			\begin{itemize}
				\item $\overline{f}$ ist injektiv $\Leftrightarrow L = \ker f$ und
				\item $\overline{f}$ ist surjektiv $\Leftrightarrow f$ ist surjektiv
			\end{itemize}
			
			\item Isomorphiesatz: Seien $M$ und $N$ $R$-Moduln und $f:M \to N$ ein Modulhomomorphismus. Dann ist $\overline{f}:(M/\ker f) \to \im f$ definiert durch $\overline{f}(\overline{x}) = f(x)$ f\"ur alle $x\in M$ ein $R$-Modulisomorphismus. Insbesondere ist $M/\ker f \cong \im f$
		\end{enumerate}
	\end{def-prop-satz-not}
	
	\begin{bemerkung}\label{1.1.5}\hfill\newline
		Sei $R$ ein kommutativer Ring. Dann sind die Untermoduln des $R$-Modul $R$ [$\to$\ref{1.1.3}(b)] (oder kurz gesagt die $R$-Untermoduln von $R$) genau die Ideale des Ringes $R$.
		Insbesondere sind zum Beispiel das von einem $a \in R$ erzeugte Ideal und der davon erzeugte Untermodul als Menge dasselbe $(a)_R = Ra \overset{R\ \textrm{komm.}}{=} \set{ab \mid b \in R} = aR$.
		Trotzdem macht es vom Sinn her einen Unterschied. ob man $(a)$ oder $Ra$ schreibt. Zum Beispiel meint man mit $R/(a)$ den Ring und mit $R/aR$ den $R$-Modul (deren zugrundeliegenden abelschen Gruppen dieselben sind)
	\end{bemerkung}

	\begin{warnung}\label{1.1.6}\hfill\newline
		F\"ur den mit Vektorr\"aumen, aber nicht mit Moduln vertrauten H\"orern ist Vorsicht geboten:
		\begin{enumerate}[(a)]
			\item In einem $R$-Modul $M$ kann $ax = 0$ f\"ur ein $a \in R$ und ein $x \in M$ gelten, ohne dass $a = 0$ oder $x = 0$ gilt (zum Beispiel $2 \cdot \overline{1} = \overline{2} = 0$ im $\Z$-Modul $\Z/2\Z$)
			
			\item Nicht jeder Modul hat eine Basis:
			zum Beispiel ist jedes Element des $\Z$-Moduls $\Z/2\Z$ linear abh\"angig, denn $1 \cdot \overline{0} = \overline{0} = 0$ und $2 \cdot \overline{1} = \overline{2} = 0$ in $\Z/2\Z$, womit die einzige linear unabh\"angige Teilmenge von $\Z/2\Z$ die leere Menge is, welche aber $\Z/2\Z$ nicht erzeugt.
		\end{enumerate}
	\end{warnung}	

	\begin{beispiel}\label{1.1.7}\hfill
		\begin{enumerate}[(a)]
			\item F\"ur jeden Ring $R$ ist $R^n$ ein $R$-Modul mit der \emph{Standardbasis}\index{Modul@\textbf{Modul}!Standardbasis} $\underline{e} = (e_1, \dots, e_n)$, wobei $e_i~:=~\begin{pmatrix}
				0\\
				\vdots\\
				0\\
				1\\
				0\\
				\vdots\\
				0
			\end{pmatrix}$ mit einer $1$ an der $i$-ten Stelle.
		
			\item $\R^2$ ist ein zyklischer $\R^{2 \times 2}$-Modul [$\to$\ref{1.1.3}(c)], welcher von jedem $x \in \R^2 \setminus \set{0}$ erzeugt ist. Da aber jedes $x \in \R^2$ linear abh\"angig ist, hat dieser Modul keine Basis.  
		\end{enumerate}
	\end{beispiel}
	
	
	\newpage

	\section{Direkte Summen von Moduln und freie Moduln}\thispagestyle{sectionstart}
	\begin{definition}\label{1.2.1}\hfill\newline
		Sei $R$ ein Ring und $(M_i)_{i \in I}$ eine Familie von $R$-Moduln. Dann nennt man den \linebreak$R$-Untermodul 
			\[\bigoplus_{i \in I} M_i := \set{x \in \prod_{i\in I}M_i \mid \supp(x)\ \mathrm{endlich} } \]
		von $\displaystyle\prod_{i\in I} M_i$ die \emph{(\"au\ss ere) direkte Summe}\index{Modul@\textbf{Modul}!\"Au\ss ere Direkte Summe} der $M_i$ ($i \in I$). Man fasst $M_j$ ($j \in I$ h\"aufig) als Untermodul von $\displaystyle\bigoplus_{i \in I} M_i$ auf verm\"oge der Einbettung 
		 \[\rho_j:M_j \to \prod_{i\in I} M_i, x \mapsto \brac{i \mapsto \begin{cases}
		 		x & \textrm{falls } i = j\\
		 		0 & \textrm{sonst}
		 \end{cases}}\] 
	 	
	 	\noindent Ist $M_i = M$ f\"ur alle $i \in I$, so schreibt man \[M^{(I)} := \bigoplus_{i \in I} M \subseteq \prod_{i\in I} M = M^I\]
	\end{definition}
	
	\begin{proposition}\label{1.2.2}\hfill\newline
		Sei $R$ ein Ring, $(M_i)_{i \in I}$ eine Familie von $R$-Moduln, $N$ ein $R$-Modul und $(f_i)_{i \in I}$ eine Familie von Modulhomomorphismen $f_i: M_i \to N$. Dann gibt es genau einen Modulhomomorphismus $\displaystyle f:\bigoplus_{i \in I} M_i \to N$ mit $f\big|_{M_i} = f_i$ f\"ur alle $i \in I$ ($f \circ \rho_i = f_i$ f\"ur $i \in I$).
	\end{proposition}

	\begin{proof}
		F\"ur jedes $\displaystyle x \in \bigoplus_{i \in I} M_i$ gilt $\displaystyle x = \sum_{i \in \supp(x)} \rho_i (x(i))$. Um $f \circ \rho_i = f_i$ f\"ur $i \in I$ zu erf\"ullen, kann man daher nur 
		\[f: \bigoplus_{i \in I} M_i \to N, x \mapsto \sum_{i \in I} f_i(x(i))\] 
		definieren. Man \"uberpr\"uft sofort, dass das so definierte $f$ ein Homomorphismus ist.
	\end{proof}
	
	\begin{prop-def}\label{1.2.3}\hfill\newline
		Sei $R$ ein Ring, $M$ ein $R$-Modul und $(N_i)_{i \in I}$ eine Familie von Untermoduln on $M$. Dann sind die folgenden Bedingungen \"aquivalent
		\begin{enumerate}[(a)]
			\item Die Abbildung von der \"au\ss eren direkten Summe $\bigoplus_{i \in I} N_i$ nach $M$, die auf $N_i$ die Identit\"at ist, ist ein Isomorphismus
			\item $\displaystyle M = \sum_{i \in I} N_i$ und f\"ur alle $n \in \N$, paarweise verschiedenen $i_1, \dots, i_n \in I$ und alle $x_1 \in N_{i_1}, \dots, x_n \in N_{i_n}$ gilt 
				\[(x_1 + \cdots + x_n = 0) \Rightarrow (x_1 = \cdots = x_n = 0)\]  
		\end{enumerate}
	Gelten diese Bedingungen, so nennt man $M$ die \emph{(innere) direkte Summe}\index{Modul@\textbf{Modul}!Innere Direkte Summe} der $N_i$ ($i \in I$) und schreibt (angesichts der Isomorphismus aus $(a)$) wieder $\displaystyle M = \bigoplus_{i \in I} N_i$
	\end{prop-def}

	\begin{definition}\label{1.2.4}\hfill\newline
		Sei $R$ ein Ring, $M$ ein $R$-Modul und $x \in M$. Der Kern des $R$-Modulhomomorphismus $R \to M, a \mapsto ax$ nennt man \emph{Annihilator}\index{Modul@\textbf{Modul}!Annihilator} von $x$, in Zeichen $\ann(x) = \set{a \in R \mid ax = 0}$.\newline
		Es hei\ss t $x$ ein \emph{Torsionselement}\index{Modul@\textbf{Modul}!Torsionselement} von $M$ wenn $\ann(x) \neq \set{0}$.
	\end{definition}

	\begin{satz}\label{1.2.5}\hfill\newline
		Sei $R$ ein Ring, $M$ ein $R$-Modul und $B \subseteq M$. Dann sind \"aquivalent
		\begin{enumerate}[(a)]
			\item $B$ ist eine Basis von $M$
			\item $\displaystyle M = \bigoplus_{x \in B} Rx$ und $B$ enth\"alt kein Torsionselement
			\item F\"ur jeden $R$-Modul $N$ und jede Abbildung $g:B \to N$ gibt e genau einen Homomorphismus $f:M \to N$ mit $f\big|_B = g$.
		\end{enumerate}
	
		\begin{proof}\hfill\newline
			\textbf{(a)$\implies$(b)} klar
			
			\medskip\noindent
			\textbf{(b)$\implies$(c)} Gelte (b). Sei $N$ ein $R$-Modul und $g:B \to N$ eine Abbildung. Zu zeigen sind Existenz und Eindeutigkeit eines Homomorphismus $f: M \to N$ mit $f\big|_B = g$

			\noindent \emph{Eindeutigkeit}: klar aus $\displaystyle M = \sum_{x \in B} Rx$
					
			\noindent \emph{Existenz}: Fixiere zun\"achst $x \in B$. Dann ist $R \to Rx, a \mapsto ax$ ein Isomorphismus (mit Kern $\ann(x) = \set{0}$), dessen Umkehrfunktion ein Isomorphismus $Rx \to R$ ist, der $x$ auf $1$ abbildet. Schaltet man den Homomorphismus $R \to N, a \mapsto ag(x)$ dahinter, so erh\"alt man einen Homomorphismus $Rx \to N$, der $x$ auf $g(x)$ abbildet. Da $x \in B$ beliebig war, erh\"alt man mit \ref{1.2.2} einen Homomorphismus $\displaystyle f:~M~=~\bigoplus_{x \in B}~Rx~\to~N$, der jedes $x \in B$ auf $g(x)$ abbildet.  
			
			\medskip\noindent
			\textbf{(c)$\implies$(a)}
			Gelte (c). Zu zeigen ist, dass $B$ linear unabh\"angig ist und $M$ erzeugt.
				
			\noindent $B$ \emph{linear unabh\"angig}: Seien $x_1, \dots, x_n \in B$ paarweise verschieden und $a_1, \dots, a_n \in R$ mit $a_1x_1 + \cdots + a_nx_n = 0$. Sei $i \in \set{1, \dots, n}$. Zu zeigen ist $a_i = 0$. Gem\"a\ss\ (c) gibt es einen Homomorphismus $f: M \to R$ mit $f(x_i) = 1$ und $f(x_j) = 0$ f\"ur $j \in \set{1, \dots, n} \setminus \set{i}$.
			Dann \[0 = f(0) = f\brac{\sum_{j=1}^n a_jx_j} = \sum_{j=1}^n a_jf(x_j) = a_if(x_i) = a_i\]
				
			\noindent $B$ \emph{erzeugt} $M$: Nach (c) gibt es einen Homomorphismus $M \to M$, der auf $B$ die Identit\"at ist. Einerseits ist $\id_M$ ein solcher, andererseits auch $\rho \circ f$, wobei $\displaystyle f: M \to N:= \sum_{x \in B} Rx$ der nach (c) existierende Homomorphismus mit $f\big|_{B} = \id_B$ ist und $\iota: N \hookrightarrow M, x \mapsto x$ die Inklusion. Also $\id_M = \iota \circ f$, insbesondere $M = \im(\id_M) = \im(f) = N$
		\end{proof}
	\end{satz}

	\begin{definition}\label{1.2.6}\hfill\newline
		Ein Modul hei\ss t \emph{frei}\index{Modul@\textbf{Modul}!Freie Moduln}, wenn er eine Basis besitzt.
	\end{definition}

	\begin{bemerkung}\label{1.2.7}\hfill\newline
		Sei $R$ ein Ring, $M$ ein $R$-Modul, $n \in \N_0$ und $x_1, \dots, x_n \in M$. Dann bilden $x_1, \dots, x_n$ genau dann eine Basis von $M$, wenn der Homomorphismus 
		\[R^n \to M, \begin{pmatrix}
			a_1\\\vdots\\a_n
		\end{pmatrix} \mapsto \sum_{i=1}^n a_ix_i\] ein Isomorphismus ist.
	\end{bemerkung}
	
	\begin{bemerkung}\label{1.2.8}\hfill\newline
		Ist $M$ ein $\set{0}$-Modul, so ist $M = \set{0}$, denn ist $x \in M$, so ist $x=1 \cdot x = 0 \cdot x = 0$
	\end{bemerkung}

	\begin{lemma}\label{1.2.9}\hfill\newline
		Ein endlich erzeugter Modul hat niemals eine unendliche Basis.
		
		\begin{proof}
			Sei $M$ ein endlich erzeugter $R$-Modul, etwa $\displaystyle M = \sum_{i=1}^n Rx_i$ mit $x_1, \dots, x_n \in M$. \newline
			Annahme: $B$ ist eine unendliche Basis von $M$. Dann gibt es f\"ur jedes $i \in \set{1, \dots, n}$ ein endliches $B_i \subseteq B$ mit $\displaystyle x_i \in \sum_{y \in B_i}Ry$. Dann ist $B' := B_1 \cup \cdots \cup B_n \subseteq B$ endlich mit $\displaystyle M = \sum_{y \in B'}Ry$. Da $B$ unendlich h ist, gibt es ein $z \in B\setminus B'$
			
			Nun gilt $\displaystyle z \in \sum_{y \in B'}Ry$, was im Widerspruch zur linearen Unabh\"angigkeit von $B$ steht, au\ss er wenn $1 = 0$ in $R$, d.h. $R = \set{0}$. Im letzten Fall ist aber nach \ref{1.2.8} nichts zu zeigen.
		\end{proof}
	\end{lemma}

	\begin{bemerkung}\label{1.2.10}\hfill
		\begin{enumerate}[(a)]
			\item 
			Jeder Modul \"uber dem Nullring hat genau zwei Basen, n\"amlich $\emptyset$ und $\set{0}$. In der Tat: Nach \ref{1.2.8} handelt es sich um den Nullmodul und in einem $\set{0}$-Modul ist $0$ linear unabh\"angig.
			\item In den \"Ubungen geben wir einen Ring $R \neq \set{0}$, der als $R$-Modul zu $R^2$ isomorph ist. Durch Induktion schlie\ss t man, dass $R \cong R^n$ f\"ur alle $n \in \N$. Damit besitzt $R$ als $R$-Modul f\"ur jedes $n \in \N$ eine $n$-elementige Basis, aber nach \ref{1.2.9} keine unendliche Basis.
		\end{enumerate}
	\end{bemerkung}

	\begin{satz}\label{1.2.11}\hfill\newline
		Sei $R$ ein kommutativer Ring mit $1 \neq 0$. Dann sind je zwei Basen eines $R$-Moduls entweder beide unendlich oder beide endlich mit der selben Anzahl von Elementen
		
		\begin{proof}
			Sei $M$ ein $R$-Modul mit Basen $B$ und $C$. Im Fall von $|B| =  \infty = |C|$ sind wir fertig, sonst ist $M$ endlich erzeugt und daher $m = |B|, n = |C| \in \N_0$ nach Lemma \ref{1.2.9}. Nach \ref{1.2.7} gilt $R^n \cong M \cong R^m$, somit reicht es zu zeigen: Sei $R$ ein kommutativer Ring und $m, n \in \N_0, m > n$ mit $R^m \cong R^n$ als $R$-Modul, dann gilt $1 = 0$ in $R$. 
			
			\noindent Um dies zu zeigen, w\"ahle zueinander inverse $R$-Modulisomorphismen $f:R^n \to R^m$, \linebreak $g: R^m \to R^n$. Bezeichne mit $\underline{x} = (x_1, \dots, x_n)$ und $\underline{y} = (y_1, \dotsm y_m)$ die Standardbasen des $R^n$ und $R^m$. W\"ahle $A=(a_{ij})_{1\leq i \leq m, 1 \leq j \leq n}\in R^{m \times n}$ mit $\displaystyle f(x_j) = \sum_{i=1}^m a_{ij}y_i$ f\"ur $j \in \set{1, \dots, n}$ und $B = (b_{ji})_{1 \leq j \leq n, 1 \leq i \leq m} \in R^{n \times m}$ mit $\displaystyle g(y_i) = \sum_{j=1}^n b_{ji}x_j$ f\"ur $i \in \set{1, \dots, m}$. Dann gilt f\"ur $k \in \set{1, \dots, m}$
				\begin{align*}
					y_k &= (f \circ g)(y_k) = f(g(y_k))\\
					&= f\brac{\sum_{j=1}^n b_{jk}x_j}\\
					&= \sum_{j=1}^n b_{jk}f(x_j)\\
					&= \sum_{j=1}^n b_{jk}\sum_{i=1}^ma_{ij}y_i\\
					&= \sum_{i=1}^m\brac{\sum_{j=1}^n b_{jk}a_{ij} }y_i \overset{R\ \mathrm{komm.}}{=} \sum_{i=1}^m\brac{\sum_{j=1}^n a_{ij}b_{jk} }y_i
				\end{align*} 
			und daher
			\[\sum_{j=1}^n a_{ij}b_{jk} = \begin{cases}
				1 & \mathrm{falls }\ k = i\\
				0 & \mathrm{sonst}
			\end{cases}\] 
			f\"ur alle $i, k \in \set{1, \dots, m}$, d.h. $AB = I_m$.\newline
			Wegen $n < m$ k\"onnen wir $\displaystyle A' := (A\ \underbrace{0}_{(m-n)-\mathrm{Spalten}}) \in R^{m\times m}$ und $B' := \begin{pmatrix}
				B\\0
			\end{pmatrix} \in R^{m \times m}$ (mit $m-n$ $0$-Zeilen) setzen, so dass $A'B' = AB = I_m$. \newline
			Mit dem Determinantenproduktsatz folgt 
			\[0 = 0\cdot 0 = (\det A')(\det B') = \det(A'B') = 1\] 
		\end{proof}
	\end{satz}

	\begin{bemerkung}\label{1.2.12}\hfill\newline
		Statt Determinantentheorie \"uber kommutativen Ringen zu verwenden, kann man den Beweis des letzten Satzes auch mit der Theorie kommutativer Ringe auf die Dimensionstheorie von Vektorr\"aumen zur\"uckspielen. 
		
		\noindent Sei $R$ ein kommutativer Ring mit $1 \neq 0$, $m, n\in \N_0$ mit $R^m \cong R^n$. Wir zeigen $m = n$.
		
		\begin{proof}
			W\"ahle ein maximales Ideal $\mathfrak{m}$ von $R$. W\"ahle einen $R$-Modulisomorphismus $f~:~R^m~\to~R^n$. Betrachte die $R$-Untermoduln
			\[\mathfrak{m}R^m := \set{\sum_{i=1}^n a_ix_i \mid n \in \N_0, a_i \in \mathfrak{m}, x_i \in R^m} = \mathfrak{m}^m\]
			von $R^m$ und 
			\[f(\mathfrak{m}R^n) = \set{\sum_{i=1}^n a_iy_i \mid n \in \N_0, a_i \in \mathfrak{m}, y_i \in R^n} = \mathfrak{m}^n\]
			von $R^n$
			
			\noindent Mit dem Isomorphiesatz erhalten wir einen Modulisomorphismus $R^m/\mathfrak{m}^m \to R^n/\mathfrak{m}^n$ und offensichtlich gilt $R^m/\mathfrak{m}^m  \cong (R/\mathfrak{m})^m$ (betrachte z.B. $R^m \to (R/\mathfrak{m})^m$).\newline
			Da nun $(R/\mathfrak{m})^m$ und $(R/\mathfrak{m})^n$ als $R$-Moduln isomorph sind, sind sie auch als \linebreak $(R/\mathfrak{m})$-Moduln isomorph. F\"ur den K\"orper $K:= R/\mathfrak{m}$ gilt also 
			\[m = \dim_K K^m = \dim_K K^n = n\]
		\end{proof}
	\end{bemerkung}

	\begin{definition}\label{1.2.13}\hfill\newline
		Sei $R$ ein kommutativer Ring mit $1 \neq 0$ und $M$ ein freier $R$-Modul mit Basis $B$. Dann hei\ss t $\rk M := |B| \in \N_0 \cup \set{\infty}$ der \emph{Rang}\index{Modul@\textbf{Modul}!Rang} von $M$ [h\"angt nach \ref{1.2.11} nicht von der Wahl der Basis $B$ ab ]
	\end{definition}
	
	
	\newpage
	
	\section{Halbeinfache Moduln}\thispagestyle{sectionstart}
	\begin{notation}\label{1.3.1}\hfill\newline
		$0 := \set{0}$ Nullmodul
	\end{notation}
	
	\begin{definition}\label{1.3.2}\hfill\newline
		Ein Modul $M$ hei\ss t \emph{einfach}\index{Modul@\textbf{Modul}!Einfache Moduln} (oder irreduzibel), falls $M \neq 0$ und $0$ und $M$ die einzigen Untermoduln von $M$ sind.
	\end{definition}
	
	\begin{bemerkung}\label{1.3.3}\hfill\newline
		Sei $N$ ein Untermoduln von $M$.
		\begin{enumerate}[(a)]
			\item Bezeichne $\varphi: M \to M/N$ den kanonischen Epimorphismus. Dann vermitteln die Zuordnungen
			\begin{align*}
				L &\mapsto L/N = \varphi(L) \\
				\varphi^{-1}(P) &\mapsfrom P
			\end{align*}
			Eine Bijektion zwischen der Menge der Untermoduln $L$ von $M$ mit $N \subseteq L$ und der Menge der Untermoduln von $M/N$
					
			\item Es folgt, dass $M/N$ einfach ist genau dann, wenn $N$ ein maximaler echter Untermodul ist.
		\end{enumerate}
	\end{bemerkung}

	\begin{beispiel}\label{1.3.4}\hfill
		\begin{enumerate}[(a)]
			\item Sei $R$ ein kommutativer Ring und $I$ ein $R$-Untermodul von $R$, d.h. ein Ideal von $R$ [$\to$\ref{1.1.5}]. Dann ist $R/I$ ein einfacher $R$-Modul $\Leftrightarrow$ $I$ ist ein maximales Ideal von $R$ $\Leftrightarrow$ $R/I$ ist ein K\"orper.
			
			\item Sei $R$ ein Hauptidealring und $p \in R\setminus \set{0}$. Dann ist $R/pR$ ein einfacher Modul genau dann, wenn $p$ irreduzibel in $R$ ist.
			
			\begin{proof}\hfill\newline
				"$\Longrightarrow$": Ist $(p)$ ein maximales Ideal von $R$, so auch ein Primideal, d.h. $p$ ist prim in $R$ und daher auch irreduzibel in $R$ (wegen $p \neq 0$) 
				
				"$\Longleftarrow$": Ist $p$ irreduzibel in $R$, so ist $R/(p)$ ein K\"orper und daher ist $(p)$ ein maximales Ideal in $R$.
			\end{proof}
		\end{enumerate}
	\end{beispiel}

	\begin{lemma}\label{1.3.5}\hfill\newline
		Sei $R$ ein Ring und $M$ ein $R$-Modul. Es sind \"aquivalent:
		\begin{enumerate}[(i)]
			\item $M$ ist einfach
			\item $M \neq 0$ und jedes Element von $M \setminus \set{0}$ erzeugt $M$
			\item Es gibt einen maximalen echten $R$-Untermodul $N$ von $R$ mit $R/N \cong M$
		\end{enumerate}
	
		\begin{proof}\hfill\newline
			\textbf{(a)$\Longrightarrow$(c)}: Gelte (a) W\"ahle $x \in M \setminus \set{0}$. 
			
			Dann ist der Homomorphismus $\varphi~:~R~\to~M,~a~\mapsto~ax$ surjektiv und daher $R/N \cong M$ mit $N:=\ker \varphi$. Mit $M$ ist auch $R/N$ einfach, weswegen nach \ref{1.3.3}(b) $N$ ein maximaler echter Untermodul von $R$ ist.
			
			\medskip\noindent
			\textbf{(c)$\Longrightarrow$(b)}: trivial
			
			\medskip\noindent
			\textbf{(b)$\Longrightarrow$(a)}: trivial
		\end{proof}
	\end{lemma}

	\begin{lemma}\label{1.3.6}
		Lemma von Schur. \newline
		Sei $R$ ein Ring, $M$ und $N$ einfache $R$-Moduln und $f:M \to N$ ein Homomorphismus. Dann ist $f$ entweder die Nullabbildung oder ein Isomorphismus
		
		\begin{proof}
			Ist $f\neq 0$, so ist $\ker f \neq M$ und $\im f \neq 0$, also $\ker f = 0$ und $\im f = N$.
		\end{proof}
	\end{lemma}

	\begin{definition}\label{1.3.7}\hfill\newline
		Ein Modul hei\ss t \emph{halbeinfach}\index{Modul@\textbf{Modul}!Halbeinfache Moduln} (oder vollst\"andig reduzibel), wenn er direkte Summe von einfachen Moduln ist.
	\end{definition}

	\begin{lemma}\label{1.3.8}\hfill\newline
		Jeder endlich erzeugte Modul $\neq 0$ besitzt einen einfachen Quotienten.
		
		\begin{proof}
			Sei $M$ ein $R$-Modul und seinen $x_1, \dots, x_n \in M$ mit $0 \neq M = Rx_1 + \cdots + Rx_n$. Zu zeigen: Es gibt einen Untermodul $N$ von $M$ mit $M/N$ einfach.
			
			\noindent Betrachte die durch Inklusion halbgeordnete Menge 
				\[X:=\set{P \mid P\ \textrm{Untermodul von} \ M, P \subsetneq M} = \set{P \mid P\ \textrm{Untermodul von} \ M, \set{x_1, \dots, x_n} \nsubseteq P}\]
			
			\noindent Jede Kette $K \subseteq X$ besitzt eine obere Schranke in $X$ ($0$ f\"ur $K = \emptyset$, da $M \neq 0$ und $\bigcup K$ f\"ur $K \neq \emptyset$, da $\set{x_1, \dots, x_n}$ endlich)
			
			\noindent Nach dem Lemma von Zorn gibt es daher ein maximales Element $N$ in $X$. Gem\"a\ss \ \ref{1.3.3}(b) ist $M/N$ einfach. 
		\end{proof}
	\end{lemma}

	\begin{definition}\label{1.3.9}\hfill\newline
		Sei $M$ ein Modul und $N$ ein Untermodul von $M$. Dann hei\ss t $N$ ein \emph{direkter Summand}\index{Modul@\textbf{Modul}!Direkter Summand} von $M$, wenn es einen Untermodul $P$ von $M$ gibt mit $M = N \oplus P$.
	\end{definition}

	\begin{satz}\label{1.3.10}\hfill\newline
		Sei $M$ ein Modul. Dann sind folgende Aussage \"aquivalent
		\begin{enumerate}[(a)]
			\item $M$ ist halbeinfach
			
			\item $M$ ist die Summe seiner einfachen Untermoduln
			
			\item Jeder Untermodul von $M$ ist ein direkter Summand von $M$.
		\end{enumerate}	
		
		\begin{proof}\hfill\newline
			\textbf{(a)$\Longrightarrow$(b)}: klar
			
			\medskip\noindent
			\textbf{(b)$\Longrightarrow$(c)}: Gelte (b) und sei $N$ ein Untermodul von $M$.
				\[X:= \set{P \mid P \ \mathrm{Untermodul von }\ M, N \cap P = 0}\]
			
			\noindent Jede Kette $K \subseteq X$ besitzt eine obere Schranke in $X$ ($0$ f\"ur $K = \emptyset$, $\bigcup K$ f\"ur $K \neq \emptyset$)
			
			\noindent Nach dem Lemma von Zorn gibt es daher ein maximales Element $P$ in $X$. Um $M = N + P$ zu zeigen, reicht es wegen (b) zu zeigen, dass jeder einfache Untermodul $L$ von $M$ in $N + P$ enthalten ist. Sei also $L$ ein einfacher Untermodul von $M$. Dann ist entweder $L \cap (N + P) = 0$ oder $L \cap (N + P) = L$. Im letzteren Fall sind wir fertig.
			
			\noindent Der erste Fall tritt aber nicht ein:\newline
			Ist $L \cap (N+P) = 0$, so $(L+P)\cap N = 0$ (ist $x \in L$ und $y \in P$ mit $x+y \in N$, so $x \in L\cap (N+P) = 0$ und daher $y \in N\cap P = 0$), woraus wegen der Maximalit\"at von $P$ folgt $P = L + P$, also $L \subseteq P \lightning$
			
			\medskip\noindent 
			\textbf{(c)$\Longrightarrow$(a)}: Gelte (c).\newline
			{\color{red}
			\textbf{Hilfsbehauptung}: Jeder Untermodul eines Untermoduls $N$ von $M$ ist ein direkter Summand von $N$.
			
			\smallskip\noindent
			\textbf{Begr\"undung}: Sei $N$ ein Untermodul von $M$ und $P$ ein Untermodul von $N$.
			W\"ahle $Q$ mit $M = P \oplus Q$. Setze $R = Q \cap N$. Wir zeigen $N = P \oplus R$. Es ist klar, dass $P \cap R = 0$ (denn $P \cap Q = 0$) und $P + R \subseteq N$. Zu zeigen ist also noch $N \subseteq P + R$. \newline
			Sei hierzu $x \in N$. Schreibe $x = p + q$ mit $p \in P$ und $q \in Q$, dann $q = x - p \in N \cap Q = R$. } 
			
			\bigskip
			\noindent Betrachte nun die durch Inklusion halbgeordnete Menge 
				\[X:= \set{Y \mid Y\ \mathrm{Menge\ von\ einfachen\ Untermoduln\ von}\ M\ \mathrm{mit}\ \sum_{N \in Y}N = \bigoplus_{N \in Y}N}\]
				
			\noindent Sei $K$ eine Kette in $X$. Wir behaupten, dass dann $Z:= \bigcup K \in X$ gilt und $Z$ eine obere Schranke von $K$ in $X$ ist. \newline
			Zu zeigen: $\displaystyle \sum_{N \in Z} N = \bigoplus_{N \in Z} N$
			
			\noindent Seien nun $n \in \N$ und $N_1, \dots, N_n \in Z$ paarweise verschieden und $x_1 \in N_1, \dots, x_n~\in~N_n$ mit $x_1 + \cdots + x_n = 0$ [$\rightarrow$\ref{1.2.3}(b)]. Da $K$ eine Kette ist, gibt es $Y \in K$ mit $\set{N_1, \dots, N_n}~\subseteq~Y$. Wegen $\displaystyle \sum_{N \in Y}N = \bigoplus_{N \in Y}N$ folgt mit \ref{1.2.3}(b), dass $x_1 = \cdots = x_n = 0$.
			
			\noindent Da die Kette $K \subseteq X$ beliebig war, gibt es nach dem Lemma von Zorn ein in $X$ maximales Element $Z$. Setze $\displaystyle P = \sum_{N \in Z} N = \bigoplus_{N \in Z}N$. Wir zeigen $M = P$.
			
			\noindent Angenommen $M \setminus P \neq \emptyset$. W\"ahle gem\"a\ss\ (c) $Q$ mit $M = P \bigoplus Q$. Dann $Q \neq 0$. W\"ahle einen endlich erzeugten Untermodul $Q'\neq 0$ von $Q$. Nach Lemma \ref{1.3.8} gibt es einen Untermodul $Q''$ von $Q'$ mit $Q'/Q''$ einfach.
			
			\noindent W\"ahle gem\"a\ss\ Hilfsbehauptung $R$ mit $Q' = Q'' \bigoplus R$. Dann ist $R \subseteq Q' \subseteq Q$ und daher $P \cap R = 0$. Weiter ist $R\cong Q'/Q''$ einfach. Es folgt $\displaystyle \sum_{N \in Z \cup \set{R}}N = \bigoplus_{N \in Z \cup \set{R}}N$. Daher ist $Z\cup \set{R} \in X$. Wegen der Maximalit\"at von $Z$ in $X$ gilt $R \in Z$ und daher $R \subseteq P \lightning$.
 		\end{proof} 
	\end{satz}

	\begin{korollar}\label{1.3.11}\hfill\newline
		Direkte Summen, Untermoduln und Quotienten von halbeinfachen Moduln sind halbeinfach.
		
		\begin{proof}
			direkte Summen: klar nach \ref{1.3.7}\newline
			Untermoduln: Sei $N$ ein Untermodul des halbeinfachen Moduls $M$. Wir verwenden \ref{1.3.10}(c) um zu zeigen, dass $N$ auch halbeinfach ist. Sei also $L$ ein Untermodul von $N$. Da $M$ halbeinfach ist, gibt es einen Untermodul $P$ von $M$ mit $M = L \oplus P$. Dann gilt $N = L \oplus (P \cap N)$, wie man sofort sieht.\newline
			Quotienten: Sei $N$ ein Untermodul des halbeinfachen Moduls $M$. Zu zeigen: $M/N$ ist halbeinfach.\newline
			W\"ahle einen Untermodul $P$ von $M$ mit $M = N \oplus P$. Dann ist $M/N \cong P$ halbeinfach nach dem gerade Gezeigten (betrachte den Homomorphismus $M = N \oplus P \to P, x+y \mapsto y$ und wende den Homomorphiesatz an).
		\end{proof}
	\end{korollar}
	\newpage
	
	\section{Noethersche und artinsche Moduln}\thispagestyle{sectionstart}
	\begin{definition}\label{1.4.1}\hfill\newline
		Ein Modul $M$ hei\ss t \textcase{\emph{noethersch}\index{Modul@\textbf{Modul}!Noethersche Moduln}}{\emph{artinsch}\index{Modul@\textbf{Modul}!Artinsche Moduln}}, wenn jede \textcase{aufsteigende}{absteigende} Kette von Untermoduln $\case{M_1 \subseteq M_2 \subseteq \cdots}{M_1 \supseteq M_2 \supseteq \cdots}$ von $M$ station\"ar wird (d.h. $\exists k \in \N: \forall n \geq k: M_n = M_k$).
		
		\noindent Ein Ring $R$ hei\ss t \textcase{noethersch}{artinsch}, wenn er als $R$-Modul \textcase{noethersch}{artinsch} ist.
	\end{definition}

	\begin{bemerkung}\label{1.4.2}\hfill\newline
		Sei $R$ ein kommutativer Ring
		
		\begin{enumerate}[(a)]
			\item $R$ ist genau dann noethersch, wenn jede aufsteigende Kette von Idealen in $R$ station\"ar wird [$\to$\ref{1.1.5}]
			
			\item Ist $S = R[a_1, \dots, a_n]$ ein kommutativer Ring mit $n \in \N_0, a_1, \dots, a_n \in S$, so besagt der \emph{Hilbersche Basissatz}\index{Modul@\textbf{Modul}!Hilbertscher Basissatz}:
				$R$ noethersch $\Longrightarrow$ $S$ noethersch.
		\end{enumerate}
	\end{bemerkung}

	\begin{satz}\label{1.4.3}\hfill\newline
		Ein Modul ist noethersch genau dann, wenn alle seine Untermoduln endlich erzeugt sind [$\rightarrow$\ref{1.1.4}(d)].
	\end{satz}

	\begin{lemma}\label{1.4.4}\hfill\newline
		Seien $L, L'$ und $N$ Untermoduln des Moduls $M$ mit $L \subseteq L'$, $L \cap N = L' \cap N$ und $L+N=L'+N$. Dann gilt $L=L'$
		
		\begin{proof}
			Sei $x \in L'$. Zu zeigen ist $x \in L$. Schreibe $x = l + n$ mit $l \in L$ und $n \in N$. Dann ist $x-l = n \in L' \cap N = L \cap N$ und daher $x = (x-l)+l \in L$. 
		\end{proof}
	\end{lemma}

	\begin{satz}\label{1.4.5}\hfill\newline
		Sei $N$ ein Untermodul des Moduls $M$. Dann ist $M$ \textcase{noethersch}{artinsch} genau dann, wenn sowohl $N$ als auch $M/N$ \textcase{noethersch}{artinsch} ist.
		
		\begin{proof}
			klar mit \ref{1.3.3}(a) und \ref{1.4.4}
		\end{proof}
	\end{satz}

	\begin{korollar}\label{1.4.6}\hfill\newline
		Endliche Summen \textcase{noetherscher}{artinscher} Moduln sind auch \textcase{noethersch}{artinsch}.
		
		\begin{proof}
			Sind $N_1, \dots, N_n$ \textcase{noethersche}{artinsche} Untermoduln des Moduls $M$ mit $M = \sum_{i=1}^n N_i$, so gibt es nach \ref{1.2.2} einen Epimorphismus 
				\[\bigoplus_{i=1}^n N_i \to \sum_{i=1}^n N_i\]
			weshalb $M = \sum_{i=1}^nN_i \cong \brac{\bigoplus_{i=1}^n N_i} / L$ f\"ur einen Untermodul $L$ von $\bigoplus_{i=1}^nN_i$ gilt.\newline
			Mit \ref{1.4.5} reicht es daher, die Behauptung f\"ur direkte Summen zu zeigen.\newline
			Durch Induktion nach $n \in \N_0$ zeigen wir daher, dass f\"ur alle \textcase{noetherschen}{artinschen} $R$-Moduln $N_1, \dots, N_n$ auch $\bigoplus_{i=1}^n N_i$ \textcase{noethersch}{artinsch} ist.
			
			\noindent Induktionsanfang f\"ur $n = 0$: klar
			
			\noindent Induktionsschritt $n-1 \to n, (n \in \N)$: Seien $N_1, \dots, N_n$ \textcase{noethersche}{artinsche} $R$-Moduln. Dann ist $\bigoplus_{i=1}^{n-1} N_i$ \textcase{noethersch}{artinsch} nach Induktionsvoraussetzung. Wegen 
				\[\brac{\bigoplus_{i=1}^n N_i} / \brac{\bigoplus_{i=1}^{n-1} N_i} \cong N_n\]
			folgt mit \ref{1.4.5}, dass $\bigoplus_{i=1}^n N_i$ auch \textcase{noethersch}{artinsch} ist.
		\end{proof}
	\end{korollar}

	\begin{korollar}\label{1.4.7}\hfill\newline
		Jeder endlich erzeugte Modul \"uber einem \textcase{noetherschen}{artinschen} Ring ist \textcase{noethersch}{artinsch}.
		
		\begin{proof}
			Sei $R$ ein \textcase{noetherscher}{artinscher} Ring und $M$ ein endlich erzeugter $R$-Modul. Nach \ref{1.4.6} ist ohne Einschr\"ankung $M$ zyklisch. Dann ist $M \cong R/N$ f\"ur einen $R$-Untermodul $N$ von $R$. Mit $R$ ist nach \ref{1.4.5} auch $R/N$ \textcase{noethersch}{artinsch}.
		\end{proof}
	\end{korollar}

	\begin{definition}\label{1.4.8}\hfill\newline
		Sei $M$ ein Modul. Dann heißt 
			\[\ell(M) := \sup \set{n \in \N_0 \mid \textrm{es gibt Untermoduln } M_0, \dots, M_n \textrm{ von } M \textrm{ mit } M_0 \supsetneq \cdots \supsetneq M_n} \in \N_0 \cup \set{\infty}\]
		die \emph{L\"ange}\index{Modul@\textbf{Modul}!L\"ange} von $M$.
	
		\noindent Es heißt $M$ von endlicher L\"ange, wenn $\ell(M) < \infty$.
	\end{definition}

	\begin{beispiel}\label{1.4.9}\hfill\newline
		Sei $M$ ein Modul. Dann 
		\begin{enumerate}[(a)]
			\item $\ell(M) = 0 \Leftrightarrow M = 0$
			\item $\ell(M) = 1 \Leftrightarrow M$ ist einfach
		\end{enumerate}
	\end{beispiel}

	\begin{satz}\label{1.4.10}\hfill\newline
		Sei $N$ ein Untermodul des Moduls $M$. Dann gilt
			\[\ell(M) < \infty \Leftrightarrow \brac{\ell(M/N) < \infty \land \ell(N)< \infty}\]
		und falls $\ell(M) < \infty$
			\[\ell(M) = \ell(M/N) + \ell(N)\]
		
		\begin{proof}
			Man sieht sofort $\ell(M) = \sup \hat M, \ell(M/N) \overset{\ref{1.3.3}(a)}{=} \sup \hat{K}$ und $\ell(N) = \sup \hat N$ mit 
				\[\hat M := \set{m \in \N_0 \mid \exists \textrm{Untermoduln } M_0, \dots, M_m \textrm{ von } M: M = M_0 \supsetneq \cdots \supsetneq M_m = 0}\]
				\[\hat K := \set{k \in \N_0 \mid \exists \textrm{Untermoduln } L_0, \dots, L_k \textrm{ von } M: M = L_0 \supsetneq \cdots \supsetneq L_k = N}\]
				\[\hat N := \set{n \in \N_0 \mid \exists \textrm{Untermoduln } N_0, \dots, N_n \textrm{ von } M: N = N_0 \supsetneq \cdots \supsetneq N_n = 0}\]
			
			Offensichtlich gilt $\forall k \in \hat K: \forall n \in \hat N: k + n \in \hat M$, was "$\Longrightarrow$" und "$\geq$" beweist.
			
			\noindent Um "$\Longleftarrow$" und "$\leq$" zu beweisen, reicht es 
				\[\forall m \in \hat M: \exists k \in \hat K: \exists n \in \hat N: m \leq k + n\]
			zu zeigen. Sei hierzu $m \in \hat M$. 
			
			W\"ahle Untermoduln $M_0, \dots, M_m$ von $M$ mit $M = M_0 \supsetneq \cdots \supsetneq M_m = 0$.
			Setze $L_i := M_i + N$ und $N_i := M_i \cap N$ f\"ur $i \in \set{0, \dots, m}$. Nach Lemma \ref{1.4.4} ist dann jeweils mindestens eine der beiden Inklusionen $L_i \supseteq L_{i+1}$ und $N_i \supseteq N_{i+1}$ echt (f\"ur $i \in \set{0, \dots, m-1}$). Setzt man
				\[k := |\set{i \in \set{0, \dots, m - 1}| L_i \supsetneq L_{i+1}}| \in \hat K\] und
				\[n := |\set{i \in \set{0, \dots, m - 1}| N_i \supsetneq N_{i+1}}| \in \hat N\]
			so folgt $m \leq k + n$ 
		\end{proof}
	\end{satz}

	\begin{definition}\label{1.4.11}\hfill\newline
		Sei $M$ ein Modul. Es heißt $(M_0, \dots, M_n)$ eine \emph{Kompositionsreihe}\index{Modul@\textbf{Modul}!Kompositionsreihe} (der \emph{L\"ange} $n$) von $M$, wenn $M_0, \dots, M_n$ Untermoduln von $M$ sind mit 
			\[M=M_0 \supsetneq \cdots \supsetneq M_n = 0\]
		derart, dass die sogenannten Faktoren $M_i/M_{i+1}$ ($i \in \set{0, \dots, n-1}$) alle einfach sind.
	\end{definition}

	\begin{bemerkung}\label{1.4.12}\hfill\newline
		Jeder endliche Modul besitzt nat\"urlich eine Kompositionsreihe. Folgender Satz verallgemeinert dies.
	\end{bemerkung}

	\begin{satz}\label{1.4.13}\hfill\newline
		Sei $M$ ein Modul. Es sind folgende Aussagen \"aquivalent
		
		\begin{enumerate}[(a)]
			\item $\ell(M) < \infty$
			\item $M$ ist noethersch und artinsch
			\item $M$ besitzt eine Kompositionsreihe.
		\end{enumerate}
		
		\noindent In diesem Fall ist die L\"ange einer jeden Kompositionsreihe von $M$ gleich der L\"ange von $M$.
		
		\begin{proof}\hfill\newline
			\textbf{(a)$\Longrightarrow$(b)}: trivial
			
			\medskip\noindent
			\textbf{(b)$\Longrightarrow$(c)}: Sei $M$ noethersch und artinsch. Da $M$ noethersch ist, gibt es zu jedem Untermodul $N \neq 0$ von $M$ einen Untermodul $N'$ von $N$ mit $N/N'$ einfach (sonst k\"onnte man eine aufsteigende Kette $0 \subsetneq N_1 \cdots \subsetneq  \cdots$ von echten Untermoduln von $M$ konstruieren).
			
			\noindent Setze nun $M_0 = N$ und w\"ahle f\"ur $i = 0, 1, \dots$ solange $M_i \neq 0$ einen Untermodul $M_{i+1}$ von $M_i$ mit $M_i/M_{i+1}$ einfach.
			
			\noindent Dieses Verfahren bricht ab, da $M$ artinsch ist.
			
			\medskip\noindent
			\textbf{(c)$\Longrightarrow$(a)} und Zusatz: 
			Sei $(M_0, \dots, M_n)$ eine Kompositionsreihe von $M$. 
			
			\noindent Dann $\ell(M) \overset{\ref{1.4.10}}{=} \ell(M_0/M_1) + \cdots + \ell(M_{n-1}/M_n) \overset{\ref{1.4.9}}{=} n$
		\end{proof}
	\end{satz}

	\begin{satz}\label{1.4.14} Satz von Jordan-H\"older\newline
		Sei $M$ ein Modul endlicher L\"ange $n$ und seien $M = M_0 \supsetneq \cdots \supsetneq M_n = 0$ und $M~=~N_0~\supsetneq~\cdots~\supsetneq~N_n~=~0$ zwei Kompositionsreihen von $M$. Dann gibt es 
		$\sigma \in S_n$ mit $M_{i-1}/M_i \cong N_{\sigma(i)-1}/N_{\sigma(i)}$ f\"ur $i \in \set{1, \dots, n}$
		
		\begin{proof}
			Induktion nach $n \in \N_0$\newline
			$n = 0$: trivial\newline
			$n-1 \rightarrow n$ ($n \in \N$): Setze $L:=N_1$ und betrachte 
				\begin{align}
					M &= L + M_0 \supseteq \cdots \supseteq L + M_n = L \label{1.4.14.1}\tag{$*$}\\
					L &= L \cap M_0 \supseteq \cdots \supseteq L \cap M_n = 0 \label{1.4.14.2}\tag{$**$}
				\end{align}
			{\color{red}
			\noindent 
			\textbf{Hilfsbehauptung}: F\"ur alle $i \in \set{1, \dots, n}$ gilt 
			
			\textbf{entweder} $(L+M_{i+1})/(L+M_i) = 0$ und $(L\cap M_{i+1})/(L\cap M_i) \cong M_{i-1}/M_i$
			
			\textbf{oder} $(L+M_{i+1})/(L+M_i) \cong M_{i-1}/M_i$ und $(L\cap M_{i+1})/(L\cap M_i) = 0$
			
			\smallskip\noindent
			\textbf{Begr\"undung}: Sei $i \in \set{1, \dots, n}$. Ist $(L\cap M_{i-1})/(L\cap M_i) \neq 0$, so ist \linebreak $(L~\cap~M_{i-1})~/~(L~\cap~M_i)~\hookrightarrow~ M_{i-1}/M_i$
			ein Isomorphismus, da $M_{i-1}/M_i$ einfach.
			
			\noindent Ist $(L+M_{i-1})/(L+M_i) \neq 0$, so ist $M_{i-1}/M_i \twoheadrightarrow (L+M_{i-1})/(L+M_i)$ ein Isomorphismus, da $M_{i-1}/M_i$ einfach. Daher reicht es zu zeigen, dass genau einer der Moduln $(L\cap M_{i-1})/(L\cap M_i)$ und $(L+M_{i-1})/(L+M_i)$ ein Nullmodul ist.
			
			\noindent Wegen Lemma \ref{1.4.4} k\"onnen nicht beide $0$ sein. Es reicht daher zu zeigen, dass genau $n$ der $2n$ Inklusionen (\ref{1.4.14.1}) und (\ref{1.4.14.2}) echt sind. Dies folgt mit Obigem aus (\ref{1.4.13}), indem man aus (\ref{1.4.14.1}) und (\ref{1.4.14.2}) eine Kompositionsreihe gewinnt. }
			
			
			\medskip\noindent
			Da $M/L$ einfach ist, ist genau eine der $n$ Inklusionen in (\ref{1.4.14.1}) echt, etwa $L+M_{k-1} \supsetneq L + M_k$. Nach der Hilfsbehauptung erh\"alt man aus (\ref{1.4.14.2}) eine Kompositionsreihe von $L$ der L\"ange $n-1$ (beachte $L\cap M_{k-1} = L\cap M_k$). Da $L=N_1\supsetneq \cdots \supsetneq N_n = 0$ ebenfalls eine solche ist, gibt es nach Induktionsvoraussetzung eine Bijektion $\tau: \set{2, \dots, n} \to \set{1, \dots, n} \setminus \set{k}$ mit $N_{i-1}/N_i \cong (L\cap M_{\tau(i)-1})/(L\cap M_{\tau(i)}) \cong M_{\tau(i)-1} / M_{\tau(i)}$ f\"ur $i \in \set{2,\dots, n}$. Zusammen mit $N_0/N_1 \cong M/L = (L+M_{k-1})/(L+M_k) \cong M_{k-1}/M_k$ liefert dies die gew\"unschte Bijektion.
		\end{proof}
	\end{satz}

	\begin{definition}\label{1.4.15}
		[$\rightarrow$\ref{1.2.4}]\newline
		Sei $R$ ein Ring, $M$ ein $R$-Modul und $E\subseteq M$. Dann nennt man den $R$-Untermodul \linebreak $\ann(E) := \set{a \in R \mid \forall x \in E: ax = 0}$ von $R$ den Annihilator von $E$.
	\end{definition}

	\begin{bemerkung}\label{1.4.16}\hfill\newline
		Sei $R$ ein kommutativer Ring, $M$ ein $R$-Modul und $E\subseteq M$. Dann ist \linebreak $\ann(E)=\ann\brac{\sum_{x\in E} Rx}$. Insbesondere gilt f\"ur $M=R/aR$ mit $a \in R$, dass
			\[\ann(R/aR) = \ann(\set{\overline{1}}) = \ann(\overline{1}) = aR\]
	\end{bemerkung}

	\begin{beispiel}\label{1.4.17}\hfill
		\begin{enumerate}[(a)]
			\item Ist $V$ ein K-Vektorraum, so $\ell(V) = \dim(V)$
			\item Sei $R$ ein Hauptidealring, $n\in\N_0, p_1, \dots, p_n \in R$ irreduzibel und $m:= p_1 \cdot \cdots \cdot p_n$.
			
			Dann gilt $\ell(R/mR) = n$ und 
				\[R/mR \supsetneq p_1R/mR \supsetneq \cdots \supsetneq p_1\cdot\cdots\cdot p_nR/mR\]
			mit Faktoren $(p_1\cdot \cdot \cdot p_{i-1}R/mR)/(p_1\cdot \cdot \cdot p_iR/mR) \cong (p_1\cdot \cdot \cdot p_{i-1}R)/(p_1\cdot \cdot \cdot p_iR) \cong R/p_iR$
			f\"ur $i \in \set{1,\dots,n}$.
			
			\noindent Nach dem Satz von Jordan-H\"older gibt es f\"ur alle Kompositionsreihen \linebreak $R/mR = M_0 \supsetneq \cdots \supsetneq M_n = 0$ ein $\sigma \in S_n$ mit $M_{i-1}/M_i \cong R/p_{\sigma(i)}R$ und daher $\ann(M_{i-1}/M_i) = \ann(R/p_{\sigma(i)}R) \overset{\ref{1.4.16}}{=}p_{\sigma(i)}R$
			
			Die Faktoren einer jeden Kompositionsreihe von $R/mR$ liefern also bis auf Reihenfolge und Assoziiertheit genau die Faktoren von $m = p_1 \cdot \cdots \cdot p_n$.
  		\end{enumerate}
	\end{beispiel}

	\newpage
	\section{Unzerlegbare Moduln}\thispagestyle{sectionstart}
	\begin{definition}\label{1.5.1}\hfill\newline
		Ein Modul $M$ hei\ss t \emph{unzerlegbar}\index{Modul@\textbf{Modul}!Unzerlegbare Moduln}, falls $M \neq 0$ und f\"ur alle Untermoduln $L$ und $N$ von $M$ gilt 
			\[M = L\oplus N \Rightarrow (L = 0 \lor N = 0)\]
	\end{definition}

	\begin{bemerkung}\label{1.5.2}\hfill\newline
		Jeder einfache Modul [$\rightarrow$\ref{1.3.2}] ist unzerlegbar, aber die Umkehrung stimmt nicht, wie \ref{1.3.4}(b) in Verbindung mit Satz \ref{1.5.4} unten zeigt.
	\end{bemerkung}

	\begin{lemma}\label{1.5.3}\hfill\newline
		Sei $M$ ein zyklischer Modul
		\begin{enumerate}[(a)]
			\item Jeder direkte Summand von $M$ [$\rightarrow$\ref{1.3.9}] ist wieder zyklisch
			\item $M\cong R/N$ f\"ur einen $R$-Untermodul $N$ von $R$.
		\end{enumerate}
	
		\begin{proof}\hfill
			\begin{enumerate}[(a)]
				\item Seien $L$ und $N$ Untermoduln von $M$ mit $M = L \oplus N$. Schreibe $M=Rx$ mit $x \in M$ und $x = y+z$ mit $y\in L, z \in N$.
				Wir zeigen $L = Ry$. Sei hierzu $w \in L$. Zu zeigen ist, dass $w \in Ry$. Schreibe $w = ax$ mit $a \in R$. Dann $ax = ay + az$ und $az = ax - ay = w - ay \in L \cap N = 0$. Also $w = ax = ay \in Ry$. 
				
				\item Schreibe $M=Rx$ mit $x \in M$. W\"ahle f\"ur $N$ den Kern des $R$-Modulhomomorphismus $R\twoheadrightarrow M, a \mapsto ax$.
			\end{enumerate}
		\end{proof}
	\end{lemma}

	\begin{satz}\label{1.5.4}\hfill\newline
		Sei $R$ ein Hauptidealring und $a \in R$. Dann ist $R/aR$ unzerlegbar genau dann, wenn es ein Primelement $p\in R$ und ein $n\in\N$ gibt mit $(a)=(p^n)$
		
		\begin{proof}
			Ohne Einschr\"ankung $a \notin R^*$. Gebe es zun\"achst keine solchen $p$ und $n$. Dann gibt es $b, c \in R \setminus R^*$ mit $a = bc$ und $(b,c) = (1)$. Nach dem Chinesischen Restsatz ist dann der kanonische $R$-Modulhomomorphismus $R/aR \to (R/bR)\times (R/cR)$ bijektiv. Daher $R/aR \cong (\underbrace{R/bR}_{\neq 0}) \oplus (\underbrace{R/cR}_{\neq 0})$
			
			\noindent Seien nun $p\in R$ prim und $n\in \N$ mit $(a) = (p^n)$.
			Gelte $R/p^nR = L \oplus M$. Zu zeigen $L=0$ oder $M = 0$. Jede Kompositionsreihe von $R/p^nR$ hat L\"ange $n$ mit allen Faktoren isomorph zu $R/pR$ nach \ref{1.4.17}(b). Alle Faktoren von Kompositionsreihen von $L$ und $M$ sind daher isomorph zu $R/pR$, denn aus je zwei Kompositionsreihen von $(L \oplus M)/M \cong L$ und $M$ kann man eine solche von $R/p^nR$ gewinnen. Nach \ref{1.5.3} gibt es aber Ideale $I$ und $J$ von $R$ mit $L \cong R/I$ und $M \cong R/J$. Da $I$ und $J$ Hauptideale sind, folgt mit \ref{1.4.17}(b) also $I=(p^l)$ und $J = (p^m)$ f\"ur gewisse $l,m \in \N_0$.
			
			Nun gilt einerseits 
			\begin{flalign*}
				n &= \ell(R/p^nR) = \ell(L \oplus M)\\ 
				&= \ell((L \oplus M)/M) + \ell(M) && \ref{1.4.10}\\
				&= \ell(L) + \ell(M) = \ell(R/p^lR) + \ell(R/p^mR) = l + m
			\end{flalign*}
			und andererseits
			\begin{flalign*}
				(p^n) &= \ann(R/p^nR) && \ref{1.4.16}\\
				&= \ann(L) \cap \ann(M) && R/p^nR = L + M\\
				&= \ann(R/p^lR) \cap \ann(R/p^mR) = (p^l) \cap (p^m)
			\end{flalign*}
			Hieraus folgt $l = 0$ oder $m = 0$. Also $L = 0$ oder $M = 0$.
		\end{proof}
	\end{satz}

	\begin{satz}\label{1.5.5}\hfill\newline
		Jeder noethersche oder artinsche Modul ist die direkte Summe endlich vieler unzerlegbarer Untermoduln.
		
		\begin{proof}
			Sei $M$ ein \textcase{noetherscher}{artinscher} Modul. Zu jedem Untermodul $N \neq 0$ von $M$ gibt es einen \textcase{maximalen}{minimalen} direkten Summanden $\case{N'' \neq N}{N' \neq 0}$ von $N$ und daher Untermoduln $N'$ und $N''$ von $N$ mit $N = N' \oplus N''$ und $N'$ unzerlegbar.
			
			Setze nun $M_0 := M$ und w\"ahle f\"ur $i=0,1,\dots$ solange $M_i\neq 0$ Untermoduln $N_{i+1}$ und $M_{i+1}$ von $M_i$ mit $M_i = N_{i+1} \oplus M_{i+1}$ und $N_{i+1}$ unzerlegbar. Dieses Verfahren bricht ab, da $\case{N_1 \subsetneq N_1 \oplus N_2 \subsetneq \cdots}{M_0 \supsetneq M_1 \supsetneq \cdots}$ und $M$ \textcase{noethersch}{artinsch} ist. Ist $M_n = 0$, so $\displaystyle M = \bigoplus_{i=1}^n N_i$
		\end{proof}
	\end{satz}

	\begin{defueb}\label{1.5.6}\hfill\newline
		Sei $M$ ein Modul. Dann bildet 
			\[\End(M):= \set{f \mid f \textrm{ Endomorphismus von } M}\] 
			mit punktweiser Addition und der Hintereinanderschaltung als Multiplikation einen Ring, den sogenannten \emph{Endomorphismenring}\index{Modul@\textbf{Modul}!Endomorphismenring} von $M$.
	\end{defueb}

	\begin{lemma}\label{1.5.7} "Fitting-Zerlegung"\newline
		Sei $M$ ein Modul und $f \in \End(M)$ mit $\ker(f) = \ker(f^2)$ und $\im(f) = \im(f^2)$. Dann $M = \ker f \oplus \im f$
		
		\begin{proof}
			Zu zeigen
			\begin{enumerate}[(a)]
				\item $\ker f \cap \im f = 0$
				\item $M = \ker f + \im f$
			\end{enumerate}
			Zu (a): Sei $x \in \ker f \cap \im f$. W\"ahle $y \in M$ mit $x = f(y)$. Dann $f^2(y) = f(x) = 0$ und daher $y \in \ker(f^2) = \ker(f)$, d.h. $x = f(y) = 0$
			
			\medskip\noindent Zu (b): Sei $x \in M$. Wegen $f(x) \in \im f = \im(f^2)$ gibt es $y \in M$ mit $f(x) = f^2(y)$. Dann $x = \underbrace{(x - f(y))}_{\in \ker f} + \underbrace{f(y)}_{\in \im f}$
		\end{proof}
	\end{lemma}

	\begin{definition}\label{1.5.8}\hfill\newline
		Sei $R$ ein Ring (z.B. $R = \End(M)$ f\"ur einen Modul $M$)
		
		\begin{enumerate}[(a)]
			\item Ein Element $a \in R$ hei\ss t \textcase{\emph{idempotent}\index{Modul@\textbf{Modul}!idempotent}}{\emph{nilpotent}\index{Modul@\textbf{Modul}!nilpotent}}, wenn $\case{a^2 = a}{a^n = 0  \ (\textrm{f\"ur ein } n \in \N)}$ 
			
			\item $R$ hei\ss t \emph{lokal}\index{Modul@\textbf{Modul}!lokal}, wenn $0\neq 1$ in $R$ und $\forall a,b \in R\setminus R^*: a + b \in R\setminus R^*$
		\end{enumerate}
	\end{definition}

	\begin{proposition}\label{1.5.9}\hfill\newline
		Sei $M$ ein Modul. Dann ist $M$ unzerlegbar genau dann, wenn $\End(M)$ genau zwei idempotente Elemente hat (n\"amlich $0$ und $1 = \id_M \neq 0$).
		
		\begin{proof}\hfill\newline
			"$\Longrightarrow$": Sei $M$ unzerlegbar. Wegen $M \neq 0$ gilt $0 \neq 1$ in $\End(M)$.
			
			Sei $f \in \End(M)$ idempotent. Dann $M = \ker f \oplus \im f$ nach \ref{1.5.7}. Es folgt $\ker f = 0$ oder $\im f = 0$. Im zweiten Fall ist $f = 0$. Im ersten Fall ist $f$ injektiv, also $f=1$ (da $f^2 = f$).
			
			\medskip\noindent
			"$\Longleftarrow$": Seien $0\neq 1$ die einzigen idempotenten Elemente von $\End(M)$. Gelte $M = L \oplus N$. Zu zeigen $L = 0$ oder $N = 0$. 
			\[\pi_L:M=L\oplus N \to L, x + y \mapsto x\]
			($x\in L, y \in N$) ist idempotent, also $\pi_L = 0$ oder $\pi_L = 1$. Dann ist $L = 0$ oder $N = 0$.
		\end{proof}
	\end{proposition}

	\begin{lemma}\label{1.5.10} Fitting Lemma\newline
		Sei $M$ ein Modul endlicher L\"ange und $f \in \End(M)$. Dann gibt es $N \in \N$ mit \linebreak $M = \ker(f^n) \oplus \im(f^n)$ f\"ur alle $n\geq N$.
		
		\begin{proof}
			Die Ketten $\ker f \subseteq \ker f^2 \subseteq \cdots$ und $\im f \supseteq \im f^2 \supseteq \cdots$ werden station\"ar. W\"ahle $N \in \N$ mit $\ker f^n = \ker f^N$ und $\im f^n = \im f^N$ f\"ur alle $n \geq N$ und nehme die Fitting-Zerlegung nach \ref{1.5.7} f\"ur $f^n$.
		\end{proof}
	\end{lemma}

	\begin{korollar}\label{1.5.11}\hfill\newline
		Jeder Endomorphismus eines Unzerlegbaren Moduls endlicher L\"ange ist entweder nilpotent oder ein Automorphismus.
	\end{korollar}

	\begin{satz}\label{1.5.12}\hfill\newline
		Der Endomorphismenring eines unzerlegbaren Moduls endlicher L\"ange ist lokal.
		
		\begin{proof}
			Sei $M$ ein unzerlegbarer Modul mit $\ell(M) < \infty$. Wegen $M \neq 0$ gilt $0 \neq 1$ in $\End(M)$.
			
			Seien $f, g \in \End(M)$. Statt 
				\[(f\notin \End(M)^* \land g\notin \End(M)^*) \Longrightarrow (f+g \notin \End(M)^*)\]
			k\"onnen wir genauso gut (beachte $\End(M)^* = \Aut(M)$)
				\[(f\notin \Aut(M) \land f+g \in \Aut(M)) \Longrightarrow g \in \Aut(M)\]
			zeigen.
			
			Gelte also $f \notin \Aut(M)$ und $f+g \in \Aut(M)$. Zu zeigen ist $g \in \Aut(M)$.
			
			\noindent Mit $h:= (f+g)^{-1}$ gilt $hf + hg = h(f+g) = 1$. Wegen $(hf) \notin \Aut(M)$ ($f$ nilpotent nach \ref{1.5.11}, also $\ker f \neq 0$) gilt nach \ref{1.5.11} $(hf)^n = 0$ f\"ur ein $n \in \N$. 
			
			\noindent Dann gilt $hg = 1 - hf \in \Aut(M)$ und daher $g \in \Aut(M)$ (sonst $g$ nilpotent nach \ref{1.5.11}, also $\ker g \neq 0$),
			denn $(1+hf+(hf)^2 + \cdots + (hf)^{n-1})(1-hf) = 1$ und \linebreak $(1-hf)(1+hf+(hf)^2 + \cdots + (hf)^{n-1}) = 1$.
		\end{proof}
	\end{satz}

	\begin{satz}\label{1.5.13} Satz von Krull-Remak-Schmidt\newline
		Seien $m, n\in \N_0$ $M_1, \dots, M_m, N_1, \dots, N_n$ unzerlegbare Moduln endlicher L\"ange mit \linebreak $M_1 \oplus \cdots \oplus M_m \cong N_1 \oplus \cdots \oplus N_n$. Dann gilt $m=n$ und es gibt 
		$\sigma \in S_n$ mit $M_i \cong N_{\sigma(i)}$ f\"ur $i \in \set{1, \dots, n}$
		
		\begin{proof}
			Induktion nach $m \in \N_0$. $m = 0$: klar\newline
			$m-1 \to m$ ($m \in \N$)
			
			W\"ahle einen Isomorphismus $f: \underbrace{\bigoplus_{i=1}^m M_i}_{M:=}\to \underbrace{\bigoplus_{j=1}^n N_j}_{N:=}$.
			\begin{center}
				\begin{tikzcd}
				M_i \arrow[hook, yshift=0.7ex]{r}{\iota_i} & M \arrow[two heads, yshift=-0.7ex]{l}{\pi_i}
				\arrow{r}{f}[swap]{\cong} & N \arrow[two heads, yshift=-0.7ex]{r}[swap]{\rho_j}
				& N_i \arrow[hook', yshift=0.7ex]{l}[swap]{\kappa_j}
				\end{tikzcd}
			\end{center}	
			\begin{align*}
				1 &= \id_{M_1}  = \pi_1\iota_1\\
				&= \pi_1f^{-1}\id_Nf\iota_1\\
				&= \pi_1f^{-1}\brac{\sum_{j=1}^m \kappa_j\rho_j}f\iota_1\\
				&= \sum_{j=1}^m\underbrace{\pi_1f^{-1}\kappa_j}_{g_j:N_j \to M_1}\underbrace{\rho_jf\iota_1}_{h_j:M_1 \to N_j}
			\end{align*}
			Da $\End(M_1)$ nach \ref{1.5.12} lokal ist, gibt es $j \in \set{1, \dots, n}$ mit $g_jh_j \in \Aut(M_1)$. Insbesondere $n \geq 1$.
			
			\noindent {\color{red} \textbf{Behauptung 1}: \begin{tikzcd}
				M_1 \arrow[yshift=0.7ex]{r}{h_j} & N_j \arrow[yshift=-0.7ex]{l}{g_j}
			\end{tikzcd} sind Isomorphismen.
			
			\smallskip\noindent
			\textbf{Begr\"undung}: Wegen $g_jh_j \in \Aut(M)$ ist $h_j$ injektiv und $g_j$ surjektiv. Es gen\"ugt zu zeigen, dass $h_jg_j \in \Aut(N_j)$. Dies ist klar, denn sonst gilt nach \ref{1.5.12} $(h_jg_j)^s = 0$ f\"ur ein $s \in \N$ und damit 
			\[0 = g_j(h_jg_j)^s = (g_jh_j)^sg_j\]
			was $g_j = 0$ impliziert $\lightning$.
			
			\medskip\noindent
			\textbf{Behauptung 2}: $M = f^{-1}(N_j) \oplus M_2 \oplus \cdots \oplus M_m$.
			
			\smallskip\noindent
			\textbf{Begr\"undung}: Zu zeigen ist 
			\begin{enumerate}[(a)]
				\item $f^{-1}(N_j) \cap \sum_{i=2}^mM_i = 0$
				
				\item $M_1 \subseteq f^{-1}(N_j) + \sum_{i=2}^mM_i$
			\end{enumerate}
		
			\noindent Zu (a) Sei $x \in f^{-1}(N_j) \cap \sum_{i=2}^mM_i$. Zu zeigen ist $x = 0$. Dann gibt es ein $y \in N_j$ mit $x = (f^{-1}\kappa_j)(y)$ und es gilt $\pi_1(x) = 0$. Dann gilt $g_j(y) = (\pi_1f^{-1}\kappa_j)(y) = \pi_1(x) = 0$ und daher $y = 0$ (denn $g_j$ ist ein Isomorphismus), also $x = 0$.
			
			\medskip\noindent Zu (b) Sei $x \in M_1$. W\"ahle ein $y\in N_j$ mit $x = g_j(y)$. Dann $x = f^{-1}(y) + (x - f^{-1}(y))$ und es reicht zu zeigen, dass $\pi_1(x-f^{-1}(y)) = 0$. Es gilt aber 
			\begin{align*}
				\pi_1(x-f^{-1}(y))&=\pi_1(x) - (\pi_1(f^{-1}\kappa_j)(y))\\
				&= \pi_1(g_j(y)) - g_j(y) \\
				&= g_j(y) - g_j(y) = 0
			\end{align*} }
		
			\noindent Der Kern von \begin{tikzcd} M \arrow{r}{f}[swap]{\cong} & N \arrow[two heads]{r} & N/N_j \end{tikzcd} ist $f^{-1}(N_j)$ und es folgt mit dem Isomorphiesatz $M/f^{-1}(N_j) \cong N/N_j$, also
				\[\bigoplus_{i=2}^mM_i \overset{\textrm{Beh 2}}{\cong} M/f^{-1}(N_j) \cong N/N_j \cong \bigoplus_{k=1, k \neq j}^n N_k\]
			Wende die Induktionsvoraussetzung an.
		\end{proof}
	\end{satz}

	\newpage
	\section{Endlich erzeugte Moduln \"uber Hauptidealringen}\thispagestyle{sectionstart}
	
	\begin{definition}\label{1.6.1}\hfill\newline
		Sei $R$ ein Integrit\"atsring. Dann hei\ss t eine Funktion $\delta:R \to \N_0$ eine \emph{euklidische Funktion}\index{Modul@\textbf{Modul}!Euklidische Funktion} auf $R$, wenn es f\"ur alle $a \in R$ und $b \in R\setminus \set{0}$ Elemente $q \in R$ ("Quotienten") und $r \in R$ ("Rest") gibt mit $a = bq + r$ und $\delta(r) < \delta(b)$ ("Division mit Rest").
		
		\medskip\noindent 
		Es hei\ss t $R$ \emph{euklidisch}, wenn $R$ eine euklidische Funktion besitzt.
	\end{definition}

	\begin{beispiel}\label{1.6.2}\hfill
		\begin{enumerate}[(a)]
			\item $\Z$ ist euklidisch mit der euklidischen Funktion 
				\[\delta: \Z \to \N_0, a \mapsto |a|\]
			\item Ist $K$ ein K\"orper, so ist $K[X]$ euklidisch mit euklidischer Funktion 
				\[\delta:K[X]\to\N_0, p \mapsto \begin{cases}
					\deg p + 1 & \textrm{falls } p \neq 0\\
					0 & \textrm{falls } p = 0
				\end{cases}\]
			\item Der Ring der Gaußschen Zahlen $\Z[i] = \set{a+bi|a,b \in \Z}$ ist euklidisch mit euklidischer Funktion 
				\[\delta: \Z[i] \to \N_0, z \mapsto |z|^2\]
			denn zu $q \in \Z[i]$ mit $\left| \frac{a}{b} - q\right| \leq \brac{\frac{1}{2}}^2 + \brac{\frac{1}{2}}^2 = \frac{1}{2}$ und f\"ur $r:= a - bq$ gilt 
				\[\delta(r) = |r|^2 = \left| \frac{a}{b} - q\right||b|^2 \leq \frac{1}{2}|b|^2 \leq \frac{1}{2}\delta(b) < \delta(b)\]
		\end{enumerate}
	\end{beispiel}

	\begin{proposition}\label{1.6.3}\hfill\newline
		Jeder euklidische Ring ist ein Hauptidealring.
		
		\begin{proof}
			Sei $R$ ein Integrit\"atsring und $\delta:R\to\N_0$ eine euklidische Funktion. Sei $I$ ein Ideal in $R$, Ohne Einschr\"ankung $I \neq (0)$. W\"ahle $a \in I\setminus\set{0}$
			mit kleinstm\"oglichem $\delta(a)$. Wir zeigen $I= (a)$. Sei hierzu $x\in I$. Schreibe $x = aq+r$ mit $q,r \in R$, $\delta(r)< \delta(a)$. Dann ist $r = x-aq \in I$ und folglich $r=0$ gem\"a\ss\ Wahl von $a$. Also $x = aq \in (a)$.
		\end{proof}
	\end{proposition}

	\begin{erinnerung}\label{1.6.4}\hfill\newline
		Sei $R$ ein Hauptidealring. Wir fixieren eine Menge $\P_R$ von irreduziblen Elementen von $R$ derart, dass jedes irreduzible Element zu genau einem Element von $\P_R$ assoziiert ist, zum Beispiel $\P_\Z := \P := \set{2,3,5,\dots}$ und $\P_{K[X]} := \set{p \in K[X] \mid p \textrm{ normiert und irreduzibel}}$ ($K$ ein K\"orper).
		
		\noindent Betrachte $\displaystyle N_R:= \set{0} \cup \set{\prod_{i=1}^np_i \mid n \in \N_0, p_1, \dots, p_n \in \P_R}$. Zum Beispiel $N_\Z = \N_0$ und $N_{K[X]} = \set{p\in K[X] \mid p= 0 \textrm{ oder } p \textrm{ normiert}}$ ($K$ K\"orper).
		
		Seien $m,n \in \N_0$ und setze $l := \min \set{m, n}$. Eine Matrix $S=(s_{ij})_{1\leq i \leq m, 1 \leq j \leq n} \in N_R^{m\times n}$ hei\ss t in \emph{Smithscher Normalform}\index{Modul@\textbf{Modul}!Smithsche Normalform}, wenn $s_{ij} = 0$ f\"ur $i\neq j$ und $s_{ii} | s_{(i+1)(i+1)}$ f\"ur alle $i \in \set{1, \dots, l-1}$. 
		
		Betrachte die Gruppen $\GL_m(R) = (R^{m\times m})^* = \set{P \in R^{m \times m} \mid \det P \in R^*}$ und $\GL_n(R)$ und betrachte die \"Aquivalenzrelation $\sim$ auf $R^{m\times n}$ definiert durch \linebreak
		$A \sim B \Leftrightarrow \exists P \in \GL_m(R): \exists Q \in \GL_n(R): A=PBQ$ ($A, B \in R^{m \times n}$)
		
		
		Dann gibt es zu jedem $A \in R^{m \times n}$ genau ein $S \in R^{m\times n}$ in Smithscher Normalform mit $A \sim S$.
		F\"ur jedes $i \in \set{1, \dots, l}$ nennt man dann $c_i(A) := s_{ii}$ den $i$-ten \emph{Elementarteiler}\index{Modul@\textbf{Modul}!Elementarteiler} von $A$ und $d_i(A) := \gcd\set{i-\textrm{Minoren von } A} \in N_R$ den $i$-ten Determinantenteiler von $A$.
		
		\medskip\noindent Es gilt $\displaystyle d_i(A) = \prod_{j=1}^i c_j(A)$ f\"ur $i \in \set{1, \dots, l}$. 
		
		\noindent Mit $c(A) := (c_1(A), \dots, c_l(A))$ und $d(A) := (d_1(A), \dots, d_l(A))$ gilt f\"ur $A, B \in R^{m\times n}$
			\begin{align*}
				A \sim B &\Leftrightarrow c(A) = c(B)\\
				&\Leftrightarrow d(A) = d(B)\\
				&\Leftrightarrow A \textrm{ und } B \textrm{ haben dieselbe Smithsche Normalform}\\
				&\Leftrightarrow A \textrm{ und } B \textrm{ gehen aus Zeilen- und Spaltenoperationen vom Typ } (1), (2) \textrm{ oder } (3) \textrm{ hervor}\\
			\end{align*} 
		Dabei ist 
			\begin{enumerate}[(1)]
				\item $Z_i \gets Z_i + aZ_j$ oder $S_i \gets S_i + aS_j$ \hfill $(i\neq j,a \in R)$
				
				\item $Z_i \gets aZ_i$ oder $S_i \gets aS_i$ \hfill $(a \in R^*)$
				
				\item $\begin{pmatrix}  Z_i\\Z_j \end{pmatrix} \gets \begin{pmatrix} aZ_i + bZ_j\\cZ_i + dZ_j \end{pmatrix}$ oder 
					$\begin{pmatrix} S_i\\S_j \end{pmatrix} \gets \begin{pmatrix} aS_i + bS_j\\cS_i + dS_j \end{pmatrix}$ 
				\hfill ($i\neq j, a, b, c, d \in R, ad -bc = 1$)
			\end{enumerate}
		All diese Operationen sind umkehrbar. Die Operationen (1) und (3) ver\"andern die Determinante nicht, die Operation (2) ver\"andert sie nur bis auf eine Einheit. 
		
		Man \"uberlegt ich leicht, dass man mit den Operationen (1) und (2) die Operation (4)\newline
		$\begin{pmatrix} Z_i\\Z_j \end{pmatrix} \gets \begin{pmatrix} Z_j\\Z_i \end{pmatrix}$ oder 
		$\begin{pmatrix} S_i\\S_j \end{pmatrix}\gets \begin{pmatrix} S_j\\S_i \end{pmatrix}$ 
		\hfill ($i\neq j$)\newline

	simulieren kann. Ist $R$ \emph{euklidisch}, so \"uberlegt man sich, dass man damit auch (3) simulieren kann, weshalb in diesem Fall (3) \"uberfl\"ussig ist.
	
	Ist $A \in R^{m\times n}$ gegeben und interessiert man sich nicht nur f\"ur ein zu $A$ \"aquivalentes $B \in R^{m\times n}$ (z.B. die Smithsche Normalform), sondern auch f\"ur ein $P \in \GL_m(R)$ und $Q \in \GL_n(R)$ mit $B = PAQ$, so kann man im Schema \schemaSmith{$A$}{$I_m$}{$I_n$} Zeilenoperationen auf \schemaSmithH{$A$}{$I_m$} und Spaltenoperationen auf \schemaSmithV{$A$}{$I_n$} anwenden, um \schemaSmith{$B$}{$P$}{$Q$} mit $P \in \GL_m(R), Q \in \GL_n(R)$ und $B=PAQ$ zu erhalten.
	
	Interessiert man sich nicht nur f\"ur $P$ oder nur f\"ur $Q$, so arbeitet man mit dem Schema \schemaSmithH{$A$}{$I_m$} oder \schemaSmithV{$A$}{$I_n$}

 	\end{erinnerung}
 
	\begin{notation}\label{1.6.5}\hfill\newline
		Sei $R$ ein kommutativer Ring. Dann definiert jede Matrix $A \in R^{m \times n}$ einen $R$-Modulhomomorphismus
			\[f_A:R^n \to R^m, x \mapsto Ax\]
		Man nennt $\im A := \im f_A$ das \emph{Bild}\index{Modul@\textbf{Modul}!Bild (Matrix)} von $A$
	\end{notation}

	\begin{bemerkung}\label{1.6.6}\hfill
		\begin{enumerate}[(a)]
			\item Sei $R$ ein kommutativer Ring und seien $A, B \in R^{m\times n}, P \in \GL_m(R)$ und \linebreak $Q \in \GL_n(R)$ mit $B=PAQ$. Dann gilt $f_P(\im A) = \im B$, weshalb es (genau) einen $R$-Modulisomorphismus
			\begin{align*}
				R^m/\im A &\to R^m/\im B\\
				\overline{x} &\mapsto \overline{Px}
			\end{align*}
			($x \in R^m$) gibt.
			
			\item Sei $R$ ein Hauptidealring und sei $A \in R^{m\times n}$. Dann kann man mittels der Operationen (1), (2), (3) (falls $R$ euklidisch ist, reichen (1) und (2)) $A$ auf Smithsche Normalform bringen, wobei man die Zeilenoperationen auf \schemaSmithH{$A$}{$I_m$} anwendet, um \schemaSmithH{$S$}{$P$} zu erhalten mit $S \in R^{m \times n}$ in Smithscher Normalform und $P \in \GL_m(R)$, derart, dass $Q \in \GL_n(R)$ existiert mit $S=PAQ$.
		
			Da $S$ in Smithscher Normalform ist, kann man sofort $a_1, \dots, a_k \in N_R \setminus \set{1}$ mit $a_1|\dots|a_k$ ablesen mit
				\[\im S = R^{m-k} \times a_1R \times \cdots \times a_kR\]
				
			Gilt $P = \begin{pmatrix} &*&\\\hline b_{11} &\cdots& b_{1m}\\\vdots&\ddots&\vdots\\ b_{k1} &\cdots & b_{km}\end{pmatrix}$, so ist 
			\begin{align*} 
				R^m/\im A &\to \prod_{i=1}^k R/a_iR\\
				\overline{x} &\mapsto (\overline{b_{11}x_1+\cdots+b_{1m}x_m}, \dots, \overline{b_{k1}x_1+\cdots+b_{km}x_m})
			\end{align*}
		
			ein $R$-Modulisomorphismus
		\end{enumerate}
	
		\begin{beispiel}\label{1.6.7}\hfill
			\begin{align*}
				\Z^3 / \brac{\Z\begin{pmatrix}
						413\\-385\\427
					\end{pmatrix} + \Z\begin{pmatrix}
						140\\-126\\147
				\end{pmatrix}} &\overset{\cong}{\to} \Z/7\Z \times \Z/133\Z \times \Z\\
				\overline{(x_1, x_2, x_3)} &\mapsto (\overline{-x_1+x_3}, \overline{-9x_1-10x_2},\overline{3x_1+x_2-2x_3})
			\end{align*}
			
			denn \begin{tabular}{|cc|ccc|}\hline
			413& 140 & 1 & 0 & 0\\
			-385& -126 & 0 & 1 & 0\\
			427& 147 & 0 & 0 & 1\\\hline
		\end{tabular} kann man mit (1) und (2) in \begin{tabular}{|cc|ccc|}\hline
			7& 0 & -1 & 0 & 1\\
			0& 133 & -9 &-10 & 0\\
			0& 0 & 3 & 1 & -2\\\hline
		\end{tabular}
		
		\end{beispiel}
	\end{bemerkung}

	\begin{prop-def}\label{1.6.8}\hfill\newline
		Sei $R$ ein Integrit\"atsring und $M$ ein $R$-Modul. Dann bildet die Menge der Torsionselemente vo $M$ [$\to$\ref{1.2.4}] einen Untermodul
			\[T(M):= \set{x \in M| \exists a \in R \setminus \set{0}: ax = 0}\]
		von $M$, den wir den \emph{Torsionsteil}\index{Modul@\textbf{Modul}!Torsionsteil} von $M$ nennen.
	\end{prop-def}

	\begin{satz}\label{1.6.9} Struktursatz f\"ur endlich erzeugte Moduln \"uber Hauptidealringen\newline
		Sei $R$ ein Hauptidealring und $M$ ein endlich erzeugter $R$-Modul. Dann gibt es eindeutig bestimmte
		
		\begin{enumerate}[(a)]
			\item 
			$k\in \N_0$ und $a_1, \dots, a_k \in N_R\setminus \set{1}$ mit $a_1|a_2|\dots|a_k$ und $M\cong \prod_{i=1}^kR/a_iR$
 			
			\item $l, n\in \N_0$ und bis auf Reihenfolge eindeutige $(p_1, k_1), \dots, (p_l, k_l) \in \P_R \times \N$ mit $M \cong \brac{\prod_{i=1}^l R/p_i^kR} \times R^n$
		\end{enumerate}
	
		\begin{proof}
			\emph{Existenz}
			\begin{enumerate}[(a)]
				\item 
				Schreibe $M = Rx_1 + \cdots + Rx_m$ f\"ur ein $m \in \N_0$ und $x_1, \dots, x_m \in M$. Als Hauptidealring ist $R$ nat\"urlich noethersch (vgl. \ref{1.4.3} und \ref{1.1.5}). Daher ist auch $R^m$ noethersch (nach \ref{1.4.6} oder \ref{1.4.7}) und w\"ahle mit \ref{1.2.5} einen Homomorphismus \begin{tikzcd} f: R^m \rar[two heads] & M \end{tikzcd} mit $f(e_i) = x_i$ f\"ur alle $i \in \set{1, \dots, m}$. Als Untermodul von $R^m$ ist der Kern von $f$ endlich erzeugt und kann daher als Bild einer Matrix $A\in R^{m\times n}$ (mit $m$ gro\ss\ genug) geschrieben werden. $\ker f = \im A$. Nun gilt nach dem Isomorphiesatz $M \cong R^m/\ker f = R^m / \im A$ und wir k\"onnen das Verfahren aus Bemerkung \ref{1.6.6}(b) anwenden.
				
				\item 
				\begin{align}
					n:= |\set{i \in \set{1, \dots, k} \mid a_i = 0}| \label{1.6.9.1}\tag{$*$}
				\end{align}
				
				Zerlege die $a_i$ mit $a_i \neq 0$ in Produkte von Potenzen von paarweise verschiedenen Primfaktoren. Wende den Chinesischen Restsatz an (vgl. Beweis von \ref{1.5.4})
			\end{enumerate}
			
			\medskip\noindent\textbf{Eindeutigkeit}:
			
			\noindent Sowohl in (a) als auch in (b) kann man $n$ aus $M$ zur\"uckgewinnen, wobei im Fall (a) $n$ durch (\ref{1.6.9.1}) definiert sei. In der Tat gilt
			\begin{align}
				T(M) &\cong \prod_{i=1, a_i \neq 0}^k R/a_iR \label{1.6.9.2} \tag{$**$}
			\end{align}
			bzw. 
			\begin{align*}
				T(M) \cong \prod_{i=1}^l R/p_i^{k_i}R
			\end{align*}
			woraus $M/T(M) \cong R^n$ und daher $n \overset{\ref{1.2.13}}{=} \rk(M/T(M))$ folgt. Deswegen und wegen (\ref{1.6.9.2}) kann man nun sowohl in (a) als auch in (b) $n=0$ voraussetzen. 
			
			Da $M$ in (b) dann endllich L\"ange hat [$\to$\ref{1.4.17}(b)] folgt dort Eindeutigkeit sofort aus dem Satz von Krull-Remak-Schmidt \ref{1.5.13} in Verbindung mit \ref{1.5.4} und \ref{1.4.16}. Schlie\ss lich zu (a).
			
			Seien $k \in \N_0$ und $a_1, \dots, a_k, b_1, \dots, b_k \in N_R \setminus \set{0}$ mit $a_1|\dots|a_k, b_1|\dots|b_k$ und 
				\[\prod_{i=1}^k R/a_iR \cong \prod_{i=1}^kR/b_iR\]
				
			\noindent Es reicht zu zeigen, dass $(a_1, \dots, a_k) = (b_1, \dots, b_k)$. Wir zeigen dazu \linebreak $(a_j, \dots, a_k) = (b_j, \dots, b_k)$ f\"ur alle $j\in \set{1,\dots,k+1}$ durch Induktion nach $j$.
			
			$j=k:$ klar
			
			$j+1\to j$: ($j \in \set{1,\dots,k}$). Zu zeigen ist $a_j = b_j$. 
			
			\begin{align*}
				\underbrace{\prod_{i=1}^j a_jR/a_iR}_{N} \times \prod_{i=j+1}^k a_j(R/b_iR) &\cong a_j \prod_{i=1}^k R/b_iR\\
				&\cong a_j \prod_{i=1}^k R/a_iR\\
				&=\prod_{i=1}^k a_j(R/a_iR)\\
				&\cong \prod_{i=j+1}^k a_j(R/a_iR)\\
				&= \prod_{i=j+1}^k a_j(R/b_iR)
			\end{align*} 
		
			Da alle beteiligten Moduln endliche L\"ange haben (wegen $a_i, b_i \neq 0$), erhalten wir $\ell(N) = 0$, also $N=0$ und insbesondere $a_j(R/b_jR) = 0$, d.h. $a_j \in b_jR$. Analog $b_j \in a_jR$. Daher $(a_j) = (b_j)$ und wegen $a_j, b_j \in N_R$ gilt dann $a_j = b_j$
		\end{proof}
	\end{satz}

	\begin{korollar}\label{1.6.10}\hfill\newline
		Jede endlich erzeugte abelsche Gruppe ist isomorph zu einem direkten Produkt endlich vieler zyklischer Gruppen.
	\end{korollar}

	\begin{korollar}\label{1.6.11}\hfill\newline
		Jede endliche abelsche Gruppe ist isomorph zu einem direkten Produkt von zyklischen Gruppen von Primzahlpotenzordnung.
	\end{korollar}

	\begin{beispiel}\label{1.6.12} Fortsetzung von Beispiel \ref{1.6.7}\newline
		$133 = 7 \cdot 19$, also nach dem Chinesischen Restsatz
		\begin{align*}
			\Z^3/\brac{\Z\begin{pmatrix}413\\-385\\427\end{pmatrix} + \Z\begin{pmatrix}140\\-126\\147\end{pmatrix}} &\overset{\cong}{\rightarrow} (\Z/7\Z) \times (\Z/7\Z) \times (\Z/19\Z) \times \Z\\
				x &\mapsto (\overline{-x_1+x_3}, \overline{-9x_1-10x_2}, \overline{-9x_1-10x_2}, 3x_1+x_2-2x_3)
		\end{align*}
	\end{beispiel}

	\newpage
	\section{Der Satz von Cayley-Hamilton}\thispagestyle{sectionstart}
	\begin{def-prop}\label{1.7.1}\hfill\newline
		Sei $R$ ein Ring und $M$ ein $R$-Modul. F\"ur $A \in R^{m\times n}$ und $X\in M^{n\times r}$ definieren wir $AX$ durch
			\[(AX)_{ik} := \sum_{j=1}^n A_{ij}X_{jk}\]
		F\"r $1\leq i \leq m, 1 \leq k \leq r$
		
		Da $R$ ein $R$-Modul ist, verallgemeinern wir damit auch die Matrizenmultiplikation von kommutativen Ringen auf beliebige Ringe. Man rechnet sofort nach, dass f\"ur alle $A \in R^{m\times n}, B \in R^{n \times r}, X \in M^{r \times s}$ gilt $(AB)X = A(BX)$. Insbesondere wird $R^{n \times n}$ ein Ring mit $1 = I_n = \begin{pmatrix}
			1 & 0 & \cdots & 0\\
			0 & 1 & \cdots & 0\\
			\vdots & \vdots & \ddots & \vdots\\
			0 & 0 & \cdots & 1
		\end{pmatrix}$ und $M^{n \times r}$ ein $R^{n \times n}$-Modul [$\to$ 1.1.1].
	
		\noindent Im Folgenden benutzen wir, dass $M^n = M^{n\times 1}$ ein $R^{n\times n}$-Modul ist.
	\end{def-prop}
	
	\begin{def-ueb}\label{1.7.2} [vgl. \ref{1.5.6}]\newline
		Sei $R$ ein \emph{kommutativer} Ring und $M$ und $N$ $R$-Moduln. Dann bildet 
			\[\Hom(M,N) := \set{f \mid f:M \to N \textrm{ Homomorphismus}}\]
		mit punktweiser Addition und Skalarmultiplikation einen $R$-Modul.
	\end{def-ueb}

	\begin{bem-not}\label{1.7.3}\hfill\newline
		Sei $R$ ein \emph{kommutativer} Ring mit $0\neq 1$, $M$ ein $R$-Modul und $f \in \End(M)$. Dann ist 
		\[R[f] := \set{\sum_k a_k f^k \mid a_k \in R}\]
		ein \emph{kommutativer} Unterring von $\End(M)$.
		
		Es gibt genau einen Ringhomomorphismus $\varphi:R[X] \to R[f]$ mit \linebreak $\displaystyle \varphi\brac{\sum_k a_k X^k} = \sum_k a_k f^k$ f\"ur alle $a_k \in R$. Schreibe $p(f) = \varphi(p)$ f\"ur $p \in R[X]$.
	\end{bem-not}

	\begin{uebung}\label{1.7.4}\hfill\newline
		Sei $R$ ein \emph{kommutativer} Ring mit $0\neq 1$ und $M$ ein $R$-Modul. Dann vermitteln die Zuordnungen
		
		\begin{align*}
			f &\mapsto \begin{pmatrix}
				R[X]\times M &\to& M\\ (p,x) &\mapsto& (p(f))(x)
			\end{pmatrix}\\
			\begin{pmatrix}
				M &\to& M\\
				x &\mapsto& X \cdot x
			\end{pmatrix} &\mapsfrom \cdot
		\end{align*}
		Eine Bijektion zwischen $\End(M)$ und der Menge der Skalarmultiplikationen, die $M$ zu einem $R[X]$-Modul machen und die Skalarmultiplikation des $R$-Moduls $M$ fortsetzen.
	\end{uebung}

	\begin{satz}\label{1.7.5} Satz von Cayley-Hamilton\newline
		Sei $R$ ein kommutativer Ring, $I \subseteq R$ ein Ideal (z.B. $I = R$), $n\in \N_0$, $M$ ein $R$-Modul, der von $n$ Elementen erzeugt ist und $f \in \End(M)$ mit 
		$\displaystyle \im f \subseteq IM := \set{\sum_i a_ix_i \mid a_i \in I, x_i \in M}$. 
		Dann gibt es $a_1 \in I, a_2 \in I^2, \dots a_n \in I^n$ mit $f^n + a_1f^{n-1} + \cdots + a_n \id_M = 0$.
		
		\begin{proof}
			Ist $0 = 1$ in $R$, so $M = 0$ nach \ref{1.2.8}. Also sei ohne Einschr\"ankung $0\neq 1$ in $R$. Schreibe 
			$M = Rx_1 + \cdots + Rx_n$ mit $x_1, \dots, x_n \in M$. W\"ahle $A \in I^{n\times n}$ mit \linebreak $\displaystyle f(x_j) = \sum_{i=1}^n A_{ij}x_i$ f\"ur alle $j \in \set{1, \dots, n}$.
			Mache nun $M$ zu einem $R[X]$-Modul verm\"oge $X \cdot x = f(x)$ f\"ur alle $x \in R^n$ [$\to$ \ref{1.7.4}].
			
			Dann ist $M^n$ ein $R[X]^{n\times n}$-Modul [$\to$ \ref{1.7.1}], in dem gilt 
			\[\underbrace{\begin{pmatrix}
					X & 0 & \cdots & 0\\
					0 & X & \cdots & 0\\
					\vdots & \vdots & \ddots & \vdots\\
					0 & 0 & \cdots & X
			\end{pmatrix}}_{\in R[X]^{n\times n}} \begin{pmatrix}
		x_1\\x_2\\\vdots\\x_n
		\end{pmatrix} = \begin{pmatrix}
		X \cdot x_1\\X \cdot x_2\\\vdots\\X \cdot x_n
		\end{pmatrix} = \begin{pmatrix}
		f(x_1)\\f(x_2)\\\vdots\\f(x_n)
		\end{pmatrix} = \underbrace{A^T}_{\in R^{n\times n} \subseteq R[X]^{n\times n}}\underbrace{\begin{pmatrix}
			x_1\\x_2\\\vdots\\x_n
		\end{pmatrix}}_{\in M^n}\] 
		also  $(A^T-XI_n)\begin{pmatrix}x_1\\x_2\\\vdots\\x_n\end{pmatrix} = 0$. Multipliziere nun von links mit der transponierten \emph{Komatrix} $\brac{\com(A^T - XI_n)}^T = \com(A-XI_n)$.
		\begin{flalign*}
			0&= \com(A - XI_n)\brac{(A^T-XI_n)\begin{pmatrix}x_1\\x_2\\\vdots\\x_n\end{pmatrix}}\\
			&=  \com(A - XI_n)(A^T-XI_n)\begin{pmatrix}x_1\\x_2\\\vdots\\x_n\end{pmatrix}   &&\ref{1.7.1}\\
			&= \det(A-XI_n)I_n\begin{pmatrix}x_1\\x_2\\\vdots\\x_n\end{pmatrix}\\
			&= \underbrace{\det(A-XI_n)}_{=:p\in R[X]}\begin{pmatrix}x_1\\x_2\\\vdots\\x_n\end{pmatrix}
		\end{flalign*}
	
		Schreibe $p = (-1)^n(X^n +a_1x^{n-1}+\cdots + a_n)$. Aus $(p(f))(x_i) = p\cdot x_i = 0$ f\"ur alle $i \in \set{1,\dots, n}$ und $M=Rx_1+\cdots + Rx_n$ folgt $p(f) = 0$.
		\end{proof} 
	\end{satz}
	\newpage
	\chapter[tocentry={Ganze Ringerweiterungen und Dedekindringe}]{Ganze Ringerweiterungen und Dedekindringe}
	
	\section{Ganzheit}
	
	\begin{sprechweise}\label{2.1.1}\hfill\newline
		Sei $A$ ein Unterring des \emph{kommutativen} Ringes $B$, so sagen wir "$A \subseteq B$ ist eine \emph{Ringerweiterung}\index{Ringerweiterung@\textbf{Ringerweiterung}}". Die Sprechweisen
		"Die Ringerweiterung $A \subseteq B$ hat eine Eigenschaft" und "$B$ hat eine Eigenschaft \"uber $A$" sind synonym.
	\end{sprechweise}

	\begin{definition}\label{2.1.2}\hfill\newline
		Sei $A \subseteq B$ eine Ringerweiterung. Dann hei\ss t $x \in B$ \emph{ganz}\index{Ringerweiterung@\textbf{Ringerweiterung}!Ganze Ringerweiterung} \"uber $A$, wenn $0=1$ in $B$ oder wenn es ein \emph{normiertes} $f \in A[X]$ mit $f(x) = 0$ gibt ("Ganzheitsgleichung").
		Es hei\ss t $A \subseteq B$ \emph{ganz} (oder $B$ ganz \"uber $A$ vgl. \ref{2.1.1}), wenn jedes $x\in B$ ganz \"uber $A$ ist.
	\end{definition}

	\begin{beispiel}\label{2.1.3}\hfill
		\begin{enumerate}[(a)]
			\item $\sqrt{2}$ ist ganz \"uber $\Z$, da $(\sqrt{2})^2 -2 = 0$
			\item $\frac{1}{2}$ ist nicht ganz \"uber $\Z$, denn w\"aren $a_1,\dots, a_n \in \Z$ mit $\brac{\frac{1}{2}}^n + a_1\brac{\frac{1}{2}}^{n-1} + \cdots + a_n = 0$, so $1 + 2a_1 + \cdots + 2^na_n = 0\lightning$.
			\item $i$ und $i+1$ sind ganz \"uber $\Z$, denn $i^2+1 = 0$ und $(i+1)^2 - 2(i+1)+2 = 0$
			\item Eine K\"orpererweiterung $L/K$ ist algebraisch genau dann, wenn sie als Ringerweiterung $K \subseteq L$ ganz ist.
		\end{enumerate}	
	\end{beispiel}

	\begin{bemerkung}\label{2.1.4}\hfill\newline
		Ist $A$ ein Unterring von $B$, so ist $B$ in offensichtlicher Weise ein $A$-Modul
	\end{bemerkung}

	\begin{satz}\label{2.1.5}\hfill\newline
		Sei $A \subseteq B$ eine Ringerweiterung und $x \in B$. Es sind \"aquivalent
		\begin{enumerate}[(a)]
			\item $x$ ist ganz \"uber $A$
			\item $A[x]$ ist endlich erzeugt als $A$-Modul
			\item $A[x]$ ist in einem Unterring von $B$ enthalten, der ein endlich erzeugter $A$-Modul ist.
		\end{enumerate}
	
		\begin{proof}
			\textbf{(a)$\Longrightarrow$(b)$\Longrightarrow$(c)}: trivial
			
			\medskip\noindent
			\textbf{(c)$\Longrightarrow$(a)}: Sei $C$ ein Unterring von $B$, der $A[X]$ enth\"alt und als $A$-Modul endlich erzeugt ist.
			F\"ur den $A$-Modulendomorphismus $f:C\to C, a\mapsto ax$ gibt es nach Cayley-Hamilton \ref{1.7.5} $n\in\N_0$ und $a_1,\dots, a_n\in A$ mit 
			$f^n +a_1f^{n-1}+\cdots+a_n\id_C= 0$. Auswerten in $1$ liefert $x^n+a_1x^{n-1}+\cdots+a_n = 0$ 
 		\end{proof}
	\end{satz}

	\begin{lemma}\label{2.1.6}\hfill\newline
		Sei $A$ ein Unterring des Ringes $B$. Sei $B$ als $A$-Modul endlich erzeugt und sei $M$ ein endlich erzeugter $B$-Modul. Dann ist $M$ auch als $A$-Modul endlich erzeugt.
		
		\begin{proof}
			Schreibe $B=Ab_1 + \cdots + Ab_m$ mit $b_1, \dots, b_m \in B$ und \linebreak $M = Bx_1 + \cdots + Bx_n, x_1, \dots, x_n \in M$. Dann 
			\[M = \sum_{i=1}^{m} \sum_{j=1}^{n} A(b_ix_j)\]
		\end{proof}
	\end{lemma}
	
	\begin{korollar}\label{2.1.7}\hfill\newline
		Sei $A \subseteq B$ eine Ringerweiterung und seien $x_1, \dots, x_n \in B$ ganz \"uber $A$. Dann ist $A[x_1, \dots, x_n]$ ein endlich erzeugter $A$-Modul.
		
		\begin{proof}
			Benutze \ref{2.1.5} $(a) \Longrightarrow (b)$, \ref{2.1.6} und Induktion nach $n$
		\end{proof}
	\end{korollar}

	\begin{korollar}\label{2.1.8} "Transitivit\"at der Ganzheit"\newline
		Seien $A \subseteq B \subseteq C$ zwei Ringerweiterungen und sei $A \subseteq B$ ganz. Ist $a \in C$ ganz \"uber $B$, so auch \"uber $A$.
		
		\begin{proof}
			Seien $b_1, \dots, b_n \in B$ Koeffizienten einer Ganzheitsgleichung von $a$ \"uber $B$. Dann ist $A[b_1, \dots, b_n]$ ein endlich erzeugter $A$-Modul nach \ref{2.1.7} und $A[b_1, \dots, b_n][a]$ ein endlich erzeugter $A$-Modul nach \ref{2.1.5}. Nach \ref{2.1.6} ist $A[b_1, \dots, b_n, a]$ dann ein endlich erzeugter $A$-Modul. Mit \ref{2.1.5} ist dann $a$ ganz \"uber $A$.
		\end{proof}
	\end{korollar}
	
	\begin{korollar}\label{2.1.9}\hfill\newline
		Seien $A \subseteq B \subseteq C$ Ringerweiterungen. Dann ist $A \subseteq C$ ganz genau dann, wenn $A \subseteq B$ und $B \subseteq C$ ganz sind.
	
	\end{korollar}
	
	
	\begin{def-satz}\label{2.1.10}\hfill\newline
		Sei $A \subseteq B$ eine Ringerweiterung. Dann bilden die Elemente von $B$, die ganz \"uber $A$ sind, einen Unterring von $B$, der $A$ enth\"alt, den sogenannten \emph{ganzen Abschluss}\index{Ringerweiterung@\textbf{Ringerweiterung}!Ganzer Abschluss} von $A$ in $B$.
		
		\begin{proof}
			Jedes Element von $A$ ist nat\"urlich ganz \"uber $A$. Sind $x,y \in B$ ganz \"uber $A$, so auch $x+y, -x, x\cdot y$, denn $x+y,-x,x\cdot y\in A[x,y]$ und $A[x,y]$ ist nach \ref{2.1.7} ein endlich erzeugter $A$-Modul (benutze \ref{2.1.5}$(c)\Longrightarrow(a)$) 
		\end{proof} 
 	\end{def-satz}
 
 	\begin{definition}\label{2.1.11}\hfill\newline
 		Sei $A \subseteq B$ eine Ringerweiterung. Dann hei\ss t $A$ \emph{ganz abgeschlossen}\index{Ringerweiterung@\textbf{Ringerweiterung}!Ganz abgeschlossen} in $B$, wenn kein Element von $B\setminus A$ ganz \"uber $A$ ist (d.h. der ganze Abschluss von $A$ in $B$ gleich $A$ ist).
 	\end{definition}
 
 	\begin{definition}\label{2.1.12}\hfill\newline
 		Sei $A$ ein kommutativer Ring
 		\begin{enumerate}[(a)]
 			\item Der \emph{ganze Abschluss}\index{Ringerweiterung@\textbf{Ringerweiterung}!Ganzer Abschluss} von $A$ ist der ganze Abschluss von $A$ in seinem totalen Quotientenring $Q(A)$
 			
 			\item $A$ hei\ss t \emph{ganz abgeschlossen}\index{Ringerweiterung@\textbf{Ringerweiterung}!Ganz abgeschlossen}, wenn $A$ ganz abgeschlossen in $Q(A)$ ist.
 		\end{enumerate}
 		Erinnerung: $\displaystyle A \subseteq Q(A) = \set{\frac{a}{s} \mid a,s \in A, \nexists b \in A\setminus\set{0}: sb = 0}$.
 		
 		Ist $A$ ein Integrit\"atsring, so $\displaystyle A \subseteq Q(A) = \qf(A) = \set{\frac{a}{s} \mid a,s \in A, s \neq 0}$ 
 	\end{definition}
 
 	\begin{proposition}\label{2.1.13}\hfill\newline
 		Jeder faktorielle Ring ist ganz abgeschlossen.
 		
 		\begin{proof}
 			Sei $A$ ein faktorieller Ring und $x \in \qf(A)$ ganz \"uber $A$. Schreibe $\displaystyle x = \frac{a}{s}$, \linebreak $a, s \in A, s \neq 0$ und $\gcd \set{a,s} = 1$. W\"ahle $a_1,\dots, a_n \in A$ mit $x^n + a_1x^{n-1}+\cdots + a_n = 0$. Multiplizieren mit $s^n$ liefert $a^n + a_1sa^{n-1}+\cdots + a_ns^n = 0$, woraus $s \mid a^n$ folgt. Wegen $\gcd\set{a,s} = 1$, folgt $s \in A^*$, also $\displaystyle\frac{a}{s} \in A$.
 		\end{proof}
 	\end{proposition}
 
 	\begin{satz}\label{2.1.14}\hfill\newline
 		Sei $A$ ein ganz abgeschlossener Integrit\"atsring, $K:= \qf(A)$ und $L/K$ eine K\"orpererweiterung. Dann ist ein Element $x\in L$ genau dann ganz \"uber $A$, wenn es algebraisch \"uber $K$ mit Minimalpolynom $\irr_K(x) \in A[X]$.
 		
 		\begin{proof}
 			Sei $x\in L$ ganz \"uber $A$, etwa $f\in A[X]$ normiert mit $f(x) = 0$. Zu zeigen $\irr_K(x) \in A[X]$.
 			Schreibe $\irr_K(x) = \prod_{i=1}^n (X-a_i) \in \overline{L}[X]$ mit $a_1,\dots,a_n \in \overline{L}$. Wegen $\irr_K(x)|f$ in $K[X]$ gilt $f(a_i) = 0$ f\"ur alle $i\in\set{1,\dots,n}$. Daher ist jedes $a_i$ ganz \"uber $A$ und damit auch alle Koeffizienten von $\irr_K(x)$, welche damit nicht nur in $K$, sondern sogar in $A$ liegen.
 		\end{proof}
 	\end{satz}
 
 	\begin{definition}\label{2.1.15}\hfill\newline
 		Ein \emph{Zahlk\"orper}\index{Ringerweiterung@\textbf{Ringerweiterung}!Zahlk\"orper} ist ein Oberk\"orper $K$ von $\Q$ mit $K/\Q$ endlich. Ist $K$ ein Zahlk\"orper, so hei\ss t $[K:\Q]$ der \emph{Grad}\index{Ringerweiterung@\textbf{Ringerweiterung}!Grad} von $K$ und der ganze Abschluss von $\Z$ in $K$ hei\ss t der \emph{Zahlring}\index{Ringerweiterung@\textbf{Ringerweiterung}!Zahlring} $\mathcal{O}_K$ (oder Ganzheitsring) von $K$. Zahlk\"orper von Grad $2$ und ihre Zahlringe hei\ss en \emph{quadratisch} 
 	\end{definition}
 
 	\begin{notation}\label{2.1.16}\hfill
 		\begin{itemize}
 			\item $\nsquare := \set{d\in\Z \setminus\set{0,1} \mid \nexists p \in \P: p^2|d}$
 			\item $\nsquare_+ := \set{d\in\Z \setminus\set{0,1} \mid \nexists p \in \P: p^2|d, d > ß}$
 			\item $\nsquare_- := \set{d\in\Z \setminus\set{0,1} \mid \nexists p \in \P: p^2|d, d < 0}$
 			\item $\nsquare_{2,3} := \set{d\in\Z \setminus\set{0,1} \mid \nexists p \in \P: p^2|d, d \equiv_{(4)} 2 \textrm{ oder } d \equiv_{(4)} 3 } $
 			\item $\nsquare_{1} := \set{d\in\Z \setminus\set{0,1} \mid \nexists p \in \P: p^2|d, d \equiv_{(4)} 1 } $
 		\end{itemize}
 	\end{notation}
 
 	\begin{proposition}\label{2.1.17}\hfill\newline
 		F\"ur jedes $d\in \nsquare$ ist $\Q(\sqrt{d}) = \Q \oplus \Q\sqrt{d}$ ein quadratischer Zahlk\"orper  mit Zahlring \[\mathcal{O}_d := \mathcal{O}_{\Q(\sqrt{d})} = \begin{cases}
 			\Z[\sqrt{d}] = \Z \oplus \Z\sqrt{d} & \textrm{falls } d\in \nsquare_{2,3}\\
 			 \Z[\frac{1+\sqrt{d}}{2}] = \Z \oplus \Z\frac{1+\sqrt{d}}{2} & \textrm{falls } d\in \nsquare_{1}
 		\end{cases}\]
 	
 	Zu jedem quadratischen Zahlk\"orper gibt es genau ein $d\in \nsquare$ mit $K \cong \Q(\sqrt{d})$
 	
 	\begin{proof}
 		$\sqrt{d} \notin \Q$ wegen $d\in \nsquare$. Mit $\sqrt{d}^2 = d$ und $\brac{\frac{1+\sqrt{d}}{2}}^2 = \frac{1+d}{4}+\frac{\sqrt{d}}{2} = \frac{d-1}{4} + \brac{\frac{1+\sqrt{d}}{2}}$ folgen die drei Zerlegungen in direkte Summen. 
 		
 		Seien $a,b\in \Q$. Wir behaupten 
 		\begin{align}
 			a+b\sqrt{d} \in \mathcal{O}_d \Leftrightarrow a+ b\sqrt{d} \in \begin{cases}
 				\Z[\sqrt{d}] & \textrm{falls } d\in \nsquare_{2,3}\\
 				\Z[\frac{1+\sqrt{d}}{2}] & \textrm{falls } d\in \nsquare_1 \label{2.1.17.1}\tag{$*$}
 			\end{cases}
 		\end{align}
 		
 		Ohne Einschr\"ankung $b\neq 0$. Dann
 		\begin{align*}
 			(X-(a+b\sqrt{d}))(X-(a-b\sqrt{d})) &= ((X-a)+b\sqrt{d})((X-a)-b\sqrt{d})\\
 			&= (X-a)^2 -b^2d\\
 			&= X^2 - 2aX + (a^2-b^2d)\\
 			&= \irr_\Q(a+b\sqrt{d})
 		\end{align*}
 		
 		daher 
 		\begin{flalign*}
 			a+b\sqrt{d} \in \mathcal{O}_d &\Leftrightarrow \set{2a, a^2-b^2d} \subseteq \Z && \ref{2.1.14}\\
 			&\Leftrightarrow \set{2a, a^2-b^2d,4b^2d} \subseteq \Z\\
 			&\Leftrightarrow \set{2a, a^2-b^2d, 2b} \subseteq \Z && d \in \nsquare\\
 		\end{flalign*}
 	
 		Setzt man $x:= 2a, y:= 2b$, so folgt aus beiden Seiten $\set{x,y} \subseteq \Z$, weshalb wir dies ab jetzt annehmen. Also 
 		\begin{align}
 			a+b\sqrt{d}\in \mathcal{O}_d \Leftrightarrow \overline{x}^2 = \overline{y}^2\overline{d} \textrm{ in } \Z/(4) \label{2.1.17.2}\tag{$**$}
 		\end{align}
 		
 		Wegen $\overline{x}^2, \overline{y}^2 \in \set{\overline{0},\overline{1}}\subseteq \Z/(4)$ gilt
 		
 		\begin{itemize}
 			\item Im Fall $d\in \nsquare_{2,3}$
 			\begin{align*}
 				(**) &\Leftrightarrow \overline{y}^2 = 0 \land \overline{x}^2 = 0\\
 				&\Leftrightarrow \set{\overline{x},\overline{y}} \subseteq \set{\overline{0}, \overline{2}} \subseteq \Z/(4)\\
 				&\Leftrightarrow \set{x,y} \subseteq 2\Z\\
 				&\Leftrightarrow \set{a,b}\subseteq \Z\\
 				&\Leftrightarrow a+b\sqrt{d} \in \Z[\sqrt{d}]
 			\end{align*}
 			
 			\item Im Fall $d\in \nsquare_{1}$
 			\begin{align*}
 				(**) &\Leftrightarrow \overline{x}^2 = \overline{y}^2\\
 				&\Leftrightarrow x \equiv_{(2)} y\\
 				&\Leftrightarrow\frac{x-y}{2} \in \Z\\
 				&\Leftrightarrow a-b \in \Z\\
 				&\Leftrightarrow (a-b) + 2b\brac{\frac{1+\sqrt{d}}{2}} \in \Z\left[\frac{1+\sqrt{d}}{2}\right]
 			\end{align*}
 		\end{itemize}
 	
 		Dass jeder quadratische Zahlk\"orper $K$ zu einem $\Q(\sqrt{d})$ mit $d\in \nsquare$ isomorph ist, ist klar.
 		Sind schlie\ss lich $d,e\in \nsquare$ mit $\Q(\sqrt{d}) \cong \Q(\sqrt{e})$, so gibt es $a,b\in \Q$ mit $(a+b\sqrt{d})^2 = e$. Es folgt $2ab = 0$. Es gilt $b\neq 0$, da $e \in \nsquare$. Also
 		$a = 0$ und $b^2d = e$. Da $e\in \nsquare$, folgt $b^2=1$, also $d=e$
 	\end{proof}
 	\end{proposition}
 	\newpage
 	
 	\section{Dedekindringe}\thispagestyle{sectionstart}
 	\begin{erinnerung}\label{2.2.1}\hfill\newline
 		Sei $R$ ein kommutativer Ring
 		\begin{enumerate}[(a)]
 			\item Ein Ideal $\mathfrak{p}$ in $R$ hei\ss t prim (oder Primideal), wenn $1\notin \mathfrak{p}$ und
 				\[\forall a,b \in R: (ab \in \mathfrak{p} \Rightarrow (a \in \mathfrak{p} \lor b \in \mathfrak{p}))\]
 			
 			\item Ein Element $p \in R$ hei\ss t prim (oder Primelement), wenn $(p)$ ein Primideal ist, d.h. $p\notin R^*$ und
 				\[\forall a,b \in R: (p \mid ab \Rightarrow (p \mid a \lor p \mid b))\]
 			
 			\item Sind $I_1, \dots, I_n$ Ideale von $R$, so nennt man
 			\begin{align*}
 				\prod_{k=1}^n I_k &:= I_1 \cdot \cdots \cdot I_n := \brac{\set{a_1\cdot\cdots\cdot a_n \mid a_1 \in I_1, \dots, a_n \in I_n}}\\
 				&= \begin{cases}
 					R & \textrm{falls } n = 0\\
 					\set{\sum_i a_{i_1}\dots a_{i_n} \mid a_{i_1} \in I_1, \dots a_{i_n} \in I_n} & \textrm{falls } n\geq 1
 				\end{cases}
 			\end{align*} 
 			deren \emph{Produkt}\index{Dedekindringe@\textbf{Dedekindringe}!Produkt} 
 			
 			\item In Integrit\"atsringen sind Primfaktorzerlegungen (im Wesentlichen) eindeutig, d.h. ist $R$ ein Integrit\"atsring, $m,n \in\N_0$ und $p_1, \dots, p_m, q_1, \dots, q_n \in R\setminus \set{0}$ prim mit $p_1\cdot\cdots\cdot p_m = q_1\cdot\cdots\cdot q_m$, so gilt $m=n$ und es gibt ein $\sigma \in S_n$ mit $(p_i) = (q_{\sigma(i)})$ f\"ur $i \in \set{1,\dots, n}$ (vergleiche auch \ref{1.4.17}(b)).
 			
 			\item $R$ hei\ss t faktoriell, wenn $R$ ein Integrit\"atsring ist und jedes $a \in R$ eine Primfaktorzerlegung besitzt, d.h. es gibt $c \in R^*, n \in \N_0$ und Primelemente $p_1, \dots, p_n \in R$ mit $a = cp_1 \cdot \cdots \cdot p_n$.
 			
 			\item $R$ euklidisch $\overset{\ref{1.6.3}}{\Longrightarrow}$ $R$ Hauptidealring $\Longrightarrow$ $R$ faktoriell $\Longrightarrow$ $R$ Integrit\"atsring  
 		\end{enumerate}
 	\end{erinnerung}
 
 	\begin{beispiel}\label{2.2.2}\hfill\newline
 		Wegen $-5 \equiv_{(4)} -1 \equiv_{(4)} 3$ sind $\Z[\sqrt{-1}] = \Z[i]$ und $\Z[\sqrt{-5}] = \Z[\sqrt{5}i]$ quadratische Zahlringe [$\to$\ref{2.1.17}].
 		
 		Der Beweis von \ref{1.6.2}(c) daf\"ur, dass $\Z[\sqrt{-1}]$ euklidisch ist, funktioniert f\"ur $\Z[\sqrt{-5}]$ nicht mehr, da man nur noch 
 			\[\left| \frac{a}{b} - q\right| \leq \brac{\frac{1}{2}}^2 + \brac{\frac{\sqrt{5}}{2}}^2 = \frac{1}{4} + \frac{5}{4} = \frac{6}{4}\]
 		erh\"alt, aber $\frac{6}{4} \geq 1$. 
 		
 		Tats\"achlich ist $\Z[\sqrt{-5}]$ nicht einmal faktoriell, denn $2$ besitzt darin keine Primfaktorzerlegung, weil sonst $2$ prim in $\Z[\sqrt{-5}]$ sein m\"usste, was aber nicht der Fall
 		ist, denn $2 \mid 6 = (1+\sqrt{-5})(1-\sqrt{-5})$, aber $2\nmid (1+\sqrt{-5})$ und $2 \nmid (1+\sqrt{-5})$ in $\Z[\sqrt{-5}]$. 
 		
 		\medskip\noindent Andererseits besitzt $(2)$ eine \emph{Primidealzerlegung}\index{Dedekindringe@\textbf{Dedekindringe}!Primidealzerlegung}, d.h. es ist Produkt von Primidealen, denn $(2,1+\sqrt{-5}) \in \Z[\sqrt{-5}]$ ist ein Primideal mit 
 		\begin{align*}
 			(2,1+\sqrt{-5})^2 &= (2,1+\sqrt{-5})(2,1+\sqrt{-5})\\
 			&= (4, 2 + 2\sqrt{-5}, -4 + 2\sqrt{-5})\\
 			&= (4, 2+2\sqrt{-5},2\sqrt{-5})\\
 			&= (4,2,2\sqrt{-5}) = (2)
 		\end{align*}
 	
 		Dass $(2,1+\sqrt{-5})$ prim (und sogar maximal) ist, folgt, daraus, dass es der Kern des Ringhomomorphismus
 			\[\varphi:\Z[\sqrt{-5}]\to \F_2, a+b\sqrt{-5} \mapsto \overline{a+b}\]
 		ist, wie man unter Beachtung von $\varphi(\sqrt{-5})^2 = 1 = -5$ in $\F_2$ sofort nachrechnet.
 	\end{beispiel}
 
 	\begin{motivation}\label{2.2.3}\hfill\newline
 		Zahlringe spielen eine wichtige Rolle bei der Untersuchung arithmetischer Eigenschaften von $\Z$. Leider sind sie nicht immer faktoriell. Der Ausweg wird sein, statt Elemente Ideale und statt Primelemente Primideale zu betrachten
 	\end{motivation}
 
 	\begin{definition}\label{2.2.4} [vgl. \ref{2.2.1}(e)]\newline
 		Ein Integrit\"atsring hei\ss t \emph{Dedekindring}\index{Dedekindringe@\textbf{Dedekindringe}}, wenn darin jedes Ideal ein Produkt von Primidealen ist.
 	\end{definition}
 
 	\begin{beispiel}\label{2.2.5}\hfill\newline
 		Jeder Hauptidealring ist ein Dedekindring
 	\end{beispiel}
 
 	\begin{definition}\label{2.2.6}\hfill\newline
 		ei $A$ ein Integrit\"atsring. Ein \emph{gebrochenes Ideal}\index{Dedekindringe@\textbf{Dedekindringe}!Gebrochenes Ideal} von $A$ ist ein $A$-Untermodul von $\qf(A)$ mit $sI \subseteq A$ f\"ur ein $s \in A\setminus\set{0}$.
 		
 		Zyklische [$\to$\ref{1.1.4}(d)] gebrochene Ideale nennt man \emph{gebrochene Hauptideale}\index{Dedekindringe@\textbf{Dedekindringe}!Gebrochene Hauptideale}.
 	\end{definition}
 
 	\begin{bemerkung}\label{2.2.7}\hfill\newline
 		Sei $A$ ein Integrit\"atsring.
 		\begin{enumerate}[(a)]
 			\item Jedes gebrochene Ideal von $A$ ist als $A$-Modul isomorph zu einem Ideal von $A$ (ist n\"amlich $s \in A \setminus \set{0}$ mit $sI \subseteq A$, so $I \xrightarrow{\cong} sI, a\mapsto sa$).
 			\item  Die gebrochenen Ideale von $A$ sind genau die $s^{-1}I$ ($s \in A \setminus \set{0}, I$ ein Ideal von $A$).
 			\item $A$ ist ein Hauptidealring genau dann, wenn jedes gebrochene Ideal von $A$ ein gebrochenes Hauptideal von $A$ ist.
 			\item Jeder endlich erzeugte $A$-Untermodul von $\qf(A)$ ist ein gebrochenes Ideal
 			\item Ist $A$ noethersch, so sind die gebrochenen Ideale von $A$ genau die endlich erzeugten $A$-Untermoduln von $\qf(A)$
 		\end{enumerate}
 	\end{bemerkung}
 
 	\begin{prop-def}\label{2.2.8}\hfill\newline
 		Seien $A$ ein Integrit\"atsring und $I, J$ gebrochene Ideale von $A$. Dann sind auch $I+J$, \linebreak $\displaystyle I \cap J, I \cdot J = \set{\sum_{i}a_ib_i \mid a_i \in I, b_i \in J}$ und f\"ur $J \neq 0$ auch 
 		$I:J := \set{a \in \qf(A)\mid aJ \subseteq I}$ gebrochene Ideale von $A$.
 		
 		\begin{proof}
 			Mit $I$ und $J$ sind auch $I + J, I \cap J$ (trivial), $IJ$ und $I:J$ $A$-Untermoduln von $\qf(A)$. Sind $s,t \in A \setminus \set{0}$ mit $sI \subseteq A$ und $tJ \subseteq A$, so ist auch $st(I+J) \subseteq A, s(I \cap J)\subseteq A$ und $(st)(IJ) \subseteq A$. Ist ferner $J \neq 0$, so gibt es $b \in J \cap (A \setminus \set{0})$ und es gilt f\"ur $a \in I:J$, dass $(sb)a = s(ab) \in s(aJ)\subseteq sI \subseteq A$, also $sb(I:J) \subseteq A$.
 		\end{proof}
 	\end{prop-def}
 
 	\begin{def-prop}\label{2.2.9}\hfill\newline
 		Sei $A$ ein Integrit\"atsring und $I$ ein gebrochenes Ideal von $A$. Dann hei\ss t $I$ \emph{invertierbar}\index{Dedekindringe@\textbf{Dedekindringe}!Invertierbares gebrochenes Ideal}, wenn es ein gebrochenes Ideal $J$ von $A$ gibt mit $IJ = A$. In diesem Fall gilt \linebreak $J=I^{-1}:=A:I$ und $I$ und $J$ sind endlich erzeugt.
 		
 		\begin{proof}
 			Gelte $IJ = A$. Dann $J \subseteq A:I = (A:I)IJ \subseteq AJ = J$, also $J = A:I$. 
 			
 			Schreibe $1 = \sum_i a_ib_i$ mit $a_i \in I, b_i \in J$. Dann gilt 
 			\[I = (\sum_i a_ib_i)I = \sum_i\underbrace{(b_iI)}_{\subseteq A}a_i \subseteq \sum_i Aa_i \subseteq I\]
 			Womit $I = \sum_i Aa_i$ endlich erzeugt ist. Analog f\"ur $J$.
 		\end{proof}
 	\end{def-prop}
 
 	\begin{beispiel}\label{2.2.10}\hfill\newline
 		Jedes gebrochene Hauptideal $\neq 0$ eines Integrit\"atsring ist invertierbar.
 	\end{beispiel}
 
 	\begin{satz}\label{2.2.11} (Eindeutigkeit der Primidealzerlegung invertierbarer Ideale, unter Beachtung von \ref{2.2.10} und \ref{2.2.1}(b) Verallgemeinerung von \ref{2.2.1}(d)) \newline
 		
 		Sei $A$ ein Integrit\"atsring und $I$ ein Ideal von $A$, was als gebrochenes Ideal invertierbar ist. Seien $m, n \in \N_0$ und $\mathfrak{p}_1,\dots, \mathfrak{p}_m,\mathfrak{q}_1,\dots,\mathfrak{q}_n$ Primideale in $A$ mit \linebreak $\mathfrak{p}_1\cdot \cdots \cdot \mathfrak{p}_m = \mathfrak{q}_1 \cdot \cdots \cdot \mathfrak{q}_n = I$. Dann gilt $m = n$ und es gibt $\sigma \in S_n$ mit $\mathfrak{p}_i = \mathfrak{q}_{\sigma(i)}$ f\"ur alle $i \in \set{1,\dots, n}$.
 		
 		\begin{proof}
 			Induktion nach $m$. $m = 0$: trivial.
 			
 			$m-1 \to m (m \in \N)$. Mit $I$ sind alle $\mathfrak{p}_i$ und $\mathfrak{q}_j$ invertierbar. Aus $\mathfrak{q}_1\cdot \cdots \cdot \mathfrak{q}_n \subseteq \mathfrak{p}_1$ folgt, dass es ein $j$ mit $\mathfrak{q}_j \subseteq \mathfrak{p}_1$ gibt (insbesondere $n\geq 1$). denn andernfalls g\"abe es $b_1 \in \mathfrak{q}_1\setminus \mathfrak{p}_1, \dots b_n \in \mathfrak{q}_n\setminus \mathfrak{p}_1$ und $b_1\cdot \cdots \cdot b_n \in \mathfrak{q}_1\cdot \cdots \cdot \mathfrak{q}_n \setminus \mathfrak{p}_1$.
 			
 			Ohne Einschr\"ankung $j=1$. Nun ist $\mathfrak{p}^{-1}\mathfrak{q}_1$ ein Ideal, weil $\mathfrak{p}_1^{-1}\mathfrak{q}_1 \subseteq \mathfrak{p}_1^{-1}\mathfrak{p}_1 = A$, weshalb $\mathfrak{p}_1(\mathfrak{p}_1^{-1}\mathfrak{q}_1) = \mathfrak{q}_1$ impliziert, dass $\mathfrak{p}_1 \subseteq \mathfrak{q}_1$ oder $\mathfrak{p}_1^{-1}\mathfrak{q}_1 \subseteq \mathfrak{q}_1$, aber letzteres ist unm\"oglich, da sonst $A = \mathfrak{p}_1\mathfrak{p}_1^{-1}\mathfrak{q}_1\mathfrak{q}_1^{-1}\subseteq \mathfrak{p}_1\mathfrak{q}_1\mathfrak{q}_1^{-1} = \mathfrak{p}_1$. Also $\mathfrak{q}_1 \subseteq \mathfrak{p}_1 \subseteq \mathfrak{q}_1$ und daher $\mathfrak{p}_1 = \mathfrak{q}_1$. Somit $\mathfrak{p}_2 \cdot \cdots \cdot \mathfrak{p}_n = I\mathfrak{p}_1^{-1} = \mathfrak{q}_2 \cdot \cdots \cdot \mathfrak{q}_m$. Wende nun die Induktionsvoraussetzung an.
 		\end{proof}
 	\end{satz}
 
 	\begin{satz}\label{2.2.12}\hfill\newline
 		Sei $A$ ein Dedekindring. Dann gilt 
 		\begin{enumerate}[(a)]
 			\item Jedes Ideal $\neq (0)$ von $A$ ist invertierbar [$\to$\ref{2.2.9}].
 			\item $A$ ist ganz abgeschlossen
 			\item $A$ ist noethersch
 			\item In $A$ ist jedes Primideal $\neq (0)$ maximal.
 		\end{enumerate}
 	
 		\begin{proof}\hfill\newline
 			{\color{red}
 			\textbf{Behauptung 1}: Jedes invertierbare Primideal von $A$ ist ein maximales Ideal [In (d) wird dies sogar f\"ur alle Primideale $\neq (0)$ behauptet].
 			
 			\smallskip\noindent
 			\textbf{Begr\"undung}: Sei $\mathfrak{p}$ ein invertierbares Primideal von $A$. Wir zeigen, dass $A/\mathfrak{p}$ ein K\"orper ist. Sei hierzu $a \in A$ mit $\overline{a} \neq 0$ in $A/\mathfrak{p}$, das hei\ss t $a \notin \mathfrak{p}$. Zu zeigen ist $\overline{a} \in (A/\mathfrak{p})^*$. Es reicht zu zeigen, dass $I:= aA + \mathfrak{p} = A$. Da $\mathfrak{p}$ invertierbar ist, reich es $a\mathfrak{p} + \mathfrak{p}^2 = \mathfrak{p}$ zu zeigen. Wir zeigen zun\"achst $I^2 = J:= a^2A + \mathfrak{p}$, was wegen $I^2 = a^2A + a\mathfrak{p} + \mathfrak{p}^2$ eine Abschw\"achung der Behauptung ist. Wir wissen $(I/\mathfrak{p})^2 = J/\mathfrak{p}$.
 			
 			Da $A$ ein Dedekindring ist, gibt es $m,n \in \N_0$ und Primideale $\mathfrak{p}_1,\dots, \mathfrak{p}_m$ und $\mathfrak{q}_1,\dots, \mathfrak{q}_n$ von $A$ mit $I = \mathfrak{p}_1 \cdots \mathfrak{p}_m$ und $J = \mathfrak{q}_1 \cdots \mathfrak{q}_n$. Es gilt $\mathfrak{p} \subseteq I \subseteq \mathfrak{p}_i$ f\"ur alle $i \in \set{1,\dots, m}$ und $\mathfrak{p} \subseteq J \subseteq \mathfrak{q}_j$ f\"ur alle $j \in\set{1,\dots,n}$. 
 			
 			Da die Primideale in $A/\mathfrak{p}$ den Primidealen von $A$ entsprechen, die $\mathfrak{p}$ enthalten, erhalten wir im Integrit\"atsring $A/\mathfrak{p}$ die Primidealzerlegungen
 				\[I/\mathfrak{p} = (\mathfrak{p}_1/\mathfrak{p}) \cdots (\mathfrak{p}_m/\mathfrak{p})\]
 			und 
 				\[J/\mathfrak{p} = (\mathfrak{q}_1/\mathfrak{p}) \cdots (\mathfrak{q}_n/\mathfrak{p})\]
 			
 			Es folgt $(\mathfrak{p}_1/\mathfrak{p})^2 \cdots (\mathfrak{p}_m/\mathfrak{p})^2 = (I/\mathfrak{p})^2 = J/\mathfrak{p} = (\mathfrak{q}_1/\mathfrak{p})\cdots (\mathfrak{q}_n/\mathfrak{p})$
 			
 			Da $J/\mathfrak{p} = (\overline{a}^2) \neq 0$ als Hauptideal nach \ref{2.2.10} invertierbar ist, folgt mit \ref{2.2.11} ohne Einschr\"ankung $(\mathfrak{p}_1/\mathfrak{p}, \mathfrak{p}_1/\mathfrak{p}, \dots, \mathfrak{p}_m/\mathfrak{p}, \mathfrak{p}_m/\mathfrak{p}) = (\mathfrak{q}_1/\mathfrak{p},\dots, \mathfrak{q}_n/\mathfrak{p})$  und daher \linebreak
 			$(\mathfrak{p}_1,\mathfrak{p}_1,\dots,\mathfrak{p}_m,\mathfrak{p}_m) = (\mathfrak{q}_1,\dots,\mathfrak{q}_n)$, also $I^2 = J$.
 			
 			Um schlie\ss lich $a\mathfrak{p} + \mathfrak{p}^2 = \mathfrak{p}$, sei $b \in \mathfrak{p}$. Zu zeigen ist $b \in a\mathfrak{p} + \mathfrak{p}^2$. Wegen $b \in J$ gilt $b \in I^2 = a^2A + a\mathfrak{p}+ \mathfrak{p}^2$. Schreibe $b = a^2c + b'$ mit $c \in A, b' \in a\mathfrak{p}+\mathfrak{p}^2$. Dann $a^2c = b - b' \in \mathfrak{p}$ und daher $c \in \mathfrak{p}$. Somit auch $b \in a \mathfrak{p} + \mathfrak{p}^2$
 			
 			\medskip\noindent 
 			\textbf{Behauptung 2}: Jedes Primideal $\neq (0)$ von $A$ ist invertierbar [In (a) wird dies sogar f\"ur alle Ideale $\neq 0$ behauptet]
 			
 			\smallskip\noindent 
 			\textbf{Begr\"undung}: Sei $\mathfrak{p} \neq (0)$ ein Primideal von $A$. W\"ahle ein $a \in \mathfrak{p}\setminus \set{0}$. Schreibe $aA = \mathfrak{p}_1 \cdots \mathfrak{p}_n$ mit Primidealen $\mathfrak{p}_1,\dots, \mathfrak{p}_n$ von $A$. Da $aA$ invertierbar ist [$\to$\ref{2.2.10}] ist es auch jedes $\mathfrak{p}_i$. Wegen $\mathfrak{p}_1\cdots \mathfrak{p}_n \subseteq \mathfrak{p}$ ist aber auch $\mathfrak{p}$ eines dieser $\mathfrak{p}_i$, denn es gibt ein $i$ mit $\mathfrak{p}_i \subseteq \mathfrak{p}$ und $\mathfrak{p}_i$ ist maximal nach Behauptung 1.
 			}
 			
 			
 			\medskip\noindent Wegen der Existenz der Primfaktorzerlegung in $A$, folgt aus Behauptung 2 sofort (a). Damit ist Behauptung 1 gleichbedeutend mit (d). 
 			
 			Aus (a) folgt (c) mit \ref{2.2.9}. Um schlie\ss lich (b) zu zeigen, sei $a \in \qf(A)$ ganz \"uber $A$. Dann ist $A[a]$ ein endlich erzeugter $A$-Modul [$\to$\ref{2.1.5}] und damit ein gebrochenes Ideal von $A$ [$\to$ \ref{2.2.7}(d)], das nach (a) invertierbar ist. 
 			
 			Aus $A[a]^2 \subseteq A[a]$ folgt daher $A[a] \subseteq A$ und daher $a \in A$ wie gew\"unscht.
   		\end{proof}
   	 \end{satz}
   	\newpage
   		
 	\section{Charakterisierung von Dedekindringen}\thispagestyle{sectionstart}
 	\begin{lemma}\label{2.3.1}
 		In einem noetherschen Integrit\"atsring enth\"alt jedes Ideal $\neq (0)$ ein Produkt von Primidealen $\neq (0)$.
 			
 		\begin{proof}
 			Sei $A$ ein Integrit\"atsring und $I \neq (0)$ ein Ideal von $A$, welches kein Produkt von Primdiealen $\neq (0)$ enth\"alt. Es reicht ein Ideal $J \supsetneq I$ von $A$ zu finden, welches auch kein Produkt von Primidealen $\neq 0$ enth\"alt. Da $I$ weder ein Primideal von ganz $A$ ist, gibt es $a,b \in A\setminus I$ mit $ab \in I$. Dann sind $J := I + (a)$ und $K:= I+(b)$ Ideale von $A$ mit $J \supsetneq I$ und $K \supsetneq I$ und $JK \subseteq I$. Es k\"onnen nicht sowohl $J$ als auch $K$ ein Produkt von Primidealen $\neq 0$ enthalten.
 		\end{proof} 
 	\end{lemma}
   	
   	\begin{satz}\label{2.3.2}
   		Sei $A$ ein Ring. Dann ist $A$ ein Dedekindring genau dann, wenn $A$ ein noetherscher, ganz abgeschlossener Integrit\"atsring ist, indem alle Primideale $\neq (0)$ maximal sind.
   			
   		\begin{proof}
   			Ist $A$ ein Dedekindring, so besitzt $A$ die gew\"unschten Eigenschaften nach \ref{2.2.12}. Sei umgekehrt $A$ ein noetherscher, ganz abgeschlossener Integrit\"atsring, in dem alle Primideale $\neq (0)$ maximal sind.
   				
   			\emph{Behauptung 1: Seien $I$ und $J$ gebrochene Ideale von $A$ mit $J \neq (0)$ und $IJ \subseteq J$. Dann $I \subseteq A$.}
   				
   			\emph{Begr\"undung}: Sei $x \in I$. Zu zeigen ist $x \in A$. Nach \ref{2.1.5} reicht es zu zeigen, dass $A[x]$ ein endlich erzeugter $A$-Modul ist. 
   				
   			Durch Induktion zeigt man $\forall n \in \N_0: x^nJ \subseteq J$  (In der Tat $x^{n-1}J \subseteq J$ f\"ur ein $n\in \N$, so $x^nJ = x(x^{n-1}J) \subseteq xJ \subseteq IJ \subseteq J$)
   				
   			Es folgt $A[x]J \subseteq J$. W\"ahlt man $y \in J \setminus \set{0}$, so folgt also $A[x]y \subseteq J$. Da $A$ noethersch ist, ist $J$ ein endlich erzeugter $A$-Modul [$\to$\ref{2.2.7}(e)] und daher selber noethersch [$\to$\ref{1.4.7}], womit $A[x]y \cong A[x]$ ein endlich erzeugter $A$-Modul ist.
   				
   			\vspace*{2mm}
   			\emph{Behauptung 2: Sei $\mathfrak{p} \neq (0)$ ein Primideal von $A$. Dann ist $\mathfrak{p}$ invertierbar.}
   		\end{proof}
   	\end{satz}
   	\newpage
   	
   	\section{Norm, Spur und Diskriminante}\thispagestyle{sectionstart}
   	\begin{definition}\label{2.4.1}
   		Sei $L|K$ eine endlich K\"orpererweiterung und $a \in L$. Dann ist $\varphi_a:L \to L, x \mapsto ax$ ein Endomorphismus des $K$-Vektorraumes $L$.
   		
   		Ist $\underline{v} = (v_1, \dots, v_n)$ eine beliebige Basis des $K$-Vektorraumes $L$ (insbesondere $n = [L:K]$) und $A = (a_{ij})_{1\leq i,j \leq n} = M(\varphi_a, \underline{v})$ die Darstellungsmatrix von $\varphi_a$ bez\"uglich $\underline{v}$ (also $av_j = \sum_{i=1}^{n} a_{ij}v_i$ f\"ur alle $j \in \set{1,\dots, n}$), so hei\ss t $\chi_{L|K}(a) := \chi_{\varphi(a)} = \det (A-XI_n) \in K[X]$ das \emph{charakteristische Polynom}\index{Norm, Spur, Diskriminante@\textbf{Norm, Spur, Diskriminante}!Charakteristisches Polynom},
   		$N_{L|K}(a) := \det(A) = \chi_{L|K}(0)$ die \emph{Norm}\index{Norm, Spur, Diskriminante@\textbf{Norm, Spur, Diskriminante}!Norm} und $\tr_{L|K}(a) := \tr A = \sum_{i=1}^n a_{ii} = (-1)^{n-1}\glqq\textrm{Koeffizient von } X^{n-1} \textrm{ in } \chi_{L|K}(a)\grqq$ die \emph{Spur}\index{Norm, Spur, Diskriminante@\textbf{Norm, Spur, Diskriminante}!Spur} von $A$ bez\"uglich $L|K$ [diese Begriffe h\"angen nicht von der Wahl der Basis $\underline{v}$ ab].
   	\end{definition}
   	
   	\begin{beispiel}\label{2.4.2}
   		Sei $d \in \nsquare$ [$\to$ \ref{2.1.16}]. Seien $a,b \in \Q$ und $x = a + b\sqrt{d} \in \Q(d)$.
   		
   		Betrachte Basis $\underline{v}= (1,\sqrt{d})$ von $\Q(\sqrt{d})$ [$\to$\ref{2.1.17}]. Dann $x\cdot 1 = a + b\sqrt{d}$ und $x \sqrt{d} = bd + a\sqrt{d}$.
   		
   		Setzt man also $A = \begin{pmatrix}
   			a&bd\\b&a
   		\end{pmatrix} \in \Q^{2\times 2}$, so $\chi_{\Q(\sqrt{d})|\Q}(x) = \det \begin{pmatrix}
   		a - X&bd\\b&a - X
   	\end{pmatrix} = (a-X)^2 - b^2d = X^2 - 2aX + (a^2-b^2d)$ und daher $N_{Q(\sqrt{d})|\Q}(x) = a^2-b^2d$ und $\tr_{Q(\sqrt{d})|\Q}(x) = 2a$
   	\end{beispiel}

	\begin{proposition}\label{2.4.3}
		Sei $L|K$ eine endliche K\"orpererweiterung. Dann ist $\tr_{L|K}:L \to K$ $K$-linear und es gilt $N_{L|K}(ab) = N_{L|K}(a)N_{L|K}(b)$ f\"ur alle $a,b \in L$.
		
		Es ist $N_{K|L}\big|_{L^*} \to K^*$ ein Gruppenhomomorphismus.
		
		\begin{proof}
			W\"ahle eine Basis $\underline{v} = (v_1, \dots, v_n)$ des $K$-Vektorraumes $L$. Sind $a,b \in L$ und $c \in K$, so gilt 
			\begin{align*}
				\tr_{L|K}(a+b) &= \tr M(\varphi_{a+b}, \underline{v})\\
				&= \tr M(\varphi_a + \varphi_b, \underline{v})\\
				&= \tr (M(\varphi_a, \underline{v}) + M(\varphi_b, \underline{v}))\\
				&= \tr M(\varphi_a, \underline{v}) + \tr M(\varphi_b, \underline{v})\\
				&= \tr_{L|K}(a) + \tr_{L|K}(b)
			\end{align*}
			und 
			\begin{align*}
				\tr_{L|K}(ca) &= \tr M(\varphi_{ca}, \underline{v})\\
				&= \tr M(c\varphi_{a}, \underline{v})\\
				&= \tr (cM(\varphi_{a}, \underline{v}))\\
				&= c \tr M(\varphi_{a}, \underline{v})\\
				&= c \tr_{L|K}(a)
			\end{align*}
			
			\begin{align*}
				N_{L|K}(ab) &= \det M(\varphi_{ab}, \underline{v})\\
				&=\det M(\varphi_{a} \circ \varphi_b, \underline{v})\\
				&=\det (M(\varphi_{a}, \underline{v}) M(\varphi_b, \underline{v}))\\
				&=\det M(\varphi_{a}, \underline{v}) \det M(\varphi_b, \underline{v})\\
				&= N_{L|K}(a)N_{L|K}(b)
			\end{align*}
		
			\begin{align*}
				N_{L|K}(a) = 0 &\Leftrightarrow \det M(\varphi_{a}, \underline{v}) = 0\\
				&\Leftrightarrow \varphi_{a} \textrm{ nicht bijektiv}\\
				&\Leftrightarrow a = 0
			\end{align*}
			(denn $a\neq 0 \Rightarrow \varphi_{a} \circ \varphi_{a^{-1}} = \id_L = \varphi_{a^{-1}} \circ \varphi_{a}$).
		\end{proof} 
	\end{proposition}

	\begin{proposition}\label{2.4.4}
		Sei $F$ ein Zwischenk\"orper der endlichen K\"orpererweiterung $L|K$. Dann gilt f\"ur alle $a \in F$
			\[\chi_{L|K}(a) = (\chi_{F|K}(a))^{[L:F]}\]
			\[N_{L|K}(a) = (N_{F|K}(a))^{[L:F]}\]
			\[\tr_{L|K}(a) = [L:F]\tr_{F|K}(a)\]
			
		\begin{proof}
			W\"ahle eine Basis $\underline{u} = (u_1, \dots, u_m)$ des $K$-Vektorraumes $F$ und eine Basis $\underline{v} = (v_1, \dots, v_n)$ des $F$-Vektorraumes $L$.
			
			Dann ist $\underline{w} := (u_1v_1, \dots, u_mv_1, \dots, u_mv_n)$ eine Basis des $K$-Vektorraumes $L$.
			
			F\"ur alle $a \in F$ gilt nun
				\[M(\varphi_{a}, \underline{w}) = \begin{pmatrix}
					M(\psi_a, \underline{u}) & & & \\
					& M(\psi_a, \underline{u}) & & \\
					& & \ddots & \\
					& & & M(\psi_a, \underline{u})
				\end{pmatrix}\]
			
			wobei $\varphi_a: L \to L, x \mapsto ax$ und $\psi: F \to F, x \mapsto ax$
		\end{proof}
	\end{proposition}
	
	\begin{proposition}\label{2.4.5}
		Sei $L|K$ eine endliche K\"orpererweiterung und $a \in L$. Dann $\chi_{L|K}(a) = (-a)^{[L:K]}(\irr_K(a))^{[L:K(a)]}$
		
		\begin{proof}
			Nach \ref{2.4.4} gen\"ugt es $\chi_{K(a)|K}(a) = (-1)^{K(a)|K} \irr_K(a)$ zu zeigen.
			
			Wegen $\deg \chi_{K(a)|K}(a) = [K(a):K] = \deg \irr_K(a)$ reicht es $\irr_K(a)|\chi_{K(a)|K}(a)$ zu zeigen, was aber aus Cayley-Hamilton folgt.
		\end{proof}
	\end{proposition}
	
	\begin{satz}\label{2.4.6}
		Sei $L|K$ eine endliche separable K\"orpererweiterung und seien $\varphi_1, \dots, \varphi_n$ die verschiedenen $K$-Einbettungen ($K$-Homomorphismen) von $L$ in einen festen algebraischen Abschluss $\overline{K}$ von $K$ (insbesondere $[L:K] = [L:K]_s = n$).
		
		Dann gilt f\"ur alle $a \in L$
			\[\chi_{L|K}(a) = \prod_{i=1}^n (\varphi_i(a) -X)\]
		und daher $N_{L|K}(a) = \prod_{i=1}^n \varphi_i(a)$ sowie $\tr_{L|K}(a) = \sum_{i=1}^{n}\varphi_i(a)$.
		
		\begin{proof}
			Sei $a \in L$. Jeder K\"orperhomomorphismus $K(a) \to \overline{K}$ l\"asst sich zu genau $[L:K(a)]_s = [L:K(a)]$ K\"orperhomomorphismen $L \to \overline{K}$ fortsetzen.
			Daher kann man die $\varphi_i$ so neu indizieren, dass $\varphi_{ij}\ (1 \leq i \leq l, 1 \leq j \leq m)$ die verschiedenen $K$-Homomorphismen $L \to \overline{K}$ sind mit $\varphi_{ij}\big|_{K(a)} = \varphi_{st}\big|_{K(a)} \Leftrightarrow i = s\ (1 \leq i,s\leq l, 1 \leq j, t \leq m)$.
			
			Hierbei gilt $l = [K(a):K]_s = [K(a):K]$ und 
			$m = [L:K(a)]_s = [L:K(a)]$.
			
			Zu zeigen ist 
				\[\chi_{L|K}(a) = \prod_{i=1}^{l}\prod_{j=1}^m(\varphi_{ij}(a)-X)\]
			Das hei\ss t
				\[\chi_{L|K}(a) = \brac{\prod_{i=1}^{l}(\varphi_{i1}(a)-X)}^m\]
				
			Nach \ref{2.4.5} reich es $\irr_K(a) = \prod_{i=1}^l (X-\varphi_{i1}(a))$ zu zeigen.
			
			Dies folgt daraus, dass mit den $\varphi_{i1}\big|_{K(a)}$ auch die $\varphi_{i1}\ (1\leq i \leq l)$ verschieden sind und die letzteren alle Nullstellen des Polynoms $\irr_K(a)$ sind, welches Grad $l$ hat.
		\end{proof}
	\end{satz}
	
	\begin{erinnerung}\label{2.4.7}
		\begin{enumerate}[(a)]
			\item Sei $K$ ein K\"orper der Charakteristik $p \in \set{0} \cup \P$ und $f \in K[X]$ irreduzibel. Dann gibt es genau ein Paar $(n,g)$ mit $n \in \N_0, g \in K[X]$ irreduzibel und separabel und $f = g(X^{p^n})$
			
			\item Sei $F$ ein Zwischenk\"orper der algebraischen K\"orpererweiterung $L|K$. Dann gilt $[L:K]=[L:F][F:K]$ und 
			$[L:K]_s = [L:F]_s[F:K]_s$, wobei man $n\cdot \infty := \infty \cdot n := \infty \cdot \infty := \infty$ setzt f\"ur $n \in \N$
			
			\item Ist $K$ ein K\"orper der Charakteristik $0$, so ist $K$ vollkommen, das hei\ss t, jede algebraische K\"orpererweiterung $L|K$ ist separabel.
			
			\item Sei $L|K$ eine algebraische K\"orpererweiterung. Dann ist der separable Abschluss von $K$ in $L$ $\overline{k^{s_L}} = \set{a \in L| a \textrm{ separabel \"uber } K}$ ein
			Zwischenk\"orper von $L|K$ mit $[L:K]_s = [\overline{K^{s_L}}:K]$
			
			\item Ist $R$ ein kommutativer Ring mit $p:= \Char R \in \P$, so ist 
				\[\Phi_R: R  \to R, a \mapsto a^p\]
			ein Endomorphismus ("Frobeniushomomorphismus")
		\end{enumerate}
	\end{erinnerung}

	\begin{definition}\label{2.4.8}
		Eine algebraische K\"orpererweiterung $L|K$ hei\ss t \emph{rein inseparabel}\index{Norm, Spur, Diskriminante@\textbf{Norm, Spur, Diskriminante}!Rein Inseparabel}, wenn kein $a \in L \setminus K$ separabel \"uber $K$ ist.
	\end{definition}

	\begin{proposition}\label{2.4.9}
		Sei $L|K$ eine algebraische K\"orpererweiterung mit $p:= \Char K > 0$. Dann sind \"aquivalent
		\begin{enumerate}[(a)]
			\item $L|K$ ist rein inseparabel
			\item $\forall x \in L: \exists n \in \N_0: x^{p^n} \in K$
			\item $\forall x \in L: \exists n \in \N_0: \exists a \in K: \irr_K(x) = X^{p^n} - a$
		\end{enumerate}
	
		\begin{proof}
			\"Ubung
		\end{proof}
	\end{proposition}

	\begin{beispiel}\label{2.4.10}
		\begin{enumerate}[(a)]
			\item 
			Ist $L|K$ eine algebraische K\"orpererweiterung, so ist wegen der Transitivit\"at der Separabilit\"at $L|\overline{K^{s_L}}$ rein inseparabel.
			
			\item 
			Jede Teilerweiterung einer rein inseparablen K\"orpererweiterung ist wegen \ref{2.4.9}(b) wieder rein inseparabel
			
			\item 
			Ist $K$ ein K\"orper mit $p:= \Char K > 0$ und ist $n \in \N_0$, so ist $K(X)|K(X^{p^n})$ rein inseparabel, wie man mit \ref{2.4.9}(b) und \ref{2.4.7}(e) leicht sieht.
		\end{enumerate}
	\end{beispiel}

	\begin{definition}\label{2.4.11}
		Sei $L|K$ eine endliche K\"orpererweiterung. Dann hei\ss t
			$[L:K]_i := \frac{[L:K]}{[L:K]_s} \overset{\ref{2.4.7}(d)}{=} \frac{[L:K]}{[\overline{K^{s_L}}:K]} = \overset{\ref{2.4.7}(b)}{=} [L:\overline{K^{s_L}}]$
			der \emph{Inseparabilit\"atsgrad}\index{Norm, Spur, Diskriminante@\textbf{Norm, Spur, Diskriminante}!Inseparabilit\"atsgrad} von $L|K$.
	\end{definition}

	\begin{proposition}\label{2.4.12}
		Sei $F$ ein Zwischenk\"orper der endlichen K\"orpererweiterung $L|K$. Dann $[L:K]_i = [L:F]_i[F:K]_i$.
		
		\begin{proof}
			\begin{flalign*}
				[L:K]_i &= \frac{[L:K]}{[L:K]_s}\\
				&= \frac{[L:F][F:K]}{[L:F]_s[F:K]_s} && \ref{2.4.7}(d)\\
				&= [L:F]_i[F:K]_i			
			\end{flalign*}
		\end{proof}
	\end{proposition}

	\begin{korollar}\label{2.4.13}
		Sei $L|K$ eine endliche K\"orpererweiterung und $p:= \Char K$. Dann gibt es $n \in \N_0$ mit $[L:K]_i = p^n$.
		
		\begin{proof}
			Benutze \ref{2.4.12} und \ref{2.4.9}(c)
		\end{proof}
	\end{korollar}

	\begin{satz}\label{2.4.14}
		(vergleiche \ref{2.4.6})\newline
		Sei $L|K$ eine endliche K\"orpererweiterung und seien $\varphi_1, \dots, \varphi_n$ die verschiedenen $K$-Einbettungen von $L$ in einen festen algebraischen Abschluss $\overline{K}$ von $K$.
		
		Dann 
			\[N_{L|K}(a) = \brac{\prod_{i=1}^n \varphi_i(a)}^{[L:K]_i} \]
		und
			\[\tr_{L|K}(a) = [L:K]_i \sum_{i=1}^n\varphi_i(a)\]
		f\"ur alle $a \in L$.
		
		\begin{proof}
			Wegen $[L: \overline{K^{s_L}}] \overset{\ref{2.4.7}(b)}{=} \frac{[L:K]_s}{[\overline{K^{s_L}}:K]_s} \overset{\ref{2.4.7}(a)}{=} \frac{[L:K]_s}{[L:K]_s} = 1$ sind 
			$\varphi_1\big|_{\overline{K^{s_L}}}, \dots, \varphi_n\big|_{\overline{K^{s_L}}}$ verschiedene $K$-Einbettungen von $\overline{K^{s_L}}$ in $\overline{K}$.
			
			Weil $[L:K]_i = [L:\overline{K^{s_L}}]$gilt und $\overline{K^{s_L}}:K$ separabel ist, folgen die behaupteten Gleichungen f\"ur alle $a \in \overline{K^{s_L}}$ mit \ref{2.4.4} und \ref{2.4.6}.
			
			Sei also nun $a \in L \setminus \overline{K^{s_L}}$. Dann gibt es nach \ref{2.4.7}(a) ein $m \in \N$ und $g \in K[X]$ irreduzibel und separabel mit $\irr_K(a) = g(X^{p^m})$, wobei $p:= \Char K \overset{\ref{2.4.7}(c)}{\in} \P$. Mit \ref{2.4.5}folgt 
				\[\chi_{L|K}(a) = (-1)^{[L:K]}\brac{g(X^{p^m})}^{[L:K(a)]} \]
			woraus sicher $\tr_{L|K}(a) = 0$ folgt (beachte $m \geq 1$), was wegen $[L:K]_i \overset{\ref{2.4.13}}{\cong}_{(p)} 0$ die Gleichung mit der Spur zeigt.
			
			Wegen \ref{2.4.7}(e) reicht es f\"ur die Gleichung mit der Norm zu zeigen, dass
				\[\Phi_p^m(N_{L|K}(a)) = \Phi_p^m\brac{\brac{\prod_{i=1}^m \varphi_i(a)}^{[L:K]_i} } \],
			das hei\ss t 
				\[N_{L|K}(a^{p^m}) = \Phi_p^m\brac{\brac{\prod_{i=1}^m \varphi_i(a)}^{[L:K]_i} }\]
			, was aber aus $a^{p^m} \in \overline{K^{s_L}}$ folgt (siehe oben).
		\end{proof}
	\end{satz}

	\begin{satz}\label{2.4.15}
		("Schachtelungsformel f\"ur Norm und Spur")\newline
		Sei $F$ ein Zwischenk\"orper der endlichen K\"orpererweiterung $L|K$. Dann
			\begin{align*}
				N_{L|K} &= N_{F|K} \circ N_{L|F}\\
				\tr_{L|K} &= \tr_{F|K} \circ \tr_{L|F}
			\end{align*}
		\begin{proof}
			\"Ubung.
		\end{proof}
	\end{satz}

	\begin{erinnerung}\label{2.4.16}
		Sei $V$ ein $K$-Vektorraum mit Basis $\underline{v} := (v_1, \dots, v_n)$
		\begin{enumerate}[(a)]
			\item 
			$\underline{v}^* := (v_1^*, \dots, v_n^*)$ definiert durch $v_i^*(v_j) := \delta_{ij} = \begin{cases}
				0 & \textrm{falls } i \neq j\\
				1 & \textrm{falls } i = j
			\end{cases}$
			($i, j \in \set{1, \dot, n}$) ist eine Basis des Dualraums $V^* = \Hom(V, K)$ [$\to$ \ref{1.6.8}] genannt die zu $\underline{v}$ duale Basis.
			
			\item $b:V \times V \to K$ hei\ss t eine Bilinearform auf $V$, wenn f\"ur alle $v \in V$ sowohl $b(\cdot, v) : V \to K, w \mapsto b(w,v)$ als auch 
			$b(v, \cdot): V \to K, w \mapsto b(v,w)$ linear sind. Es hei\ss t $b$ symmetrisch, wenn $b(v, w) = b(w,v)$ f\"ur alle $v,w \in V$.
			
			\item Sei $b$ eine Bilinearform auf $V$. Dann sind 
			$\overset{\leftarrow}{b}: V \to V^*, v \mapsto b(\cdot, v)$ und 
			$\overset{\rightarrow}{b}: V \to V^*, v \mapsto b(v, \cdot)$ linear und es gilt 
			\[M(b,\underline{v}) := (b(v_i, v_j))_{1 \leq i,j \leq n} = M(\overset{\leftarrow}{b}, \underline{v}, \underline{v}^*) = M(\overset{\rightarrow}{b}, \underline{v}, \underline{v}^*)^T\]
			denn $\overset{\leftarrow}{b}(v_j) = \sum_{i=1}^n b(v_i, v_j)v_i^*$ und $\overset{\rightarrow}{b}(v_j) = \sum_{i=1}^n b(v_j, v_i)v_i^*$ f\"ur $j \in \set{1, \dots, n}$,
			was man durch Auswerten in $v_k$ ($k \in \set{1, \dots, n}$) sofort sieht. Es gilt daher:
			\begin{align*}
				b \textrm{ nicht ausgeartet} &:\Leftrightarrow \overset{\leftarrow}{b} \textrm{ injektiv}\\
				&\Leftrightarrow \overset{\leftarrow}{b} \textrm{ surjektiv}\\
				&\Leftrightarrow \overset{\leftarrow}{b} \textrm{ Isomorphismus}\\
				&\Leftrightarrow \overset{\rightarrow}{b} \textrm{ injektiv}\\
				&\Leftrightarrow \overset{\rightarrow}{b} \textrm{ surjektiv}\\
				&\Leftrightarrow \overset{\rightarrow}{b} \textrm{ Isomorphismus}\\
				&\Leftrightarrow \det(M(b, \underline{v})) \neq 0
			\end{align*}
		
			\item Sei $b$ eine Bilinearform auf $V$ und $\underline{w} := (w_1, \dots, w_n)$ eine weitere Basis von $V$. Dann gilt 
				\[M(\underline{w}^*, \underline{v}^*) = M(\underline{v}, \underline{w})^T\]
			denn ist $M(\underline{v}, \underline{w}) = (a_{ij})_{1 \leq i,j \leq n}$, so $v_j = \sum_{i=1}^{n} a_{ij}w_i$ und daher
			$w_j^* = \sum_{i=1}^{n} a_{ji}v_i^*$ f\"ur $j \in \set{1, \dots, n}$, was man durch Auswerten in $v_k$ ($k \in \set{1, \dots, n}$) sieht. 
			
			Daher gilt
			\begin{flalign*}
				M(b, \underline{v}) &= M(\overset{\leftarrow}{b}, \underline{v}, \underline{v}^*) && (c)\\
				&= M(\underline{w}^*, \underline{v}^*)M(\overset{\leftarrow}{b}, \underline{w}, \underline{w}^*)M(\underline{v}, \underline{w})\\
				&= M(\underline{v}, \underline{w})^TM(\overset{\leftarrow}{b}, \underline{w}, \underline{w}^*)M(\underline{v}, \underline{w}) && (c)
			\end{flalign*}
			und $\det(M(b, \underline{v})) = (\det(M(\underline{v}, \underline{w})))^2 \det(M(b, \underline{w})) $
		\end{enumerate}
	\end{erinnerung}

	\begin{prop-def}\label{2.4.17}
		Sei $V$ ein $K$-Vektorraum mit Basis $\underline{v} := (v_1, \dots, v_n)$ und $b: V \times V \to K$ eine nicht ausgeartete Bilinearform. 
		
		Dann gibt es genau ein Tupel $\underline{w} = (w_1, \dots, w_n) \in V^n$ mit $b(v_i, w_j) = \delta_{ij}$ f\"ur alle $i,j \in \set{1,\dots, n}$.
		
		Es ist $\underline{w}$ eine Basis von $V$, genannt die zu $\underline{v}$ bez\"uglich $b$ duale Basis.
		
		\begin{proof}
			F\"ur alle $w_1, \dots, w_n \in V$ gilt 
			\begin{align*}
				\forall i,j: b(v_i, w_j) = \delta_{ij} &\Leftrightarrow \forall i,j: b(v_i, w_j) = v_j^*(v_i)\\
				&\Leftrightarrow \forall j: \overset{\leftarrow}{b}(w_j) = v_j^*\\
				&\Leftrightarrow \forall j: w_j = \brac{\overset{\leftarrow}{b}}^{-1}(v_j^*)
			\end{align*}
		und $\overset{\leftarrow}{b}$ ist ein Isomorphismus.
		\end{proof}  
	\end{prop-def}

	\begin{sprechweise}\label{2.4.18}
		Sei $L|K$ eine endliche K\"orpererweiterung. Dann ist $L \times L \to K, (x,y)\mapsto \tr_{L|K}(xy)$ eine symmetrische Bilinearform auf dem $K$-Vektorraum $L$, genannt die \emph{Spurform}\index{Norm, Spur, Diskriminante@\textbf{Norm, Spur, Diskriminante}!Spurform} von $L|K$
	\end{sprechweise}

	\begin{definition}\label{2.4.19}
		Sei $L|K$ eine K\"orpererweiterung mit $n:= [L:K] < \infty$. F\"ur alle $x_1, \dots, x_n \in L$ hei\ss t 
			\[d_{L|K}(x_1,\dots, x_n) := \det((\tr_{L|K}(x_ix_j))_{1\leq i,j \leq n} )\]
		die \emph{Diskriminante}\index{Norm, Spur, Diskriminante@\textbf{Norm, Spur, Diskriminante}!Diskriminante} von $(x_1, \dots, x_n)$ bez\"uglich $L|K$.
	\end{definition}

	\begin{bemerkung}\label{2.4.20}
		Sei $L|K$ eine K\"orpererweiterung mit $n:=[L:K] < \infty$.
		
		Dann $d_{L|K}(x_1, \dots, x_n) = d_{L|K}(x_{\sigma(1)}, \dots, x_{\sigma(n)})$ f\"ur alle $x_1, \dots, x_n \in L$ und $\sigma \in S_n$, da jede Permutation ein Produkt von Transpositionen ist und eine simultane Zeilen- und Spaltenvertauschung die Determinante nicht \"ander.
	\end{bemerkung}

	\begin{proposition}\label{2.4.21}
		Sei $L|K$ endlich und separabel. Seien $\varphi_1, \dots, \varphi_n$ die verschiedenen $K$-Einbettungen von $L$ in $\overline{K}$. 
		
		Dann gilt f\"ur alle $x_1, \dots, x_n \in L$
			\[d_{x_1, \dots, x_n} = \brac{\det\begin{pmatrix}
					\varphi_1(x_1) & \cdots & \varphi_1(x_n)\\
					\vdots & \ddots & \vdots\\
					\varphi_n(x_1) & \cdots & \varphi_n(x_n)
			\end{pmatrix}}^2\]
		
		\begin{proof}
			F\"ur $x_1, \dots, x_n \in L$ gilt 
			\begin{flalign*}
				d_{L|K}(x_1, \dots, x_n) &= \det((\tr_{L|K}(x_ix_j)_{1 \leq i,j\leq n} )\\
				&= \det\brac{\brac{\sum_{k=1}^{n} \varphi_k(x_ix_j) }_{1\leq i,j \leq n}} && \ref{2.4.6}\\
				&= \det \brac{(\varphi_k(x_i))_{1\leq i,k\leq n} \cdot (\varphi_k(x_j))_{1\leq k,j\leq n} }\\
				&= \det ((\varphi_k(x_i))_{1 \leq k,i \leq n})^2
			\end{flalign*}
		\end{proof}
	\end{proposition}

	\begin{satz}\label{2.4.22}
		Sei $L|K$ endlich und separabel und $a \in L$ mit $L=K(a)$. Seien $\varphi_1, \dots, \varphi_n$ die verschiedenen $K$-Einbettungen von $L$ in $\overline{K}$. Bezeichne $f$ das Minimalpolynom von $a$ \"uber $K$ und $f'$ seine formale Ableitung.
		
		Dann gilt 
			\[N_{L|K}(f'(a)) = \prod_{i,j=1; i \neq j}^n (\varphi_i(a)- \varphi_j(a)) = (-1)^{\frac{n(n-1)}{2}}d_{L|K}(1,a,\dots, a^{n-1}) \]
			
		\begin{proof}
			\"Ubung
		\end{proof} 
	\end{satz}

	\begin{korollar}\label{2.4.23}
		Sei $L|K$ eine endliche K\"orpererweiterung und $\underline{v} =(v_1, \dots, v_n)$ eine Basis des $K$-Vektorraumes $L$.
		
		Dann sind \"aquivalent:
		
		\begin{enumerate}[(a)]
			\item $L|K$ ist separabel
			\item Sie Spurform [$\to$ \ref{2.4.18}] von $L|K$ ist nicht ausgeartet
			\item $d_{L|K}(v_1, \dots, v_n) \neq 0$
			\item $\tr_{L|K} \neq 0$
		\end{enumerate}
	
		\begin{proof}
			$(a)\Rightarrow (b)$: Gelta (a). W\"ahle mit dem Satz vom primitiven Element ein $a \in L$ mit $L= K(a)$. Sind $\varphi_1, \dots, \varphi_n$ die verschiedenen $K$-Homomorphismen $K \to \overline{K}$ so 
			\begin{flalign*}
				d_{L|K}(1, \dots, a^{n-1}) &= (-1)^{\frac{n(n-1)}{2}} \prod_{i,j=1; i \neq j}^{n} (\varphi_i(a) - \varphi_j(a)) \neq 0&& \ref{2.4.22}
			\end{flalign*}
		
			Es ist $\underline{w} := (1, a, \dots, a^{n-1})$ eine Basis des $K$-Vektorraumes $L$ und daher $d_{L|K}(1, a, \dots, a^{n-1})$ die Determinante der Darstellungsmatrix der Spurform bez\"uglich $\underline{w}$. Nach \ref{2.4.16} folgt (b)
			
			
			$(b)\Rightarrow (c)$: wieder mit \ref{2.4.16}
			
			$(c)\Rightarrow (d)$: trivial
			
			$(d) \Rightarrow (a)$: mit \ref{2.4.14}
		\end{proof}
 	\end{korollar}
 
 	\begin{proposition}\label{2.4.24}
 		Sei $L|K$ separabel mit $n:=[L:K] < \infty$ und seien $x_1, \dots, x_n \in L$. Dann gilt
 		\begin{enumerate}[(a)]
 			\item \[d_[{L|K}(x_1, \dots, x_n) = 0 \Leftrightarrow x_1, \dots, x_n \textrm{ linear abh\"angig \"uber } K]\]
 			\item F\"ur jede $K$-lineare Abbildung $f::L \to L$ gilt 
 				\[d_{L|K}(f(x_1), \dots, f(x_n) ) = (\det f)^2 d_{L|K}(x_1, \dots, x_n)\]
 		\end{enumerate}
 	
 		\begin{proof}
 			\begin{enumerate}[(a)]
 				\item "$\Rightarrow$" aus \ref{2.4.23} und "$\Leftrightarrow$" aus Definition \ref{2.4.19}
  				\item Setze $\underline{v}:=(x_1, \dots, x_n)$ und $\underline{w}:=(f(x_1), \dots, f(x_n))$.
  				
  				Ist $\underline{v}$ keine Basis von $L$, so auch $\underline{w}$ nicht und beide Diskriminanten sind null.
  				
  				Ist $\underline{v}$ eine Basis von $L$, aber $\underline{w}$ nicht, so $d_{L|K}(\underline{w}) = \det f = 0$.
  				
  				Sind $\underline{v}$ und $\underline{w}$ $K$-Basen von $L$ und bezeichnet $b$ die Spurform von $L|K$, so 
  				\begin{flalign*}
  					d_{L|K}(\underline{w}) &= \det M(b,\underline{w})\\
  					&= \det(M(\underline{w}, \underline{v}))^TM(b,\underline{v})M(\underline{w}, \underline{v}) && \ref{2.4.16}(d)\\
  					&= (\underbrace{\det M(\underline{w}, \underline{v})}_{M(f(\underline{v}))})^2 d_{L|K}(\underline{v})
  				\end{flalign*}			
  			\end{enumerate} 
 		\end{proof}
 	\end{proposition}
 
 	\begin{proposition}\label{2.4.25}
 		Sei $A$ ein ganz abgeschlossener Integrit\"atsring, $K:=\qf(A)$, $L|K$ eine endliche K\"orpererweiterung und $B$ der ganze Abschluss von $A$ in $L$.
 		(z.B. $A \Z, K = \Q, L$ ein Zahlk\"orper und $B$ ein Zahlring [$\to$ \ref{2.1.15}]).
 		
 		Dann gilt $N_{L|K}(B) \subseteq A, \tr_{L|K}(B) \subseteq A$ und $B^* = \set{b \in B| N_{L|K}(b) \in A^*}$.
 		
 		\begin{proof}
 			Sei $b \in B$. Nach \ref{2.4.15} gilt $\chi_{L|K}(b) = (-1)^{[L:K]}(\irr_K(b))^{[L:K]}$ und nach \ref{2.1.14} gilt $\irr_K(b) \in A[X]$, also $\chi_{L|K}(b) \in A[X]$ und damit
 			$N_{L|K}(b) \in A$ und $\tr_{L|K}(b) \in A$ [$\to$ \ref{2.4.1}].
 			
 			Wegen $N_{L|K}(B) \subseteq A$ und der Multiplikativit\"at von $N_{L|K}$ folgt $B^* \subseteq \set{b \in B| N_{L|K}(b) \in A^*}$
 			
 			Sei umgekehrt $b \in B$ mit $N_{L|K}(b) \in A^*$. Dann gibt es $a_1, \dots, a_n \in A$ mit $b^n + a_1b^{n-1} + \cdots a_n = 0$ und
 			$a_n \in A^*$ (wegen $(\chi_{L|K}(b))(b) = 0$ und $(\chi_{L|K}(b)(0) \in A^*)$).
 			
 			Teilt man durch $a_nb^n \neq 0$, so ist 
 				\[\brac{\frac{1}{b}}^n + \frac{a_{n-1}}{a_n}\brac{\frac{1}{b}} + \cdots + \frac{a_1}{a_n}\brac{\frac{1}{b}} + \frac{1}{a_n} = 0\]
 			eine Ganzheitsgleichung von $\frac{1}{b}$ \"uber $A$, also ist $\frac{1}{b} \in B$ und somit $b \in B^*$. 
 		\end{proof}
 	\end{proposition}
 
 	\newpage
 	\section{Dedekindringe und K\"orpererweiterungen}\thispagestyle{sectionstart}
 	\begin{lemma}\label{2.5.1}
 		Sei $A$ ein Integrit\"atsring, $K:= \qf(A), L|K$ eine endliche K\"orpererweiterung und $B$ der ganze Abschluss von $A$ in $L$.
 		
 		Dann ist $L = \brac{A\setminus \set{0}}^{-1} B$ und es gibt Elemente von $B$, die eine Basis de $K$-Vektorraumes $L$ bilden.
 		
 		\begin{proof}
 			Sei $x \in L$. Dann gibt $n \in \N$ und $a_0, \dots, a_n \in A$ mit $a_nx^n + \cdots + a_0 = 0$ und $a_n \neq 0$. Multiplizieren mit $a_n^{n-1}$ liefert
 				\[(a_nx)^n + a_{n-1}(a_nx)^{n-1} + \cdots + a_0a_n^{n-1} = 0\]
 			woraus $a_nx \in B$ folgt und damit $x = a_n^{-1}(a_nx) \in \brac{A \setminus \set{0}}^{-1}B$ folgt.
 		\end{proof} 
 	\end{lemma}
 	
 	\begin{satz}\label{2.5.2}
 		Sei $A$ ein ganz abgeschlossener noetherscher Integrit\"atsring, $K:= \qf(A), L|K$ eine endliche separable K\"orpererweiterung und $B$ der ganze Abschluss von $A$ in $L$.
 		
 		Dann ist $B$ als $A$-Modul und daher als Ring noethersch.
 		
 		\begin{proof}
 			Nach Lemma \ref{2.5.1} gibt es $n \in \N_0$ und $v_1, \dot, v_n \in B$ derart, dass $\underline{v} := (v_1, \dots, v_n)$ eine Basis des $K$-Vektorraumes $L$ ist.
 			
 			Bezeichne $\underline{w} := (w_1, \dots, w_n)$ die dazu bez\"uglich der nach \ref{2.4.23} nicht ausgearteten Spurform von $L|K$ duale Basis [$\to$ \ref{2.4.17}]
 			
 			\emph{Behauptung: $\forall x \in Lx = \sum_{i=1}^{n} \tr_{L|K}(v_ix)w_i$}.
 			
 			\emph{Begr\"undung: } W\"ahle $a_1, \dots, a_n \in K$ mit $x = \sum_{i=1}^{n} a_iw_i$. Dann 
 			\begin{align*}
 				\tr_{L|K}(v_jx) &= \tr_{L|K}\brac{v_j\sum_{i=1}^{n}a_iw_i}\\
 				&= \sum_{i=1}^{n} a_i\tr_{L|K}(v_jw_i)\\
 				&= a_j
 			\end{align*}
 			f\"ur alle $j \in \set{1,\dots, n}$.
 			
 			Wegen $v_1, \dots, v_n \in B$ und $\tr_{L|K}(B) \overset{\ref{2.4.25}}{\subseteq} A$, folgt, dass $B \subseteq \sum_{i=1}^{n} Aw_i =: M$. Da $A$ noethersch ist,
 			ist $M$ als endlich erzeugter $A$-Modul nach \ref{1.4.7} auch noethersch. Damit ist auch $B$ ein noetherscher $A$-Modul.
 			
 			Nat\"urlich ist $B$ auch als $B$-Modul noethersch (d.h. als Ring), denn jeder $B$-Untermodul (d.h. jedes Ideal) von $B$ ist auch ein $A$-Untermodul von $B$ und daher als $A$-Modul 
 			und dann erst recht als $B$-Modul endlich erzeugt.
 		\end{proof}
 	\end{satz}
 
 	\begin{lemma}\label{2.5.3}
 		Sei $A \subseteq B$ ein ganze Erweiterung von Integrit\"atsringe. Sei $\mathfrak{p}$ ein Primideal und $I$ ein Ideal von $B$ mit $\mathfrak{p} \subseteq I$. Dann 
 		\[(A \cap \mathfrak{p} = A \cap I) \Rightarrow \mathfrak{p} = I\]
 		
 		\begin{proof}
 			Gelta $A \cap \mathfrak{p} = A \cap I$ und sei $x \in I$. Zu zeigen ist $x \in \mathfrak{p}$.
 			
 			W\"ahle $n \in \N_0$ und $a_0, \dots, a_n \in A$ mit $a_0x^n + \cdots + a_n = 0$ und $a_0 = 1$. W\"ahle $m \in \set{0, \dots, n}$ mit $a_m \notin \mathfrak{p}$ und $a_{m+1}, \dots, a_n \in \mathfrak{p}$. Nun 
 			\[x^{n-m}(a_0x^m + a_1x^{m-1} + \cdots + a_m) = -a_{m+1}x^{n-m-1} \cdots - a_n \in \mathfrak{p}\]
 			also $x \in \mathfrak{p}$ oder $a_0x^m + \cdots a_m \in \mathfrak{p} \subseteq I$. G\"alte letzteres, so $a_m \in I$ wegen $x \in I$ und daher $a_m \in A \cap I = A \cap \mathfrak{p} \subseteq \mathfrak{p}$.
 		\end{proof}
 	\end{lemma}
 
 	\begin{satz}\label{2.5.4}
 		Sei $A$ ein Dedekindring, $K:= \qf(A), L|K$ eine endliche separable K\"orpererweiterung und $B$ der ganze Abschluss von $A$ in $L$. Dann ist $B$ ein Dedekindring.
 		
 		\begin{proof}
 			Wir benutzen die Charakterisierung \ref{2.3.2} von Dedekindringen.
 			
 			Wegen \ref{2.5.2} ist $B$ noethersch. Wegen Lemma \ref{2.5.1} ist $L = \qf(B)$ und daher $B$ ganz abgeschlossen.
 			
 			Sei schlie\ss lich $\mathfrak{p} \neq (0)$ ein Primideal von $B$. ZU zeigen: $\mathfrak{p}$ ist maximal. Sei $I$ ein Ideal von $B$ mit $1 \notin I$ und $\mathfrak{p} \subseteq I$. Zu zeigen $\mathfrak{p} = I$.
 			
 			Nach Lemma \ref{2.5.3} reicht es $A \cap \mathfrak{p} = A \cap I$ zu zeigen. Dies folgt daraus, dass $A$ ein Dedekindring ist, denn $A \cap \mathfrak{p}$ ist ein Primideal und $A \cap I$ ist ein Ideal von $A$ mit $A \cap \mathfrak{p} \subseteq A \cap I, 1 \notin A \cap I$ und $A \cap \mathfrak{p} \neq (0)$ (w\"are $A \cap \mathfrak{p} = (0)$, so wende man Lemma \ref{2.5.3} nochmal an mit $(0)$ als Primideal und $\mathfrak{p}$ als Ideal).
 		\end{proof}
 	\end{satz}
 
 	\begin{satz}\label{2.5.5}
 		Sei $A$ ein Hauptidealring, $K:=\qf(A), L|K$ eine endliche separable K\"orpererweiterung und $B$ der ganze Abschluss von $A$ in $L$. 
 		
 		Dann ist $B$ ein freier $A$-Modul vom Rang $[L:K]$. Jede Basis des $A$-Moduls $B$ ist auch eine Basis des $K$-Vektorraumes $L$.
 		
 		\begin{proof}
 			Nach \ref{2.5.2} ist $B$ ein endlich erzeugter (sogar noetherscher) $A$-Modul. Da $B$ ein Integrit\"atsring ist, besitzt dieser Modul offensichtlich keine Torsionselemente $\neq 0$. Aber nach \ref{1.6.9} ist offensichtlich jeder endlich erzeugte Modul \"uber einem Hauptidealring, der keine Torsionselemente $\neq 0$ besitzt, frei. Insbesondere ist $B$ ein freier $A$-Modul.
 			
 			Insbesondere ist $B$ ein freier $A$-Modul. Wegen $K := \qf(A)$, ist jede seiner Basen auch $K$-linear unabh\"angig und wegen Lemma \ref{2.5.1} auch ein Erzeugenden System des $K$-Vektorraumes $L$.
 		\end{proof}
 	\end{satz}
 
 	\begin{korollar}\label{2.5.6}
 		Jeder Zahlring vom Grad $n$ [$\to$ \ref{2.1.15}] ist ein Dedekindring, dessen additive GRuppe ein freier $\Z$-Modul vom Rang $n$ ist.
 	\end{korollar}
 
 	\newpage
 	
 	\section{Die Idealklassengruppe}\thispagestyle{sectionstart}
 	
 	\begin{proposition}\label{2.6.1}
 		Sei $A$ ein kommutativer Ring. Es sind \"aquivalent:
 		\begin{enumerate}[(a)]
 			\item $A$ ist lokal [$\to$ \ref{1.5.8}(b)]
 			\item $A \setminus A^*$ ist ein Ideal von $A$
 			\item $A$ besitzt genau ein maximales Ideal
 			\item $0\neq 1$ und $\forall x \in A: (x \in A^* \lor 1-x \in A^*)$
 		\end{enumerate}
 	
 		\begin{proof}
 			$(a) \Rightarrow (b)$: Gelta (a) und setzte $I:= A \setminus A^*$. Dann $ 0 \in I$ (denn $0 \neq 1$ in $A$), $I + I \subseteq I$ und 
 			$AI \subseteq I$ (denn sind $a \in A$ und $x \in I$, so $ax \in I$, denn w\"are $ax \in A^*$, so auch $x \in A^*$, dann A kommutativ ist)
 			
 			$(b) \Rightarrow (c)$: Ist $I := A \setminus A^*$ ein Ideal von $A$, so ist jedes Ideal $J$ von $A$ mit $1 \notin J$ in $I$ enthalten. Es ist also $I$ das gr\"o\ss te Ideal $\neq A$ von $A$.
 			Insbesondere ist $I$ ein maximales Ideal und jedes maximale Ideal von $A$ gleich $I$.
 			
 			$(c) \Rightarrow (d)$: Beweis durch Kontraposition. Gelte $0 = 1$ oder $\exists x \in A: (x \notin A^* \land 1-x \notin A^*)$. 
 			
 			Falls $0 = 1$ in $A$, so besitzt $A$ kein maximales Ideal. 
 			
 			Sei nun $x \in A$ mit $x \notin A^*$ und $1-x \notin A^*$. Wegen $1 \notin (x)$ und $1 \notin (1-x)$ gibt es maximale Ideale $(x) \subseteq \mathfrak{m}, (1-x) \subseteq \mathfrak{n}$. Es gilt $\mathfrak{m} \neq \mathfrak{n}$, denn sonst $1 = x + (1-x) \in \mathfrak{m} = \mathfrak{n}$.
 			
 			$(d) \Rightarrow (a)$: Gelte (d) und seien $a, b \in A$ mit $a+b \in A^*$. Zu zeigen $a \in A^*$ oder $b \in A^*$.
 			
 			W\"ahle $c \in A$ mit $ac+ bc = (a+b)c = 1$. Wegen (d) gilt $ac \in A^*$ oder $bc \in A^*$, also $a \in A^*$ oder $b \in A^*$.
 		\end{proof}
 	\end{proposition}
 
 	\begin{erinnerung}\label{2.6.2}
 		\begin{enumerate}[(a)]
 			\item Sei $K$ ein K\"orper. Dann hei\ss t $v:K \to \Z \cup \set{\infty}$ ein \emph{diskrete Bewertung} auf $K$, wenn $v(0) = \infty$, $v\big|_{K^*}$ ein Gruppenhomomorphismus von $(K^*, \cdot)$ nach $(\Z, +)$ ist und $v(a + b)\geq \min \set{v(a), v(b)}$ f\"ur alle $a, b \in K$.
 			
 			\item Sei $K$ ein K\"orper und $v$ eine diskrete Bewertung auf $K$. Dann ist $\mathcal{O}_v := \set{a \in K| v(a) \geq 0}$ ein Unterring von $K$, er sogenannte Bewertungsring von $v$. Es gilt 
 			$\mathcal{O}_v^* = \set{a \in K| v(a) = 0}$ und $\mathcal{O}_v$ ist ein lokaler Ring mit maximalem Ideal $\mathfrak{m}_v = \set{a \in K| v(a) > 0}$ und Restklassenk\"orper $\mathcal{O}_v / \mathfrak{m}_v$.
 			
 			\item Sei $A$ ein faktorieller Ring, $K:= \qf(A)$. F\"ur jedes $x \in K^*$ gibt es genau ein $(c \alpha_x) \in A^*\times \Z^{(\P_A)}$ mit $x = c \prod_{p \in \supp(\alpha_x)} p^{\alpha_x(p)}$. Dann ist f\"ur jedes $p \in \P_A$ die $p$-Bewertung $v_p: K \to \Z, x \mapsto \begin{cases}
 				\infty & \textrm{falls } x = 0\\
 				\alpha_x(p) & \textrm{sonst}
 			\end{cases}$ eine diskrete Bewertung auf $K$.
 		\end{enumerate}
 	\end{erinnerung}
 
 	\begin{notation}\label{2.6.3}
 		Sei $A$ ein Dedekindring.  $M_A := \set{\mathfrak{p}| \mathfrak{p} \textrm{ Primideal von } A, \mathfrak{p} \neq (0)} \overset{\ref{2.2.12}(d)}{=} \set{\mathfrak{m}| \mathfrak{m} \textrm{ maximales von } A, \mathfrak{m} \neq (0)}$
 		
 		$I_A := \set{I| I \textrm{ gebrochenes Ideal von } A, I \neq (0)} \overset{\ref{2.2.7}(e), \ref{2.2.12}(c)}{=} \set{I | I \textrm{ e.e. A-Untermodul von } \qf(A), I \neq 0}$
 		
 		$P_A := \set{I| I \textrm{ gebrochenes Hauptideal von } A, I \neq (0)} = \overset{\ref{2.2.6}}{=} \set{I | I \textrm{ zyklischer A-Untermodul von} \qf(A), I \neq 0}$
 	\end{notation}
 
 	\begin{satz-not}\label{2.6.4}
 		Sei $A$ ein Dedekindring, $K:= \qf(A)$. F\"ur jedes $I \in I_A$ gibt es genau ein $\alpha_I \in \Z^{(M_A)}$ mit $I=\prod_{\mathfrak{p} \in \supp(\alpha_I)} \mathfrak{p}^{\alpha_I(\mathfrak{p})}$
 		
 		Definiere f\"ur $\mathfrak{p} \in M_A$ $\tilde{v}_\mathfrak{p}: I_A \cup \set{0} \to \Z \cup \set{\infty}$, $I \mapsto \begin{cases}
 			\infty & \textrm{ falls } I = 0\\
 			\alpha_I(\mathfrak{p}) & \textrm{sonst}
 		\end{cases}$
 	
 		und die \emph{$\mathfrak{p}$-Bewertung} $v_\mathfrak{p}: K \to \Z \cup \set{\infty}, x \mapsto \tilde{v}_\mathfrak{p}(xA)$.
 		
 		Dann gilt 
 		\begin{enumerate}[(a)]
 			\item 
 			$I_A \to \Z^{(M_A)}, I \mapsto \alpha_I = (\tilde{v}_\mathfrak{p}(I))_{\mathfrak{p} \in M_A}$ ist ein Isomorphismus zwischen der Menge der durch Inklusion halbgeordneten Menge $I_A$ und der durch
 			$\alpha \preceq \beta : \Leftrightarrow \forall \mathfrak{p} \in M_A: \alpha(\mathfrak{p}) \geq \beta(\mathfrak{p})$ ($\alpha, \beta \in \Z^{(M_A)}$) halbgeordneten Menge $\Z^{M_A}$
 			
 			\item
 			F\"ur alle $\mathfrak{p} \in M_A$ und $I, J \in I_A$ gilt $\tilde{v}_\mathfrak{p}(IJ) = \tilde{v}_\mathfrak{p}(I) + \tilde{v}_\mathfrak{p}(J)$ und $\tilde{v}_\mathfrak{p}(I+J) = \min \set{\tilde{v}_\mathfrak{p}, \tilde{v}_\mathfrak{p}(J)}$
 			
 			\item F\"ur alle $\mathfrak{p} \in M_A$ ist $v_\mathfrak{p}$ ein diskrete Bewertung auf $K$. 
 		\end{enumerate}
 	
 		\begin{proof}
 			Die Existenz von $\alpha_I \in \N_0^{(M_A)}$ mit $I = \prod_{\mathfrak{p} \in \supp(\alpha_I)} \mathfrak{p}^{\alpha_I(\mathfrak{p})}$ folgt f\"ur Ideale $I \in I_A$ aus der Definition eines Dedekindringes \ref{2.2.4}. 
 			
 			Da jedes $I \in I_A$ von der Form $JK^{-1}$ f\"ur Ideale $J,K \in I_A$ ist (sogar mit $K$ Hauptideal, siehe \ref{2.2.7}(b))m folgt die Existenz von $\alpha_I \in \Z^{(M_A)}$ mit $I = \prod_{\mathfrak{p} \in \supp(\alpha_I)} \mathfrak{p}^{\alpha_I(\mathfrak{p})}$.
 			
 			Die Eindeutigkeit dieses $\alpha_I$ folgert man leicht aus \ref{2.2.11} mit \ref{2.2.12}(a).
 			
 			(a) Betrachte $\Phi: I_A \to \Z^{(M_A)}, I \mapsto \alpha_I$ und $\Psi: \Z^{(M_A)} \to I_A, \alpha \mapsto \prod_{\mathfrak{p} \in \supp(\alpha)} \mathfrak{p}^{\alpha(\mathfrak{p})}$.
 			
 			Es reicht zu zeigen 
 			\begin{enumerate}[(1)]
 				\item $\Phi \circ \Psi = \id_{\Z^{(M_A)}}$
 				\item $\Psi \circ \Phi = \id_{I_A}$
 				\item $\forall I, J \in I_A: (I \subseteq J \Rightarrow \Phi(I) \preceq \Phi(J))$
 				\item $\forall \alpha, \beta \in \Z^{(M_A)}: ( \alpha \preceq \beta \Rightarrow \Psi(\alpha) \subseteq \Psi(\beta))$
 			\end{enumerate}
 		
 			(1) und (2) sind klar. Zu (3). Seien $I, J \in I_A$ mit $I \subseteq J$. Dann ist $J^{-1}I \subseteq J^{-1}J = A$ ein Produkt von Primidealen, also $\Phi(J^{-1}I) \preceq 0$. Somit 
 			$\Phi(I) = \Phi((JJ^{-1})I) = \Phi(J(J^{-1}I)) = \Phi(J) + \Phi(J^{-1}I) \preceq \Phi(J)$.
 			
 			Zu (4). Seien $\alpha, \beta \in \Z^{(M_A)}$ mit $\alpha \preceq \beta$. Dann $\alpha - \beta \preceq 0$ und daher $\Psi(\alpha - \beta ) \subseteq A$.
 			
 			Somit $\Psi(\alpha) = \Psi(\alpha-\beta + \beta) = \Psi(\alpha-\beta)\Psi(\beta) \subseteq A\Psi(\beta) \subseteq \Psi(\beta)$
 			
 			(b) Die erste Gleichung ist trivial. Die beiden anderen Gleichungen folgen aus (a) durch die folgenden Beobachtungen:
 			
 			In der halbgeordneten Menge $I_A$ ist $\inf \set{I, J} = I \cap J$ und $\sup \set{I, J} = I + J$ f\"ur alle $I, J \in I_A$ und in der durch (*) halbgeordneten Menge
 			$\Z^{(M_A)}$ ist $\inf \set{\alpha, \beta} = \begin{pmatrix}
 					M_A &\to \Z\hfill\\
 					\mathfrak{p} &\mapsto \max \set{\alpha(\mathfrak{p}), \beta(\mathfrak{p})}
 			\end{pmatrix}$
 		
 			und $\sup \set{\alpha, \beta} = \begin{pmatrix}
 					M_A &\to \Z\hfill\\
 					\mathfrak{p} &\mapsto \min \set{\alpha(\mathfrak{p}), \beta(\mathfrak{p})}
 			\end{pmatrix}$ f\"ur alle $\alpha, \beta \in \Z^{(M_A)}$
 		
 			(c) Sei $\mathfrak{p} \in M_A$. Dann gilt $v_\mathfrak{p}(0) = \tilde{v}_\mathfrak{p}(0) = \infty$, $v_\mathfrak{p}(xy) = \tilde{v}_\mathfrak{p}((xy)A) = \tilde{v}_\mathfrak{p}((xA)(yA)) \overset{(b)}{=} \tilde{v}_\mathfrak{p}(xA) + \tilde{v}_\mathfrak{p}(yA)$ f\"ur alle $x, y\in K^*$ und $v_\mathfrak{p}(x+y) = \tilde{v}_\mathfrak{p}((x+y)A) \geq \tilde{v}_\mathfrak{p}(xA) + \tilde{v}_\mathfrak{p}(yA) \overset{(b)}{=} \min \set{\tilde{v}_\mathfrak{p}(xA), \tilde{v}_\mathfrak{p}(yA)} = \min \set{v_\mathfrak{p}(x), v_\mathfrak{p}(y)}$ f\"ur alle $x,y \in K^*$ mit $x+y\neq 0$
 		\end{proof}
 	\end{satz-not}
 
 	\begin{korollar}\label{2.6.5}
 		Sei $A$ ein Dedekindring. $I_A$ ist eine multiplikativ geschriebene abelsche Gruppe und als solche ein freier $\Z$-Modul mit Basis $M_A$.
 		
 		\begin{proof}
 			Dass $I_A$ eine abelsche Gruppe ist , ist klar. Aus \ref{2.6.4} folgt, dass $I_A \to \Z^{(M_A)}, I \mapsto  (\tilde{v}_\mathfrak{p}(I))_{\mathfrak{p} \in M_A}$ auch ein Isomorphismus zwischen der multiplikativ geschriebenen Gruppe $I_A$ und der additiv geschriebenen Gruppe $\Z^{(M_A)}$ ist. Unter diesem Isomorphismus wird $M_A$ auf die kanonische Basis des $\Z$-Moduls $\Z^{(M_A)}$ abgebildet wird. 
 		\end{proof}
 	\end{korollar}
 
 	\begin{bem-not}\label{2.6.6}
 		Sei $A$ ein Dedekindring. Dann ist $P_A$ eine Untergruppe von $I_A$. Man nennt $C_A := I_A /P_A$ die \emph{(Ideal-)Klassengruppe} von $A$ und deren Ordnung $\#C_A \in \N \cup \set{\infty}$ die \emph{Klassenzahl} von $A$. 
 	\end{bem-not}
 
 	\begin{proposition}\label{2.6.7}
 		Sei $A$ ein Dedekindring. Es sind \"aquivalent
 		
 		\begin{enumerate}[(a)]
 			\item $A$ ist ein Hauptidealring
 			
 			\item $\#C_A = 1$
 			
 			\item $A$ ist faktoriell
 		\end{enumerate}
 	
 		\begin{proof}
 			(a) $\Leftrightarrow I_A = P_A \Leftrightarrow I_A / P_A = \set{1} \Leftrightarrow C_A = \set{1} \Leftrightarrow \#C_A = 1 \Leftrightarrow (b)$.
 			
 			(a) $\Rightarrow$ (c) klar
 			
 			(c) $\Rightarrow$ (a). Gelte (c). Da die Gruppe $I_A$ von $M_A$ erzeugt wird, reicht es $M_A \subseteq P_A$ zu zeigen. Sei hierzu $\mathfrak{p} \in M_A$. W\"ahle $x \in \mathfrak{p} $ mit $x \neq 0$. W\"ahle $n \in \N$ und Primelemente $p_1, \dots, p_n \in A$ mit $x = p_1 \cdot \cdots \cdot p_n$. Wegen $p_1 \cdot \cdots \cdot p_n \in \mathfrak{p}$ gibt es $i \in \set{1,\dots, n}$ mit $p_i \in \mathfrak{p}$. Weil $(p_i) \in M_A$ gilt muss $(p_i) = \mathfrak{p}$ sein.		
 		\end{proof}
 	\end{proposition}
 
 	\begin{korollar}\label{2.6.8}
 		Ein Ring genau dann ein Hauptidealring, wenn er ein faktorieller Dedekindring ist.
 	\end{korollar}
 
 	\begin{bemerkung}\label{2.6.9}
 		Sei $A$ ein Hauptidealring. Dann gilt f\"ur alle $p \in \P_A, v_p = v_\mathfrak{p}$, wobei $\mathfrak{p} = (p) \in M_A$.
 	\end{bemerkung}

	\newpage
 	\section{Zerlegungsgesetze}\thispagestyle{sectionstart}
 	
 	\begin{satz}\label{2.7.1}
 		(Nakayama-Lemma)\newline
 		Sei $R$ ein kommutativer Ring, $I$ ein Ideal von $R$ und $M$ ein endlich erzeugter $R$-Modul mit $IM := \set{\sum_i a_ix_i|a_i \in R, x_i \in M} = M$.
 		
 		Dann gibt es $a \in R$ mit $1-a \in I$ und $aM = 0$.
 		
 		\begin{proof}
 			Wendet man Cayley-Hamilton \ref{1.7.5} auf $f := \id_M$ an, so erh\"alt man $n \in \N_0$ und $a_1, \dots, a_n \in I$ mit $f^n + af^{n-1} + \cdots + a_n\id_M = 0$. Dann $1-a = -(a_1 + \cdots + a_n) \in I$ und $a \id_M = 0$.
 		\end{proof}
 	\end{satz}
 
 	\begin{bemerkung}\label{2.7.2}
 		In der Situation \ref{2.7.1} ist $1-a \in I$ ein "Zeuge" f\"ur $IM = M$, denn $M = 1 \cdot M = (1-a+a)M \subseteq (1-a)M + aM = (1-a)M$, also $(\underbrace{1-a}_{\in I})M = M$.
 	\end{bemerkung}
 
 	\begin{lemma}\label{2.7.3}
 		Sei $A$ ein Dedekindring, $I \in I_A$ und $\mathfrak{p} \in M_A$. Dann $A/\mathfrak{p} \cong I / I\mathfrak{p}$ als $A$-Modul [schon in \ref{1.4.17}(b) gezeigt, falls $A$ Hauptidealring]
 		
 		\begin{proof}
 			Es gibt $x \in I \setminus I_\mathfrak{p}$ (sonst $I = I\mathfrak{p}$ und daher $A = \mathfrak{p}$). Der Kern des $A$-Modulhomomorphismus $A\to I/I\mathfrak{p}, a \mapsto \overline{ax}$ umfasst $\mathfrak{p}$, aber enth\"alt nicht $1$, und somit $\mathfrak{p}$. Z.z. ist dieser Homomorphismus ist surjektiv, da $I = Ax + I\mathfrak{p}$. Dies folgt mit $I\mathfrak{p} \subsetneq Ax + I \mathfrak{p} \subseteq I$ aus \ref{2.6.4}(a)
 		\end{proof}
 	\end{lemma}
 
 	\begin{bem-not}\label{2.7.4}
 		Ist $A\subseteq B$ eine Ringerweiterung und $I$ ein Ideal von $A$, so bezeichne $BI$ das von $I$ in $B$ erzeugte Ideal. Es gilt $BI = \set{\sum_i b_ia_i|b_i \in B, a_i \in I}$
 	\end{bem-not}
 	
 	\begin{def-satz}
 		Sei $A$ ein Dedekindring, $K:= \qf(A)$, $L|K$ eine endliche separable K\"orpererweiterung und $B$ der ganze Abschluss von $A$ in $L$ (damit $B$ ein Dedekindring nach \ref{2.5.4})
 		
 		\begin{enumerate}[(a)]
 			\item 
 			Sei $\mathfrak{q} \in M_B$. Dann gibt es genau ein $\mathfrak{p} \in M_A$ mit $\mathfrak{p} \subseteq \mathfrak{q}$, n\"amlich $\mathfrak{p} := A \cap \mathfrak{q}$.
 			
 			Man nennt $e_A(\mathfrak{q}) = \tilde{v}_\mathfrak{q}(B\mathfrak{p})$ den \emph{Verzweigungsindex} und $f_A(\mathfrak{q}) = [(B/\mathfrak{q}):(A/\mathfrak{p})]$ den \emph{Tr\"agheitsindex} von $\mathfrak{q}$ \"uber $A$, wobei man $A/\mathfrak{p}$ verm\"oge $A/\mathfrak{p} \hookrightarrow B/ \mathfrak{q}, \overline{a^\mathfrak{p}} \mapsto \overline{a^\mathfrak{q}} (a \in A)$ als Unterk\"orper von $B/\mathfrak{q}$ auffasst. Es gilt $e_A(\mathfrak{q}) \in \N$ und $f_A(\mathfrak{q}) \in \N$.
 			
 			\item 
 			Sei $\mathfrak{p} \in M_A$. Dann ist $Q:= \set{\mathfrak{q} \in M_B| \mathfrak{p} \subseteq \mathfrak{q}}$ endlich. Es gilt $B\mathfrak{p} = \prod_{\mathfrak{q} \in Q} \mathfrak{q}^{e_A(\mathfrak{q})}$ und 
 			$\sum_{\mathfrak{q} \in Q} e_A(\mathfrak{q})f_A(\mathfrak{q}) = \dim_{A/\mathfrak{p}}(B/B\mathfrak{p}) = [L:K]$, wobei man $B/B\mathfrak{p}$ verm\"oge $\overline{a^\mathfrak{p}}\overline{b^{B\mathfrak{p}}} (a \in A, b \in B)$ als $A/\mathfrak{p}$ Vektorraum auffasst. Insbesondere $Q \neq \emptyset$
 		\end{enumerate}
 	
 		\begin{proof}
 			\begin{enumerate}[(a)]
 				\item 
 				Nach Lemma \ref{2.5.3} gilt $A \cap \mathfrak{q} \neq (0)$ also $A \cap \mathfrak{q} \in M_A$. Ist $\mathfrak{p} \in M_A$ mit $\mathfrak{p} \subseteq \mathfrak{q}$, so ist $\mathfrak{p} \subseteq A \cap \mathfrak{q}$ und daher $\mathfrak{p} = A \cap \mathfrak{q}$.
 				
 				Wegen $(0) \neq B\mathfrak{p} \subseteq B$ ist $e_A(\mathfrak{q}) = \tilde{v}_\mathfrak{q}(B\mathfrak{p}) \in \N$ klar.
 				
 				Nach Satz \ref{2.5.2} (und \ref{2.2.12}) ist $B$ als $A$-Modul endlich erzeugt und daher $B/\mathfrak{q}$ ein endlich erzeugter  $A/\mathfrak{p}$ Vektorraum, also $f_A(\mathfrak{q}) = [(B/\mathfrak{q}):(A/\mathfrak{p})] = \dim_{A/\mathfrak{p}}(B/\mathfrak{q} ) \in \N$.
 				
 				
 				\item 
 				Es ist $Q = \set{\mathfrak{q} \in M_B| B\mathfrak{p} \subseteq \mathfrak{q}} \overset{\ref{2.4.6}(a)}{\subseteq} \set{\mathfrak{q} \in M_B| \tilde{v}_\mathfrak{q}(B\mathfrak{p}) \geq 1} $ endlich und $B\mathfrak{p} = \prod_{\mathfrak{q} \in Q} \mathfrak{q}^{e_A(\mathfrak{q})}$ nach \ref{2.6.4}.
 				
 				Wie in \ref{1.4.17}(b) kann man eine Kompositionsreihe des $B$-Moduls $B/B\mathfrak{p}$ der L\"ange $\sum_{\mathfrak{q} \in Q} e_A(\mathfrak{q})$ hinschreiben, deren Faktoren nach Lemma \ref{2.7.3} gerade die $B/\mathfrak{q}$ ($\mathfrak{q} \in Q, B/\mathfrak{q} \ e_A(\mathfrak{q})$-mal) sind.
 				
 				Die in dieser Kompositionsreihe vorkommenden abelschen Gruppen bilden auch Untervektorr\"aume des $A/\mathfrak{p}$-Vektorraum $B/B\mathfrak{p}$ und es folgt (vgl. \ref{1.4.10}) $\dim_{A/\mathfrak{p}} (B/B\mathfrak{p}) = \sum_{\mathfrak{q} \in Q} e_A(\mathfrak{q})\dim_{A/\mathfrak{p}} (B/\mathfrak{q}) = \sum_{\mathfrak{q} \in Q} e_A(\mathfrak{q})f_A(\mathfrak{q})$.
 				
 				Es bleibt noch $\dim_{A/\mathfrak{p}}(B/B\mathfrak{p}) = \dim_K L$ zu zeigen. W\"ahle hierzu $n \in \N_0$ und $x_1, \dots, x_n \in B$ derart, dass $\overline{x_1^{B\mathfrak{p}}}, \dots, \overline{x_n^{B\mathfrak{p}}}$ eine Basis des $A/\mathfrak{p}$-Vektorraumes $B/B\mathfrak{p}$ ist (beachte, dass $B$ ein endlich erzeugter $A$-Modul ist, wie schon erw\"ahnt).
 				
 				Wir zeigen, dass $x_1, \dots, x_n$ eine Basis des $K$-Vektorraumes $L$ bilden.
 				
 				Um die lineare Unabh\"angigkeit zu zeigen: Seien $a_1, \dots, a_n \in K$ mit $a_1x_1 + \cdots + a_nx_n = 0$. Annahme $I:= a_1A + \cdots a_nA \neq 0$. Dann $I \in I_A$. F\"ur jedes $s \in I^{-1}$ gilt dann $\overline{sa_1^{B}}\overline{x_1^{B\mathfrak{p}}} + \cdots + \overline{sa_n^{B}}\overline{x_n^{B\mathfrak{p}}} = 0$, also $\overline{sa_1^{B}} = \cdots = \overline{sa_n^{B}}  = 0$ und somit $sI \subseteq \mathfrak{p}$. Da $s\in I^{-1}$ beliebig war, folgt $A = I^{-1}I \subseteq \mathfrak{p}$.
 				
 				Schlie\ss lich zeigen wir $L = Kx_1 + \cdots + Kx_n$. Wegen $L = (A \setminus \set{0})^{-1}B$ [$\to$ \ref{2.5.1}] reicht e $B \subseteq Kx_1 + \cdots + Kx_n$ zu zeigen.  
 				
 				Tats\"achlich zeigen wir $B \subseteq \frac{1}{a} (Ax_1 + \cdots + Ax_n)$ f\"ur ein $a \in A \setminus \set{0}$.
 				
 				Dies ist \"aquivalent zu $aB \subseteq Ax_1 + \cdots Ax_n$ f\"ur ein $a \in A \setminus \set{0}$, was wiederum zu $a(B/(Ax_1 + \cdots Ax_n)) = 0$ f\"ur ein $a \in A \setminus \set{0}$ \"aquivalent ist.
 				
 				Wegen des Nakayama Lemma \ref{2.7.1} reicht es zu zeigen $B/(Ax_1 + \cdots+ Ax_n) = \mathfrak{p}(B/(Ax_1 + \cdots + Ax_n))$. Die folgt aber aus $B \subseteq Ax_1 
 				+ \cdots Ax_n + B\mathfrak{p}$
 			\end{enumerate}
 		\end{proof}
 	\end{def-satz}
 	
 	\begin{satz-def}\label{2.7.6}
 		Sei $A$ ein Dedekindring, $K:= \qf(A), L|K$ eine endliche Galoiserweiterung, $B$ der ganze Abschluss von $A$ in $L$ und $\mathfrak{p} \in M_A$. Dann wirkt die Galoisgruppe $G:= \Aut(L|K)$ in nat\"urlicher Weise auf $L$, auf $B$, auf $M_B$ und auf $Q:= \set{\mathfrak{q} \in M_B| \mathfrak{p} \subseteq \mathfrak{q}} \overset{\ref{2.7.5}(b)}{\neq} \emptyset$.
 		
 		Die Wirkung von $G$ auf $Q$ ist transitiv. Insbesondere ist f\"ur $\mathfrak{p} \in M_A$ der \emph{Verzweigungsgrad \"uber} $\mathfrak{p}$ in $B$ $e_\mathfrak{p}(B) := e_A(\mathfrak{q})$ und der Tr\"agheitsindex \"uber $\mathfrak{p}$ in $B$ $f_\mathfrak{p}(B) := f_A(\mathfrak{q})$ unabh\"angig von $\mathfrak{q} \in Q$ und es gilt $e_\mathfrak{p}(B)f_\mathfrak{p}(B)\#Q = [L:K]$
 		
 		\begin{proof}
 			Angenommen $G:= \Aut(L|K)$ wirkt auf $Q$ nicht transitiv. Dann gibt es $\mathfrak{m}, \mathfrak{q} \in Q$ mit $\mathfrak{m} \neq \varphi \mathfrak{q}$ f\"ur alle $\varphi \in G$. Nach \ref{2.6.4}(b) gilt dann $\mathfrak{m} + \prod_{\varphi \in G} \varphi \mathfrak{q} = B$. W\"ahle $x \in \mathfrak{m}, y \in \prod_{\varphi \in G} \varphi \mathfrak{q}$  mit $x+y = 1$.
 			
 			Es gilt $x \notin \varphi \mathfrak{q}$ f\"ur alle $\varphi \in G$, denn sonst $1 = x + y \in \varphi \mathfrak{q}$ f\"ur ein $\varphi \in G$.
 			\newline
 			
 			Also $\varphi^{-1}(x) \notin \mathfrak{q}$ f\"ur alle $\varphi \in G$.
 			
 			Da $\mathfrak{q}$ ein Primideal ist, folgt $\prod_{\varphi \in G} \varphi^{-1}(x) \notin \mathfrak{q} $, also $N_{L|K}(x) \overset{\ref{2.4.6}}{=} \prod_{p \in G} \varphi(x) = \prod_{\varphi \in G} \varphi^{-1}(x) \notin \mathfrak{q}$. Andererseits $N_{L|K}(x) = x \prod_{\varphi \in G \setminus \set{1}} \varphi(x) \in \mathfrak{m} \cap A = \mathfrak{p} \subseteq \mathfrak{q}$, da $x \in \mathfrak{m}$ und $N_{L|K}(x) \in A$ (\ref{2.4.5}).
 		\end{proof}
 	\end{satz-def}
 
 	\newpage
 	\chapter[tocentry={Zahlringe}]{Zahlringe}
 	\section{Gitter in Zahlk\"orpern}
 	\begin{prop-def}\label{3.1.1}
 		Sei $K$ ein Zahlk\"orper vom Grad $n$. 
 		
 		Jede endlich erzeugte Untergruppe $M$ von $K$ ist als $\Z$-Modul frei vom Rang $\leq n$ und es sind \"aquivalent:
 		\begin{enumerate}[(a)]
 			\item $\rk M = n$
 			\item $M$ hat eine $\Z$-Basis, welche eine $\Q$-Basis von $K$ ist
 			\item Jede $\Z$-Basis von $M$ ist eine $\Q$-Basis von $K$  
 			\item $\forall a \in K: \exists s \in \N: sa \in M$
 		\end{enumerate}
 		
 		Sind (a)-(d) erf\"ullt, so hei\ss t $M$ ein \emph{Gitter} in $K$.
 		\begin{proof}
 			\"Ubung
 		\end{proof}
 	\end{prop-def}
 
 	\begin{definition}\label{3.1.2}
 		Ein Gitter $M$ hei\ss t \emph{multiplikativ}, wenn es eine multiplikative Menge ist, das hei\ss t $1 \in M$ und $\forall x,y \in M: xy \in M$.
 	\end{definition}
 
 	\begin{beispiel}\label{3.1.3}
 		Nach \ref{2.5.6} ist jeder Zahlring ein multiplikatives Gitter im zugeh\"origen Zahlk\"orper.
 	\end{beispiel}
 
 	\begin{lemma}\label{3.1.4}
 		Seien $M$ und $N$ Gitter im Zahlk\"orper $K$. 
 		
 		Dann gibt es ein $s \in \N$ mit $sM \subseteq N$ und $sN \subseteq M$. Gilt zus\"atzlich $N \subseteq M$, so ist $M/N$ endlich.
 		
 		\begin{proof}
 			Nach \ref{3.1.1}(d) gilt $\forall a \in K: \exists s \in \N: sa \in N$, insbesondere $\forall a \in M: \exists s \in \N: sa \in N$. Da $M$ endlich erzeugt ist, haben wir sogar 
 			$\exists s \in \N: \forall a \in M: sa \in N$. Also gibt es $s_1 \in \N$ mit $s_1M \subseteq N$ und analog $s_2 \in \N$ mit $s_2N \subseteq M$. Dann $sM \subseteq N$ und $sN \subseteq M$ f\"ur $s:= s_1s_2 \in \N$.
 			
 			Gelte nun $N \subseteq M$. Dann $s(M/N) = 0$ und da $M/N$ ein endlich erzeugter $\Z$-Modul ist, folgt $\#(M/N) < \infty$.
 		\end{proof}
 	\end{lemma}
 
	\begin{satz}\label{3.1.5}
		Sei $K$ ein Zahlk\"orper und $M \subseteq K$. 
		
		Dann ist $M$ ein multiplikatives Gitter in $K$ genau dann, wenn $M$ ein Unterring von $\mathcal{O}_K$ mit $K = \qf(M)$ ist.
		
		\begin{proof}
			\"Ubung
		\end{proof}
	\end{satz}

	\begin{def-prop}\label{3.1.6}
		Sei $M$ ein Gitter im Zahlk\"orper $K$ und $x_1, \dots, x_n$ ein $\Z$-Basis von $M$.
		
		Dann hei\ss t $d(M) := d_{K|\Q}(x_1, \dots, x_n) \overset{\ref{2.4.24}(a)}{\in} \Q^*$ die \emph{Diskriminante} des Gitters $M$. Sie h\"angt nicht 
		von der Wahl der Basis ab. Gilt $M \subseteq \mathcal{O}_K$, so $d(M) \in \Z$.
		
		\begin{proof}
			Seien $\underline{x} = (x_1, \dots, x_n)$ und $\underline{y} = (y_1, \dots, y_n)$ $\Z$-Basen von $M$. 
			
			Dann sind nach \ref{3.1.1}(c) $\underline{x}$ und $\underline{y}$ auch $\Q$-Basen von $K$. Bezeichne $f:K\to K$ die $\Q$-lineare Abbildung von $K$ mit 
			$f(x_i) = y_i$ f\"ur $i \in \set{1,\dots, n}$. Wegen $y_i \in \Z x_1 + \cdots + \Z x_n$ f\"ur $i \in \set{1,\dots, n}$ gilt $M(f, \underline{x}) \in \Z^{n\times n}$.
			Also $\det f = \det M(f, \underline{x}) \in \Z$.
			Analog $\det f^{-1} \in \Z$. Wegen $(\det f) (\det f^{-1}) = \det \id_K = 1$, also $\det f \in \Z^* = \set{-1, 1}$, also $(\det f)^2 = 1$. Nach \ref{2.4.24}(b) gilt 
			$d_{K|\Q}(f(x_1), \dots, f(x_n)) = d_{K|\Q}(x_1, \dots, x_n)$.
			
			Ist schlie\ss lich $M \subseteq \mathcal{O}_K$, so $d(M) \in \Z$ wegen \ref{2.4.19} und \ref{2.4.25}. 
		\end{proof}
	\end{def-prop}

	\begin{satz}\label{3.1.7}
		Seien $M$ und $N$ Gitter des Zahlk\"orpers $K$ mit $N \subseteq M$. Dann $d(N) = [M:N]^2 d(M)$.
		
		\begin{proof}
			W\"ahle $\Z$-Basen $\underline{x} = (x_1, \dots, x_n)$ von $M$ und $\underline{y} = (y_1, \dots, y_n)$ von $N$.
			
			Bezeichne wieder $f:K\to K$ die $\Q$-lineare Abbildung mit $f(x_i) = y_i$ f\"ur $i \in \set{1, \dots, n}$.
			
			Nach \ref{2.4.24}(b) ist $\det f = [M:N]$ zu zeigen. Betrachte den $\Z$-Modulisomorphismus $\iota : \Z^n \to M, \begin{pmatrix}
				a_1\\ \vdots \\ a_n
			\end{pmatrix} \mapsto \sum_{i=1}^{n} a_ix_i$ [$\to$ \ref{1.2.7}].
		
			Setzt man $N' := \iota^{-1} (N)$, so gilt nat\"urlich $[M:N] = [\Z^n:N'] = \#(\Z^n/N')$. Weiter ist $\iota^{-1}(y_i)$
			die $i$-te Spalte von $M(f, \underline{x})$, also $N' = \Z\iota^{-1}(y_1) + \cdots + \Z\iota^{-1}(y_n) = \im M(f, \underline{x})$.
			
			Wendet man nun das Verfahren aus \ref{1.6.6}(b) auf $M(f, \underline{x}) \in \Z^{n \times n}$ an, so erh\"alt man $S \in \N^{n\times n}$ in Diagonalform (sogar 
			Smithscher Normalform) und $P, Q \in \GL_n(\Z)$ mit $S = PM(f, \underline{x})Q$.
			
			Ist $S = \begin{pmatrix}
				a_1, 0 \cdots & 0\\
				0 & a_2 & \cdots 0\\
				\vdots & \vdots & \ddots & \vdots\\
				0 & 0 & \cdots & a_n
			\end{pmatrix}$, so gilt nach \ref{1.6.6}(a) $\Z^n/N' \cong \Z^n / (a_1\Z \times \cdots \times a_n \Z) \cong (\Z/a_1\Z) \times \cdots \times (\Z/a_n\Z)$ 
			und somit $[M:N] = [\Z^n:N'] = a_1 \cdots a_n = \det S = \underbrace{(\det P)}_{\in \Z^*} (\det M(f, \underline{x})) \underbrace{(\det Q)}_{\in \Z^*}$.
			
			Also wegen $\Z^* = \set{-1, 1}$ 
				\[|\det f| = |\det M(f, \underline{x})| = [M:N]\]
 		\end{proof}
	\end{satz}

	\begin{korollar}\label{3.1.8}
		Sei $K$ ein Zahlk\"orper und $M$ ein multiplikatives Gitter in $K$ mit $\forall p \in \P: p^2 \nmid d(M)$.
		
		Dann gilt $M = \mathcal{O}_K$
		
		\begin{proof}
			Nach \ref{3.1.5} gilt $M \subseteq \mathcal{O}_K$ und daher nach \ref{3.1.7} $d(M) = [\mathcal{O}_K:M]^2d(\mathcal{O}_K)$, 
			also $[\mathcal{O}_K:M] = 1$
		\end{proof}
	\end{korollar}

	\begin{beispiel}\label{3.1.9}
		Sei $d \in \nsquare$ [$\to$ \ref{2.1.16}]. Dann sind die Identit\"at und $x+y\sqrt{d} \mapsto (x+y\sqrt{d})^* := x- y \sqrt{d}$ ($x,y \in \Q$) die beiden verschiedenen
		Einbettungen des quadratischen Zahlk\"orpers $\Q(\sqrt{d})$ in seinen algebraischen Abschluss.
		
		Dann ist $\Z[\sqrt{d}] = \Z \oplus \Z\sqrt{d}$ ein multiplikatives Gitter in $\Q(\sqrt{d})$ mit Diskriminante [$\to$ \ref{2.4.21}]
		\[\brac{\det \begin{pmatrix}
				1 & \sqrt{d}\\1^* & \sqrt{d}^*
		\end{pmatrix}}^2 = (-2\sqrt{d})^2 = 4d\]
	
		Ist $d \in \nsquare_1$ (d.h. $d \equiv_{(4)} 1$), so ist 	
			\[\brac{\frac{1+\sqrt{d}}{2}}^2 = \frac{1+2\sqrt{d} + d}{4} = \frac{1+\sqrt{d}}{2} + \frac{d-1}{4}\]
		eine Ganzheitsgleichung f\"ur $\frac{1+\sqrt{d}}{2}$ und daher $\Z[\frac{1+\sqrt{d}}{2}] = \Z \oplus \Z\frac{1+\sqrt{d}}{2}$
		ein multiplikatives Gitter in $\Q(\sqrt{d})$, dessen Diskriminante 
		\[\brac{\det \begin{pmatrix}
				1 & \frac{1+\sqrt{d}}{2} \\ 1^* & \brac{\frac{1+\sqrt{d}}{2}}^2
		\end{pmatrix}}^2 = \brac{\frac{1-\sqrt{d}}{2} - \frac{1+\sqrt{d}  }{2}  }^2 = d\]
		ist.
		
		F\"ur $d \in \nsquare$ ergibt sich also mit \ref{3.1.8} ein neuer Beweis f\"ur die in \ref{2.1.17} schon bewiesene Tatsache
		$\mathcal{O}_d = \Z[\frac{1+\sqrt{d}}{2}]$.
		
		F\"ur $d \in \nsquare_{2,3}$ liefert \ref{3.1.7} die (auch sonst leicht zu sehende) Tatsache 
		\[[(\Z+\Z\frac{1+\sqrt{d}}{2}): \Z[\sqrt{d}]] = \sqrt{\frac{d(\Z[\sqrt{d}])}{d(\Z + \Z\frac{1+\sqrt{d}}{2})}} = \sqrt{\frac{4d}{d}} = 2\]
	\end{beispiel}
	\newpage
	
	\section{Zerlegung von Primzahlen in Zahlringen}\thispagestyle{sectionstart}
	
	\begin{bemerkung}\label{3.2.1}
		Ein wesentlicher Grund f\"ur die Betrachtung von Gittern und vor allem von multiplikativen Gittern ist, dass sie oftmals " einfacher" sind als der Zahlring (zum Beispiel ist f\"ur $d \in \nsquare_{1}$ das multiplikative Gitter $\Z[\sqrt{d}]$ " einfacher" als $\Z[\frac{1+\sqrt{d}}{2}] = \mathcal{O}_d$) und f\"ur gewisse Zwecke doch den Zahlring ersetzen k\"onnen.
		Siehe Zeile (b) und (c) dieser Bemerkung und Satz \ref{3.2.2} unten.
		
		Seien $K$ ein Zahlk\"orper und $x_1, \dots, x_n$ eine $\Z$-Basis von $\mathcal{O}_K$.
		\begin{enumerate}[(a)]
			\item Sei $I \neq (0)$ ein Ideal von $\mathcal{O}_K$. Nach Lemma \ref{2.5.3} gilt $I \cap \Z \neq (0)$, das hei\ss t es gibt ein eindeutig bestimmtes $m \in \N$ mit $I \cap \Z = (m)$.
			
			Insbesondere gilt $m \mathcal{O}_K \subseteq I$ und man kann $I$ sehen als $m$ zusammen mit dem Bild von $I$ unter $\mathcal{O}_K \to \mathcal{O}_K/m\mathcal{O}_K$.
			
			ein Ideal $\neq (0)$ des Zahlringes $\mathcal{O}_K$ ist also gegeben durch eine nat\"urliche Zahl $m$ und ein Ideal des endlichen Ringes $\mathcal{O}_K/m\mathcal{O}_K = (\Z x_1 \oplus \cdots \oplus \Z x_n) / (m\Z x_1 + \cdots + m\Z x_n)$,
			dessen additive Gruppe in nat\"urlicher Weise ein freier $\Z/m\Z$-Modul mit Basis $\overline{x_1}, \dots, \overline{x_n}$ ist.
			
			Insbesondere ist $\mathcal{O}_K /I$ endlich mit $\#(\mathcal{O}_K / I) | m^n$
		\end{enumerate}
	\end{bemerkung}
\backmatter
\printindex
\end{document}