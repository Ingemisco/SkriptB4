\documentclass{book}

\usepackage{amsmath}
\usepackage{amssymb}
\usepackage{enumerate}
\usepackage{inputenc}
\usepackage[german]{babel}
\usepackage{fancyhdr}
\usepackage[pass]{geometry}
\usepackage{csquotes}
\usepackage{hyperref}
\MakeOuterQuote{"}

\newcommand{\N}{\mathbb{N}}
\newcommand{\Z}{\mathbb{Z}}
\newcommand{\Q}{\mathbb{Q}}
\newcommand{\R}{\mathbb{R}}
\newcommand{\F}{\mathbb{F}}

\newcommand{\comment}[1]{}

\newcommand{\brac}[1]{\left( #1 \right)}
\newcommand{\bracB}[1]{\left[ #1 \right]}
\newcommand{\bracC}[1]{\left< #1 \right>}
\newcommand{\abs}[1]{\left| #1 \right|}
\newcommand{\set}[1]{\left\{ #1 \right\}}

\pagestyle{fancy}
\fancyhf{}

\renewcommand{\headrulewidth}{2pt}

\begin{document}
	\begin{titlepage}
		\newgeometry{margin=3cm}
		\centering
		
		\vfill
		
		\Large Algebra II
		
		\vfill

	\end{titlepage}
	
	
	\chapter*{1 Moduln}\addcontentsline{toc}{chapter}{1 Moduln}
	
	
	\section*{1.1 Definitionen und grundlegende Tatsachen}\addcontentsline{toc}{section}{1.1 Definitionen und grundlegende Tatsachen}
	
	\subsection*{1.1.1 Definition}\addcontentsline{toc}{subsection}{1.1.1 Definition Modul}
	Ein {\bf Modul} ist ein Tupel $(R, +_R, \cdot_R, M, +, \cdot)$, wobei $(R, +_R, \cdot_R)$ ein Ring (mit $1$, nicht notwendigerweise kommutativ), $(M, +)$ eine abelsche Gruppe und \\\noindent$\cdot:R\times M \to M$ eine (meist gar nicht oder infix geschriebene) Abbildung mit folgenden Eigenschaften
	
	\begin{itemize}
		\item[$(\overset{\rightarrow}{D})$] $\forall a \in R: \forall x, y \in M: a(x + y) = ax + ay$ \hfill "distributiv"
		
		\item[$(D')$] $\forall a, b \in R: \forall x \in M: (a+b)x = ax + bx$ \hfill "distributiv"
		
		\item[$(N)$] $\forall x \in M: 1_R \cdot x = x$ \hfill "normiert"
		
		\item[$(V)$] $\forall a, b \in R: \forall x \in M: (ab)x = a(bx)$ \hfill "vertr\"aglich"
	\end{itemize}
	
	
	\subsection*{1.1.2 Bemerkung}\addcontentsline{toc}{subsection}{1.1.2 Bemerkung}
	
	\begin{enumerate}[(a)]
		\item Schlampiger Sprachgebrauch: 
		\begin{itemize}
			\item "Sei $M$ ein $R$-Modul" statt "Sei $(R, +_R, \cdot_R, M, +, \cdot)$ ein Modul"
			
			\item "Sei $M$ ein Modul" statt "Es gebe einen Ring $R$ so, dass $M$ ein $R$-Modul ist"
		\end{itemize}
		
		\item Statt "$R$-Modul" sagt man auch "Modul \"uber $R$"	
		
		\item Vektorr\"aume sind Moduln \"uber K\"orper. Viele Sprechweisen (wie "Skalar", "Linearkombination", nicht jedoch "Vektor") \"ubertragen wir stillschweigend von Vektorr\"aumen auf Moduln, ebenso 
		Konventionen (wie "Punkt vor Strich").
		
		\item Abelsche Gruppen "sind" $\Z$-Moduln. Sei $G$ eine abelsche Gruppe. Dann gibt es genau eine Skalarmultiplikation $\cdot:\Z\times G \to G$ verm\"oge derer $G$ zu einem $\Z$-Modul wird, n\"amlich die nat\"urliche, die durch 
			\[n \cdot a := \begin{cases}
				\underbrace{a + a + \cdots + a}_{n\textrm{-mal}} & \textrm{falls } n > 0\\
				0 & \textrm{falls } n = 0\\
				\underbrace{-a - a - \cdots - a}_{(-n)\textrm{-mal}} & \textrm{falls } n < 0
			\end{cases}\]
		gegeben ist.
		
		\item $(\overset{\rightarrow}{D})$ besagt, dass f\"ur alle $a \in R$ die Abbildung $M \to M, x \mapsto ax$ ein Gruppenhomomorphismus ist. Insbesondere gilt $a \cdot 0 = 0$ und $a \cdot (-x) = -ax$ f\"ur alle $a \in R, x \in M$.
		
		$(D')$ besagt, dass f\"ur alle $x \in M$ die Abbildung $R \to M, a \mapsto ax$ ein Gruppenhomomorphismus ist. Insbesondere gilt $0 \cdot x = 0$ und $(-a) \cdot x = -ax$ f\"ur alle $a \in R, x \in M$.
	\end{enumerate}

	\subsection*{1.1.3 Beispiele}\addcontentsline{toc}{subsection}{1.1.3 Beispiele}
	\begin{enumerate}[(a)]
		\item Nullmoduln $\set{0}$
		
		\item Sei $A$ ein Unterring des Ringes $B$. Dann ist $B$ ein $A$-Modul verm\"oge der Skalarmultiplikation $\cdot: A \times B \to B, (a, x) \mapsto ax$
		
		Insbesondere ist jeder Ring ein Modul \"uber sich selbst.
		
		\item Sei $R$ ein kommutativer Ring und $n \in \N_0$. Dann wird die abelsche Gruppe $R^n$ zu einem $R^{n\times n}$-Modul verm\"oge der Skalarmultiplikation
			\[\cdot: R^{n\times n} \times R^n \to R^n, (A, x) \mapsto Ax\]
		Dies folgt aus den Rechenregeln f\"ur Matrixmultiplikation.
	\end{enumerate}

	\subsection*{1.1.4 Definitionen, Propositionen, S\"atze und Notationen}\addcontentsline{toc}{subsection}{1.1.4 Definition, Propositionen, S\"atze und Notationen}
	Sei $R$ ein Ring. Die folgenden f\"ur die Theorie der $R$-Moduln grundlegenden Begriffe und Resultate sind eine direkte Verallgemeinerung der entsprechenden Tatsachen f\"ur Vektorr\"aume (also f\"ur den Fall, dass $R$ ein K\"orper) und f\"ur abelsche Gruppen (also $R = \Z$) aus der Linearen Algebra:
	
	\begin{enumerate}[(a)]
		\item Genauso wie bei Vektorr\"aumen f\"uhrt man {\bf direkte Produkte} von $R$-Moduln ein.
		
		\item Sind $M$ und $N$ $R$-Moduln, so hei\ss t $N$ ein {\bf Untermodul} von $M$, wenn die $N$ zugrunde liegende abelsche Gruppe eine Untergruppe der $M$ zugrunde liegenden abelschen Gruppe ist und 
			\[\forall a \in R: \forall x \in M: a \cdot_N x = a \cdot_M x\]
			
		Ein Untermodul eines Moduls ist offenbar durch seine Tr\"agermenge (d.h. seine zugrunde liegende Menge) eindeutig bestimmt.
		
		Ist $M$ ein $R$-Modul und $N \subseteq M$, so ist $N$ offenbar genau dann (Tr\"agermenge) ein(e) Untermodul(s) von $M$, wenn $0\in N, \forall x, y\in N: x + y \in N, \forall a \in R: \forall x \in N: ax \in N$
		
		\item Sei $M$ ein Modul und $(N_i)_{i \in I}$ eine Familie von Untermoduln von $M$. Dann ist 
		$\bigcap_{i \in I} N_i := \bigcap \set{N_i | i \in I}$ (mit $\bigcap_{i \in I} N_i = M$, falls $I = \emptyset$)
		wieder ein Untermodul von $M$ und zwar der gr\"o\ss te Untermodul von $M$, der in allen $N_i$ enthalten ist.
		
		Weiter ist auch $\sum_{i \in I} N_i := \set{\sum_{i \in I} x_i | (x_i)_{i \in I} \in \prod_{i\in I} N_i, \set{i \in I| x_i \neq 0} \textrm{endlich}}$ Untermodul von $M$ und zwar der kleinste Untermodul von $M$, der alle $N_i$ enth\"alt.
		
		\item Sei $M$ ein $R$-Modul. Ist $x \in M$, so ist $Rx := \set{ax| a \in R}$ ein Untermodul von $M$ und zwar der kleinste Untermodul, der $x$ enth\"alt.
		
		Ist $(x_i)_{i \in I}$ eine Familie von Elementen von $M$, so ist $\sum_{i \in I} Rx_i$ der kleinste Untermodul von $M$, der alle $x_i$ enth\"alt.
		
		Man nennt ihn den von den $x_i$ ($i \in I$) (oder $\set{x_i|i \in I}$) erzeugten Untermodul von $M$ (oder lineare H\"ulle der Span von $\set{x_i| i \in I}$). 
		
		Man nennt $M$ {\bf zyklisch}, wenn $M$ von einem Element erzeugt wird, d.h. es ein $x \in M$ gibt mit $M = Rx$. Man nennt $M$ endlich erzeugt (e.e.), wenn $M$ von endlich vielen Elementen
		erzeugt wird, d.h. es ein $n \in \N_0$ und $x_1, \dots, x_n \in M$ gibt mit 
		\[M = Rx_1 + \cdots + Rx_n := \sum_{i=1}^{n} Rx_i := \sum_{i \in \set{1, \dots, n}} Rx_i\] 
	\end{enumerate}


	\tableofcontents
\end{document}